% sll-triptic.tex (RMCG20120317)

\font\ptitlefont=texnansi-lmssbx10 at 90pt
\font\psubtitlefont=texnansi-lmss10 at 40pt
\font\pauthorfont=texnansi-lmss10 at 50pt

\font\spinetitlefont=texnansi-lmssbx10 at 24pt
\font\spineauthorfont=texnansi-lmss10 at 24pt

\font\pcitefont=texnansi-lmss10 at 21pt
\font\ptextfont=texnansi-lmss10 at 14pt

\ansifont

\def\pdfBlack{\pdfliteral{0 0 0 1 k 0 0 0 1 K}}
\def\pdfWhite{\pdfliteral{0 0 0 0 k 0 0 0 0 K}}
\def\pdfRed{\pdfliteral{0 1 1 0 k 0 1 1 0 K}}
\def\pdfBlue{\pdfliteral{1 1 0 0 k 1 1 0 0 K}}
\def\pdfGreen{\pdfliteral{1 0 1 0 k 1 0 1 0 K}}

\pdfpageheight=9.5in \voffset=-1in 
\pdfpagewidth=12.9in \hoffset=-1in

\newbox\background
\setbox\background=\hbox{\pdfGreen\vrule width 6.25in height 0pt depth 9.5in}
\wd\background=0pt 
\ht\background=0pt
\dp\background=0pt

\newif\iftesting % \testingtrue

\iftesting \showboxbreadth=10 \showboxdepth=6 \fi

%%%%%%%%%%%%% Portada

\newbox\portada
\setbox\portada=\vbox to 9.5in{\copy\background
 \vss\hbox to 6.25in{
 \vbox{\hsize=6in
  \pdfWhite
  \hbox{}
  \vskip12pc
  \centerline{\ptitlefont Sobre la}
  \vskip1pc
  \centerline{\ptitlefont libertad}
  \vskip10pc
  \centerline{\psubtitlefont La teoría de la}
  \vskip6pt
  \centerline{\psubtitlefont subjetividad}
  \vskip2pc
  \centerline{\pauthorfont Ramón Casares}
 \vss}\hss}\kern0.25in\vss}

\iftesting \showbox\portada \fi
\message{Portada (\the\wd\portada, \the\ht\portada, \the\dp\portada)}

%%%%%%%%%%%%% Lomo

\def\rotatebox#1{%
 \wd#1=0pt \ht#1=0pt \dp#1=0pt
 \pdfliteral{q}% \pdfsave
 \pdfliteral{0 -1 1 0 0 0 cm}% rotate -90º
 \box#1%
 \pdfliteral{Q}% \pdfrestore
}

\newbox\lomotext
\setbox\lomotext=\vbox{\hbox to 9.5in{\pdfGreen
 \vrule width 9.5in height 0.31in depth 0.09in
 \kern -7.25in \pdfWhite
  \spinetitlefont Sobre la libertad\hss
  \spineauthorfont Ramón Casares\kern24pt
  \kern0.25in}\nointerlineskip\hbox{}}

\message{Lomox (\the\wd\lomotext, \the\ht\lomotext, \the\dp\lomotext)}

\newbox\lomo
\setbox\lomo=\vbox to 9.5in{\hbox to 0.4in{\hss\rotatebox\lomotext}\vss}

\message{Lomoy (\the\wd\lomo, \the\ht\lomo, \the\dp\lomo)}

\wd\lomo=0.4in
\dp\lomo=0pt
\ht\lomo=9.5in

\iftesting \showbox\lomo \fi
\message{Lomo (\the\wd\lomo, \the\ht\lomo, \the\dp\lomo)}

%%%%%%%%%%%%% Contraportada

\newbox\contra
\setbox\contra=\vbox to 9.5in{\copy\background
 \vss\hbox to 6.25in{
 \hss\vbox to 9in{\hsize=5.5in
  \pdfWhite
  \hbox{}
  \kern1pc
  \begingroup\pcitefont\parindent=0pt\baselineskip=24pt\obeylines
   \kern12pt
   ``La vida es una burbuja de conocimiento
   \leavevmode\kern12pt y de libertad'' {\ptextfont(§154 ¶2)}
   \kern12pt
  \endgroup
\vskip 1pc
\begingroup\ptextfont\parindent=0pt\baselineskip=16pt\obeylines

El sujeto es la base de la teoría del conocimiento,
y, sin embargo, apenas se le ha prestado atención.
La causa de este desinterés es, seguramente, que,
como ya vio Descartes, el sujeto es la verdad indubitable;
siendo yo mismo, ¿cómo no voy a saberlo todo sobre el sujeto?
\null
Pero esta renuncia a estudiar al sujeto del conocimiento
produce anomalías que alcanzan tamaños enormes.
Por un lado, la ciencia prescinde por completo del sujeto, porque
postula que en las leyes de la naturaleza no cabe libertad alguna.
Y, por el otro lado, el arte aspira a la creatividad absoluta,
lo que nos descubre que el arte da por hecho que
el sujeto disfruta de una libertad sin límites.
\null
``Sobre la libertad'' presenta una teoría del sujeto
para remediar tan disparatada situación. Primero
reconstruye el proceso evolutivo que ha producido los sujetos.
Entonces propone considerar que la vida es,
desde el punto de vista epistemológico, un problema,
y analiza las consecuencias. Una es que
la resolución de tal problema coincide
con el proceso evolutivo, lo que legitima la hipótesis.
Las otras consecuencias son muy numerosas y variadas,
y le dejo a usted que las evalúe.
Eso sí, para juzgarlas ha de leer este libro.

\endgroup
\vfil
\pdfBlack
\input ean13
\line{
\pdfximage{cc-by-sa-www.pdf}\pdfrefximage\pdflastximage
\hfil
\pdfliteral{q}% save current graphic state in the stack
  \pdfliteral{0 0 0 0 k 0 0 0 1 K}% set black stroke on white fill
  \pdfliteral{-5 -5 m}% moves to the origin
  \pdfliteral{110 -5 l}%
  \pdfliteral{110 78 l}% 92 -> 78
  \pdfliteral{-5 78 l}%
  \pdfliteral{-5 -5 l}% 
  \pdfliteral{b Q}% close, stroke, fill, and restore graphic state
\ISBN 978-1-4750-4270-2
\barheight=2cm % this must be after \ISBN call
\EAN 978-1475042702
\hskip18pt
}\kern0.2in}\kern0.1in}\vss}

%\showbox\contra
\message{Contra (\the\wd\contra, \the\ht\contra, \the\dp\contra)}

%%%%%%%%%%%%% Tríptico

\iftesting
 \shipout\hbox{\copy\portada}
 \shipout\hbox{\copy\lomo}
 \shipout\hbox{\copy\contra}
\fi

\newbox\tripbox
\iftesting
 \setbox\tripbox=\hbox to \pdfpagewidth{\box\contra\hss\box\lomo\hss\box\portada}
 \showbox\tripbox
 \shipout\box\tripbox
\else
 \shipout\hbox to \pdfpagewidth{\box\contra\hss\box\lomo\hss\box\portada}
\fi

\end

