% APE7.TEX (RMCG20020501)


\def\siglo#1 {\par\noindent\hang {\bf #1}\space\ignorespaces}

\Section ¡Arriba el subjetivismo!

\siglo XVII ^[Descartes] estableció en el siglo XVII los fundamentos de
la filosofía moderna: lo único indudable es el yo. Además señaló la
naturaleza irreductible de la realidad y de la libertad, que resolvió
con un dualismo ontológico.

\siglo XVIII ^[Kant] adviertió que, previo al entender, es preciso un
aparato que entienda, al que aquí denominamos lógica. La lógica
kantiana, sin embargo, sólo era capaz de representar la realidad, y no
la libertad.

\siglo XIX ^[Darwin] postuló que el hombre, \latin{homo sapiens}, es
producto de la evolución de las especies. En consecuencia, también lo ha
de ser su lógica y su yo. E incluso su libertad.

\siglo XX ^[Gödel] sintactizó las matemáticas y ^[Turing] inventó el motor
sintáctico, aportando las claves de los simbolismos. Un simbolismo es
una lógica gramatical y recursiva, esto es, una lógica de expresividad
máxima y dividida en dos capas denominadas semántica y sintaxis.

\siglo XXI Ahora nos queda simplemente la tarea de integrar los
descubrimientos de los cuatro siglos anteriores.

Para ello sólo hay que sustituir la lógica de ^[Kant] por un simbolismo,
con lo que el dualismo ontológico de ^[Descartes] se transforma en un
dualismo lógico. No hay dos sustancias, sino dos tipos de objetos
lógicos: los semánticos, que son las cosas reales que se ven sin
necesidad de pensar, y sintácticos, que son los conceptos teóricos que
hay que pensar y que no se ven.

La explicación del dualismo es histórica y contingente, es decir, es la
que proporciona ^[Darwin]. Para adecuarnos a la explicación darviniana
hemos de mostrar que el simbolismo mejora las posibilidades de
supervivencia. Y esto es así si definimos la vida como un problema
aparente, o sea, como libertad y condición, y nada más. Porque, para que
un resolutor de un problema aparente sea capaz de representarse la
situación completa, incluido el problema con su libertad y él mismo como
resolución, su lógica ha de ser simbólica.
