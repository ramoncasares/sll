% APE7.TEX (RMCG20020501)


\def\siglo#1 {\par\noindent\hang {\bf #1}\space\ignorespaces}

\Section �Arriba el subjetivismo!

\siglo XVII ^[Descartes] estableci� en el siglo XVII los fundamentos de
la filosof�a moderna: lo �nico indudable es el yo. Adem�s se�al� la
naturaleza irreductible de la realidad y de la libertad, que resolvi�
con un dualismo ontol�gico.

\siglo XVIII ^[Kant] advierti� que, previo al entender, es preciso un
aparato que entienda, al que aqu� denominamos l�gica. La l�gica
kantiana, sin embargo, s�lo era capaz de representar la realidad, y no
la libertad.

\siglo XIX ^[Darwin] postul� que el hombre, \latin{homo sapiens}, es
producto de la evoluci�n de las especies. En consecuencia, tambi�n lo ha
de ser su l�gica y su yo. E incluso su libertad.

\siglo XX ^[G�del] sintactiz� las matem�ticas y ^[Turing] invent� el motor
sint�ctico, aportando las claves de los simbolismos. Un simbolismo es
una l�gica gramatical y recursiva, esto es, una l�gica de expresividad
m�xima y dividida en dos capas denominadas sem�ntica y sintaxis.

\siglo XXI Ahora nos queda simplemente la tarea de integrar los
descubrimientos de los cuatro siglos anteriores.

Para ello s�lo hay que sustituir la l�gica de ^[Kant] por un simbolismo,
con lo que el dualismo ontol�gico de ^[Descartes] se transforma en un
dualismo l�gico. No hay dos sustancias, sino dos tipos de objetos
l�gicos: los sem�nticos, que son las cosas reales que se ven sin
necesidad de pensar, y sint�cticos, que son los conceptos te�ricos que
hay que pensar y que no se ven.

La explicaci�n del dualismo es hist�rica y contingente, es decir, es la
que proporciona ^[Darwin]. Para adecuarnos a la explicaci�n darviniana
hemos de mostrar que el simbolismo mejora las posibilidades de
supervivencia. Y esto es as� si definimos la vida como un problema
aparente, o sea, como libertad y condici�n, y nada m�s. Porque, para que
un resolutor de un problema aparente sea capaz de representarse la
situaci�n completa, incluido el problema con su libertad y �l mismo como
resoluci�n, su l�gica ha de ser simb�lica.
