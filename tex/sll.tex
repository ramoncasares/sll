% sll.tex (RMCG20000104)

%\mag1200 % OPTIONAL MAGNIFICATION

\input explain
\input fonts
\input index
\input metatex

%\input crops % OPTIONAL CROPS (Beware the next line!)
\newdimen\hpage \newdimen\vpage % already defined in crops.tex

\def\[{\string\[} % to allow ^álgebra de \[Boole]^
%%%\files

% Fonts

% Fonts for titles

\font\fonttwosym=cmbsy10 scaled\magstep1
\def\Stitle{{\fonttwosym \char"78}}

\lmfonts\lmtitles\ansifont
 \font\tenrm=rm-lmr10 \textfont0=\tenrm
\font\frakx=eufm10
\font\frakvii=eufm7
\newfam\frakfam \textfont\frakfam=\frakx \scriptfont\frakfam=\frakvii
\def\frak{\fam\frakfam\frakx}

\font\msbm=msbm10
\font\logo=logo10

\catcode`\@=11

% Layout

\let\docinfo\relax \let\infodoc\relax

%\nopagenumbers
\headline={\hfil}
\footline={\tenrm\ifodd\pageno \docinfo\hfil\folio
            \else \folio\hfil\infodoc \fi\strut}

\hpage=6in \hsize=10.5cm
 \hoffset\hpage \advance\hoffset-\hsize \divide\hoffset2
 \advance\hoffset-1in
\vpage=9in \vsize=39\baselineskip \advance\vsize\topskip % 39 + 1 lines
 \voffset\vpage \advance\voffset-\vsize \advance\voffset-2\baselineskip
 \divide\voffset2 \advance\voffset-1in

\pdfcode \pdfpageheight=\vpage \pdfpagewidth=\hpage \pdfendcode

%\def\titulo{Ramón Casares\cr Sobre la libertad}
%\setcrops % OPTIONAL (Beware the top of this file!)

\parindent=20pt

\predisplaypenalty=10
\abovedisplayskip=12pt plus 6pt minus 6pt
\belowdisplayskip=12pt plus 6pt minus 6pt
\abovedisplayshortskip=0pt plus 6pt minus 6pt
\belowdisplayshortskip=12pt plus 6pt minus 6pt


% Every point is a period
\count255=`A \loop\sfcode\count255=1000
 \ifnum\count255<`Z\advance\count255 1 \repeat
\def~{\nobreak\ \ignorespaces}

% Emphasis with automatic italic correction (\/).
% Use: {\em italic, but {\em roman}, text}.
\def\em{\ifdim \fontdimen1\font>0pt \rm
 \else \it \expandafter\aftergroup \fi \itcor}
\def\itcor{\ifhmode \expandafter\itpuncl@ok \fi}
\def\itpuncl@ok{\begingroup\futurelet\ITCt@mpa\itcort@st}
\def\itcort@st{\def\ITCt@mpb{\ITCt@mpa}%
 \ifcat\noexpand\ITCt@mpa,\setbox0=\hbox{\ITCt@mpb}%
  \ifdim\ht0<0.3ex \let\itc@rdo=\endgroup \fi\fi \itc@rdo}
\def\itc@rdo{\skip0=\lastskip \ifdim\skip0=0pt \/\else
 \unskip \/\hskip\skip0 \fi \endgroup}

% Verbatim

\def\uncatcodeall{\deactivate\dospecials\dohigh}
\def\iverb#1{\begingroup\uncatcodeall\obeyspaces\obeylines\d@iverb#1}
\def\d@iverb#1#2{\def\next##1#2{\immediate\write#1{##1}\endgroup}\next}

% Other

\def\siglo#1 {\par \noindent\hang {\sc #1} $\cdot$ \ignorespaces}

\def\point{\par\hang\indent\llap{$\bullet$\enspace}}
\def\goodbreak{\vskip0pt plus 4\baselineskip\penalty-250
 \vskip0pt plus-4\baselineskip}

\def\breakif#1{\vskip0pt plus #1\baselineskip \penalty-250
 \vskip0pt plus-#1\baselineskip}


% INDEX

% PDFTeX

\def\stringactives{\stringate\doaccents\dosymbols}


%%% OPTION
\nocolor % default value
%\colors


% Numbered footnotes
\newcount\footnoteno
\saveplain\footnote
\def\footnote{\global\advance\footnoteno1
 \plain\footnote{$^{\number\footnoteno}$}}
\def\footnoteoptions{\everypar{}\parindent=20pt}

% OUTPUT ROUTINE

\def\pagecontents{\dimen@=\dp\strutbox \advance\dimen@-\dp\@cclv
 \advance\dimen@\ht\strutbox \advance\dimen@\dp\strutbox
 \ifnum\pageno=136 \global\tracingpages=1
  \global\showboxbreadth=\maxdimen \global\showboxdepth=1 \fi
 \ifnum\pageno=138 \global\tracingpages=0
  \global\showboxbreadth=2 \global\showboxdepth=2 \fi
 \unvbox\@cclv
 \ifvoid\footins\else\kern\dimen@\footnoterule\unvbox\footins\fi}

%\saveplain\pagebody
%\def\pagebody{\hbox{\plain\pagebody
% \ifvoid\margin\else\kern3pt\vbox to0pt{\vss\unvbox\margin}\fi}}



% Sectioning

\outer\def\Part#1 \par{\vfill\break \null\setdest
 \nointerlineskip\vbox to7\baselineskip{\vfil
  \centerline{\fontone #1\label{#1}\toc0{#1}}\vfil}}


\newcount\secno \secno=0
\newcount\parno \parno=0

\def\section{\number\secno}

\def\Section#1 \par{\everypar{}\parno=0  \advance\secno1 
 \vskip0pt plus 4\baselineskip\penalty-43
 \vskip0pt plus-4\baselineskip \vskip\baselineskip
 \titleline{\noindent\Stitle\fonttwo\section\space}%
   {\fonttwo#1\pdflabel\toc1{#1}\lbl{#1}{\section}}%
 \everypar{\numberedpars}\nobreak}

\def\titleline#1#2{\setbox0\hbox{#1}\dimen0=\hsize \advance\dimen0 -\wd0
 \line{\box0\hss\vtop{\hsize=\dimen0 \raggedright\let\\=\ \noindent #2}}}

\newdimen\oldparindent \oldparindent=\parindent \parindent=0pt

\def\numberedpars{\global\advance\parno1 \pdflabel
 \noindent\hbox to\oldparindent{{\sevensy\char"7B %" little par sign
  \teni\number\parno}\hfil$\cdot$\hfil}\ignorespaces}




\newcount\xs@one

\outer\def\xSection#1 \par{\vskip0pt plus 4\baselineskip\penalty-250
 \vskip0pt plus-4\baselineskip
 \vskip\baselineskip \advance\s@one1 \setdest
 \noindent{\fonttwo\Stitle\the\s@one\space#1\label{#1}}\toc1{#1}\par
 \nobreak}

% Indexing

%\newcount\d@stno \newcount\p@rno \d@stno=1 \p@rno=1

%\def\setdest{\global\advance\d@stno1 \global\p@rno=\d@stno
% \pdfcode \pdfdest num \number\d@stno fitbh\pdfendcode}
\let\setdest=\pdflabel

\def\destpar{\ifhmode \let\next=\pdflabel \else \let\next=\relax \fi
 \endgraf \next}

%%%\def\stringall{\stringate\dohigh\stringaccents}

\def\label#1{\pdflabel\lbl{#1}{\number\secno}}
\def\index#1{\pdflabel\ndx{#1}{1}}

\files

% end index

% DIAGRAMS

\def\onitself#1{\leavevmode\vbox{
 \baselineskip=0pt \lineskip=0.25ex \everycr={}\tabskip=0pt
 \halign{\hfil##\hfil\cr\msbm\char"78\cr#1\crcr}}}
\def\column#1{\leavevmode\vtop{
 \baselineskip=0pt \lineskip=0.25ex \everycr={}\tabskip=0pt
 \halign{\hfil##\hfil\cr#1\crcr}}}
\def\base#1{\leavevmode\vbox{
 \baselineskip=0pt \lineskip=0.25ex \everycr={}\tabskip=0pt
 \halign{\hfil##\hfil\cr#1\crcr}}}

\def\subject{\leavevmode
 \vbox{\everycr={}\tabskip=0pt \lineskip=0pt
  \halign{\hfil##\hfil\cr
   \hidewidth\strut Idea\hidewidth\cr
   \hfil\hfil\vbox{\hbox{\big\downarrow}\nointerlineskip\null}\hfil
   \hbox{\msbm\char"78}\hfil
   \vbox{\hbox{\big\uparrow}\nointerlineskip\null}\hfil\hfil\cr
   \noalign{\kern-3pt}
   \column{
    \strut Objeto\cr
    $\uparrow$\cr
    \hidewidth\strut Sentimiento\hidewidth\cr
    $\uparrow$\cr
    \strut Libido\cr}\cr}}}

% Bibliography

\def\cite#1{\ndx{#1}0\footnote{\ref{#1}.}}
\def\citation#1{\footnote{{\def\cite##1{\ndx{##1}0\ref{##1}}#1}}}

\newtoks\TITLE \newtoks\SUBTITLE \newtoks\AUTHOR \newtoks\YEAR
\newtoks\PUBLISHER \newtoks\ADDRESS \newtoks\YEAREDITION\newtoks\ISBN
\newtoks\EDITOR \newtoks\BOOKTITLE \newtoks\EDITION \newtoks\NOTE
\newtoks\JOURNAL \newtoks\VOLUME \newtoks\MONTH \newtoks\PAGES
\def\cleantoks{%
 \TITLE={}\SUBTITLE={}\AUTHOR={}\YEAR={}%
 \PUBLISHER={}\ADDRESS={}\YEAREDITION={}\ISBN={}%
 \EDITOR={}\BOOKTITLE={}\EDITION={}\NOTE={}%
 \JOURNAL={}\VOLUME={}\MONTH={}\PAGES{}}

% METATeX

\def\MTendmark{:::}

\MTcode
pickup pencircle scaled 0.3pt; thin_pen:=savepen;
pickup pencircle scaled 0.6pt; med_pen:=savepen;
pickup pencircle scaled 1.2pt; thick_pen:=savepen;

arrow_head_length := 6pt; arrow_head_width := 2.4pt;
point_diameter := 2.4pt; aperture := 3pt; jot := 2pt;
u := 1pt; v := 1pt;

def rectangle(suffix s)(expr width,height) =
 x.s.l = x.s - width/2;
 x.s.r = x.s + width/2;
 y.s.t = y.s + height/2;
 y.s.b = y.s - height/2;
 draw (x.s.l,y.s.b) -- (x.s.r,y.s.b) --
  (x.s.r,y.s.t) -- (x.s.l,y.s.t) -- cycle;
enddef;

def square(suffix s)(expr side) =
 x.s.l = x.s - side/2;
 x.s.r = x.s + side/2;
 y.s.t = y.s + side/2;
 y.s.b = y.s - side/2;
 draw (x.s.l,y.s.b) -- (x.s.r,y.s.b) --
  (x.s.r,y.s.t) -- (x.s.l,y.s.t) -- cycle;
enddef;

def circle(suffix s)(expr diameter) =
 x.s.l = x.s - diameter/2;
 x.s.r = x.s + diameter/2;
 y.s.t = y.s + diameter/2;
 y.s.b = y.s - diameter/2;
 draw (fullcircle scaled diameter shifted z.s);
enddef;

% point_diameter;

def point(suffix s) =
 fill fullcircle scaled point_diameter shifted z.s;
enddef;

% arrow_head_length, arrow_head_width;

def arrowhead(suffix orig,dest) =
 z.dest.head  = (arrow_head_length/length(z.dest-z.orig))[z.dest,z.orig];
 z.dest.right = z.dest.head + arrow_head_width / 2 *
                dir(angle(z.dest-z.orig)+90);
 z.dest.left  = z.dest.head + arrow_head_width / 2 *
                dir(angle(z.dest-z.orig)-90);
 fill z.dest -- z.dest.right -- z.dest.left -- cycle;
enddef;

def arrow(suffix orig,dest) =
 arrowhead(orig,dest); draw z.orig .. z.dest.head;
enddef;

% aperture;

def soft(suffix orig,med,dest) =
 z.med.o = (aperture/length(z.med-z.orig))[z.med,z.orig];
 z.med.d = (aperture/length(z.med-z.dest))[z.med,z.dest];
 draw z.orig --- z.med.o ... z.med.d --- z.dest;
enddef;

def softt(suffix orig,medone,medtwo,dest) =
 z.medone.o = (aperture/length(z.medone-z.orig))[z.medone,z.orig];
 z.medone.d = (aperture/length(z.medone-z.medtwo))[z.medone,z.medtwo];
 z.medtwo.o = (aperture/length(z.medtwo-z.medone))[z.medtwo,z.medone];
 z.medtwo.d = (aperture/length(z.medtwo-z.dest))[z.medtwo,z.dest];
 draw z.orig --- z.medone.o ... z.medone.d
  --- z.medtwo.o ... z.medtwo.d --- z.dest;
enddef;

def fork(suffix orig,med,dest) =
 arrowhead(med,dest); point(orig); soft(orig,med,dest.head);
enddef;

def arroww(suffix orig,med,dest) =
 arrowhead(med,dest); soft(orig,med,dest.head);
enddef;

def arrowww(suffix orig,medone,medtwo,dest) =
 arrowhead(medtwo,dest); softt(orig,medone,medtwo,dest.head);
enddef;

def back(suffix s,o,d)(expr upper,size,margin) =
 pickup thick_pen;
 y.s.del.t = upper; y.s.del.t - y.s.del.b = size;
 y.s.del.l = y.s.del.r = 1/2[y.s.del.b,y.s.del.t];
 x.s.del.r + x.s.del.l = x.o + x.d;
 x.s.del.r - x.s.del.l = size;
 x.s.del.t = x.s.del.b = x.s.del.r;
 draw z.s.del.l -- z.s.del.b -- z.s.del.t -- cycle;
 pickup med_pen;
 z.s.tr = (x.o,y.s.del.r);
 z.s.r =  (x.o+margin,1/2[y.o,y.s.del.r]);
 z.s.tl = (x.d,y.s.del.l);
 z.s.l =  (x.d-margin,1/2[y.d,y.s.del.l]);
 draw z.o {right} .. z.s.r {up} .. z.s.tr {left};
 arrow(s.tr,s.del.r);
 draw z.s.del.l --- z.s.tl {left} .. z.s.l {down} .. z.d {right};
enddef;

def feedback(suffix s,orig,dest)(expr upper,size,margin) =
 z.s.br = z.orig + (margin,0); z.s.bl = z.dest - (margin,0);
 back(s,s.br,s.bl,upper,size,margin);
 draw z.orig .. z.s.br; arrow(s.bl,dest);
enddef;

def arrowlessfeedback(suffix s,orig,dest)(expr upper,size,margin) =
 z.s.br = z.orig + (margin,0); z.s.bl = z.dest - (margin,0);
 back(s,s.br,s.bl,upper,size,margin);
 draw z.orig .. z.s.br; draw z.s.bl .. z.dest;
enddef;

def forkback(suffix s,orig,dest)(expr upper,size,margin) =
 z.s.bl = z.dest - (margin,0); point(orig);
 back(s,orig,s.bl,upper,size,margin); arrow(s.bl,dest);
enddef;

def shortforkback(suffix s,orig,dest)(expr upper,size,margin) =
 z.s.bl = z.dest - (1/2margin,0); point(orig);
 back(s,orig,s.bl,upper,size,margin); arrow(s.bl,dest);
enddef;

:::
% end METATeX

% Maths

\def\Metric{\ifmmode {\cal M}^{\circ} \else ${\cal M}^{\circ}$\fi}
\def\no#1{{\bf #1}}

\def\inmmode$#1${\ifmmode #1\else $#1$\fi}

\def\aut#1{\inmmode$\mathop{\cal #1}\nolimits$}
\def\syn#1{\inmmode$\mathop{\frak #1}\nolimits$}

\def\universo{\vindex{universo}~$\aut U$}
\def\mecanismo{\vindex{mecanismo}~$\aut A_0$}
\def\adaptador{\vindex{adaptador}~$\!\aut A_1$}
 \def\gobernador{\vindex{gobernador}~$\aut G$}
 \def\cuerpo{\vindex{cuerpo}~$\aut B$}
\def\aprendiz{\vindex{aprendiz}~$\aut A_2$}
 \def\modelador{\vindex{modelador}~$\!\aut M$}
 \def\simulador{\vindex{simulador}~$\!\aut S$}
 \def\realidad{\vindex{realidad}~$\aut R$}
\def\conocedor{\vindex{conocedor}~$\!\aut A_3$}
 \def\inteligencia{\vindex{inteligencia}~$\syn A$}
 \def\mente{\vindex{mente}~$\syn M$} % $\Re$
\def\sujeto{\vindex{sujeto}~$\aut A_4$}
 \def\inquisidor{\vindex{inquisidor}~$\syn I$}
 \def\razon{\vindex{razón}~$\syn R$}
 \def\yo{\vindex{yo}~$\syn X$}

\def\Corporal#1$#2#3${\vindex{#1}~${\aut #2}#3$}
\def\corporal#1$#2#3${#1~${\aut #2}#3$}
\def\Mental#1$#2#3${\vindex{#1}~${\syn #2}#3$}
\def\mental#1$#2#3${#1~${\syn #2}#3$}
\def\TM{máquina de \vperson[Turing]%
 \hindex{máquina de \string\string\string\vperson[Turing]}~$\mathop{\frak T}$}
\def\TMes{máquinas de \vperson[Turing]%
 \hindex{máquina de \string\string\string\vperson[Turing]}~$\mathop{\frak T}$}
\def\UTM{máquina universal de \vperson[Turing]%
 \hindex{máquina universal de \string\string\string\vperson[Turing]}~$\mathop{\frak U}$}
\def\UTMes{máquinas universales de \vperson[Turing]%
 \hindex{máquina universal de \string\string\string\vperson[Turing]}~$\mathop{\frak U}$}
\def\procesador$#1${procesador~${\cal P}_{\frak #1}$}
\def\Procesador$#1${\vindex{procesador}~${\cal P}_{\frak #1}$}
\def\UP{\vindex{procesador universal}~${\cal P}_{\frak U}$} % cal

\def\true{\hbox{\sc true}}
\def\false{\hbox{\sc false}}

\def\llave#1{\inmmode$\left\lbrace\vcenter{
 \halign{&\hbox{\rm\strut##}\hfil\cr#1\crcr}}\right.$}
\def\lopen#1#2{\inmmode$\left#1\vcenter{
 \halign{&\hbox{\rm\strut##}\hfil\cr#2\crcr}}\right.$}

\def\etapa{\inmmode$\mapstochar\Rightarrow$} % for the reader's guide
\def\etapa{\inmmode$\mathrel{\vrule height 3.5pt depth-1.5pt}\kern-1pt
  =\kern-1pt\mathrel{\triangleright}$}

% KEYS

\def\meaning#1{{\it#1\aftergroup\itcor}}
\def\latin#1{{\it#1}\itcor}
\def\booktitle#1{{\sl#1}\itcor}

% Index: 0, for cite, 1 for definition, 2 for person, 3 for index.

% VISIBLE       HIDDEN
% ^{index}      _{index}
% ^[Person]     _[Person]
% ^(cite1999)   _(cite1999)
% ^|definition| _|definition|
% ^<booktitle>  _<label>
% ^>ref>        _>ref>   ( V = \S2.1, pág.~123     H = \S2.1 )

\catcode`\^=7 \catcode`\_=8
\def\specialhat{\ifmmode\def\next{^}\else\let\next=\beginvref\fi\next}
\def\specialund{\ifmmode\def\next{_}\else\let\next=\beginhref\fi\next}
\catcode`\^=13 \let^\specialhat \catcode`\_=13 \let_\specialund

\def\beginvref{\futurelet\next\beginvrefx}
\def\beginhref{\futurelet\next\beginhrefx}

\def\beginvrefx{\begingroup
 \ifx\next[\aftergroup\vperson \else
 \ifx\next(\aftergroup\vcite \else
 \ifx\next|\aftergroup\vdefinition \else
 \ifx\next<\aftergroup\vlabel \else
 \ifx\next>\aftergroup\vref \else
 \aftergroup\vindex \fi\fi\fi\fi\fi \endgroup}
\def\beginhrefx{\begingroup
 \ifx\next[\aftergroup\hperson \else
 \ifx\next(\aftergroup\hcite \else
 \ifx\next|\aftergroup\hdefinition \else
 \ifx\next<\aftergroup\hlabel \else
 \ifx\next>\aftergroup\href \else
 \aftergroup\hindex \fi\fi\fi\fi\fi \endgroup}

\def\allowhyphens{\nobreak\hskip\z@skip}

\def\vperson[#1]{\leavevmode\ndx{#1}{2}\allowhyphens{\sc#1}}
\def\hperson[#1]{\allowhyphens{\sc#1}}
\def\vcite(#1){\ndx{#1}0\footnote{\ref{#1}.}}
\def\hcite(#1){\ndx{#1}0\ref{#1}}
\def\vdefinition|#1|{\leavevmode\Presignal\ndx{#1}{1}%
 \Green\allowhyphens#1\allowhyphens\Black\Signal}
\def\hdefinition|#1|{\allowhyphens\ndx{#1}{1}}%\tomargin{\it#1}}
\def\vlabel<#1>{{\sl#1}\itcor} % used for book titles only
%\def\hlabel<#1>{\label{#1}}
\def\vref>#1>{la~\S\refsc{#1}, pá\-gi\-na~\refpg{#1}}
\def\href>#1>{la~\S\refsc{#1}}
\def\vindex#1{\leavevmode\presignal\ndx{#1}{3}%
 \Blue\allowhyphens#1\allowhyphens\Black\signal}
\def\hindex#1{\allowhyphens\ndx{#1}{3}\hsignal}

%\def\signal{\hbox to0pt{\Blue\hss\vbox to0pt{
% \hbox{$\longleftarrow$}\vss}\Black}}
\def\Signal{\hbox to0pt{\hss\vrule width10pt height-2pt depth2.4pt
 \vrule width0.4pt height4pt depth2pt}}
\def\Presignal{\hbox to0pt{\vrule width0.4pt height4pt depth2pt
 \vrule width10pt height-2pt depth2.4pt\hss}}
\def\signal{\hbox to0pt{\hss\vrule width10pt height-2pt depth2.4pt}}
\def\hsignal{\hbox to0pt{\hss\vrule width10pt height-2pt depth2.4pt}}
\def\presignal{\hbox to0pt{\vrule width10pt height-2pt depth2.4pt\hss}}


%% OPTION comment out to get signals
\let\Signal=\null \let\Presignal=\null
\let\signal=\null \let\presignal=\null
\let\hsignal=\null

\def\EPA#1{{\sc epa}\ndx{Casares1999}0~\S{\tenrm#1}} %%%%%%%%%% Provisional

% TEXT

\catcode`\@=12

\input sll0.tex

\setdest \let\par=\destpar

\input sll1.tex
\input sll2.tex
\input sll3.tex
\input sll4.tex
\input sll5.tex
\input sll6.tex

\let\par=\endgraf

\catcode`\@=11
\catcode`\^=7 \catcode`\_=8
\everypar{}\parindent=20pt

\vfill\break %%%%%%%%%%%%%%%%%%%%%%%%%%%%%%%%%%%

\setdest
\vbox to22pt{\centerline{\fontone Referencias}\toc3{Referencias}\vss}
\vskip 1pc plus1pc minus6pt

Se señalan las ediciones de las obras consultadas. Los números que
aparecen al final de cada reseña, detrás del triángulo~{\teni.}, indican
las páginas en las que la obra está citada, así {\teni.}\refpg{La
jerarquía de Chomsky} significa que la cita se encuentra en la
página~\refpg{La jerarquía de Chomsky}.

\newwrite\bibnotes \immediate\openout\bibnotes=auxiliar.bdb
\iverb\bibnotes"% AUXILIAR.BDB"
\iverb\bibnotes"\record{Lettvin1959}{INCOLLECTION}{"
\iverb\bibnotes" \AUTHOR{Lettvin, J.\kern-1pt Y\kern-1pt.; "
\iverb\bibnotes" Maturana, H.R.; McCulloch, \kern-1ptW\kern-2pt.S.; "
\iverb\bibnotes" Pitts, W\kern-2pt.H.}}"
\iverb\bibnotes"\record{Descartes1641}{BOOK}{"
\iverb\bibnotes" \YEAR{\kern-1.5pt 1\kern-0.3pt 6\kern-0.2pt 3\kern-0.2pt 7\kern-1pt , "
\iverb\bibnotes"  \kern-1pt 1641\kern-1pt}}"
\iverb\bibnotes"\record{Freud1900}{BOOK}{\YEAREDITION{\kern-1pt 1966}"
\iverb\bibnotes" \EDITION{Traducción de Luis López-Ballesteros y de Torres (\kern-1pt1923)}}"
\iverb\bibnotes"\record{Rumelhart1986}{BOOK}{"
\iverb\bibnotes" \AUTHOR{Rumelhart, D.E.; McClelland, J.L.; \& the PDP Research Group}}"
\iverb\bibnotes"\endinput"
\immediate\closeout\bibnotes

\newcount\lastpgno \newcount\thispgno
\newif\ifrange

\def\gobblethree#1#2#3{\ignorespaces}

\def\ndxline#1#2{\ifnum#2=0 \let\oldk@y=\newk@y \def\newk@y{#1}%
 \let\next=\bibline \else \let\next\gobblethree \fi \next}

%\def\bibbreak{\ifdim\lastskip<\smallskipamount \removelastskip
% \penalty-200 \vskip3pt plus 2pt \relax\fi} %plus 1.5pt minus 1pt \fi}
%\let\bibbreak=\filbreak
\def\bibbreak{\par\vskip-2pt plus 1fil \penalty-200 \vskip2pt plus -1fil\relax }

\def\bibline#1#2#3{\ifx\newk@y\oldk@y \thispgno=#3
 \ifnum\lastpgno=\thispgno \else \advance\lastpgno1
  \ifnum\lastpgno=\thispgno \rangetrue \else
   \ifrange \advance\lastpgno-1 --\number\lastpgno\fi
   \lastpgno=#3\rangefalse
   , \pdfgoto{#3}{#1}%
  \fi\fi
 \else % \newk@y<>\oldk@y
 \ifrange --\number\lastpgno\rangefalse \fi
 \bibbreak
 \setdest \cleantoks \lastpgno=0
 \def\record##1##2##3{\def\secondk@y{##1}\ifx\newk@y\secondk@y##3\fi}
 \input sll.bdb
 \input auxiliar.bdb
 \hang\noindent\rm{\sc\the\AUTHOR}:%
 {\def\em{\string\em\space}\def~{\string~}\def\it{\string\it\space}%
  \def\sc{\string\sc\space}%
  \lbl{\newk@y}{\string\rm{\string\sc\space
  \expandafter\expandafter\the\AUTHOR}\space
  (\expandafter\expandafter\the\YEAR):\space
  {\string\sl\space\expandafter\expandafter\the\TITLE}}}%
 {\sl\null{ }\the\TITLE
 \setbox0=\hbox{\the\SUBTITLE}\ifdim\wd0=0pt\else\the\SUBTITLE\fi}%
 \setbox0=\hbox{\the\YEAREDITION}\ifdim\wd0=0pt
  \YEAREDITION={\the\YEAR}\else\ (\the\YEAR)\fi
 \setbox0=\hbox{\the\EDITOR}\ifdim\wd0=0pt\else.
   Recopilado por \the\EDITOR\fi
 \setbox0=\hbox{\the\BOOKTITLE}\ifdim\wd0=0pt\else
   \ en ``\the\BOOKTITLE''\fi
 \setbox0=\hbox{\the\EDITION}\ifdim\wd0=0pt\else. \the\EDITION\fi
 \setbox0=\hbox{\the\PUBLISHER}\ifdim\wd0=0pt\else, \the\PUBLISHER\fi
 \setbox0=\hbox{\the\ADDRESS}\ifdim\wd0=0pt\else, \the\ADDRESS\fi
 \setbox0=\hbox{\the\JOURNAL}\ifdim\wd0=0pt\else, \the\JOURNAL\fi
 \setbox0=\hbox{\the\VOLUME}\ifdim\wd0=0pt\else, vol.~\the\VOLUME\fi
 \setbox0=\hbox{\the\MONTH}\ifdim\wd0=0pt
  \setbox0=\hbox{\the\YEAREDITION}\ifdim\wd0=0pt\else, \the\YEAREDITION\fi
  \else, \the\MONTH\ de \the\YEAREDITION\fi
 \setbox0=\hbox{\the\PAGES}\ifdim\wd0=0pt\else, pp.~\the\PAGES\fi
 \setbox0=\hbox{\the\ISBN}\ifdim\wd0=0pt\else, {\sc isbn} {\the\ISBN}\fi.
 \setbox0=\hbox{\the\NOTE}\ifdim\wd0=0pt\else \the\NOTE. \fi
 {\teni.}\pdfgoto{#3}{#1}\fi
 \ignorespaces}

\bigskip

\input auxiliar.abc


\vfill\break %%%%%%%%%%%%%%%%%%%%%%%%%%%%%%%%%%%

% Output for INDEX adapted from [417]

\newdimen\fullhsize \fullhsize=\hsize
\newdimen\fullvsize \fullvsize=\vsize
\def\fullline{\hbox to\fullhsize}

\newdimen\gutter \gutter=1pc
\newbox\partialpage
\def\begindoublecolumns{\begingroup
 \output={\global\setbox\partialpage=\vbox{\unvbox255\bigskip}}\eject
 \output={\doublecolumnout}%
 \hsize=\fullhsize \advance\hsize by-\gutter \divide\hsize by2
 \vsize=\fullvsize \multiply\vsize by2 \advance\vsize by2pc}
\def\enddoublecolumns{\output={\balancecolumns}\eject
 \endgroup \pagegoal=\vsize}

\def\doublecolumnout{\splittopskip=\topskip \splitmaxdepth=\maxdepth
 \dimen0=\fullvsize \advance\dimen0 by-\ht\partialpage
 \setbox0=\vsplit255 to\dimen0 \setbox2=\vsplit255 to\dimen0
 \shipout\vbox{\vbox to 0pt{\vskip-22.5pt
  \fullline{\vbox to8.5pt{}\the\headline}\vss}\nointerlineskip
  \vbox to\fullvsize{\boxmaxdepth=\maxdepth \pagesofar}
  \baselineskip=24pt \fullline{\the\footline}}\advancepageno
 \unvbox255 \penalty\outputpenalty}
\def\balancecolumns{\setbox0=\vbox{\unvbox255}\dimen0=\ht0
 \advance\dimen0 by\topskip \advance\dimen0 by-\baselineskip
 \divide\dimen0 by2 \splittopskip=\topskip
 {\vbadness=10000 \loop \global\setbox3=\copy0
  \global\setbox1=\vsplit3 to\dimen0
  \ifdim\ht3>\dimen0 \global\advance\dimen0 by1pt \repeat}
 \setbox0=\vbox to\dimen0{\unvbox1}%
 \setbox2=\vbox to\dimen0{\dimen2=\dp3 \unvbox3 \kern-\dimen2 \vfil}%
 \pagesofar}
\def\pagesofar{\unvbox\partialpage
 \wd0=\hsize \wd2=\hsize \fullline{\box0\hfil\box2}}

%%%

\setdest
\vbox to22pt{\centerline{\fontone Índice temático}\toc3{Índice temático}\vss}
\vskip 1pc plus1pc minus6pt
%\vskip 5.87955pt % just to make exact room for columns

Los conceptos escritos en letra negrilla son los más importantes de la
teoría de la subjetividad, y los números escritos también en negrilla
indican las páginas en las que están definidos. Los otros conceptos son
menos importantes, y los otros números señalan las páginas en las que
aparecen, independentemente de su importancia, los distintos conceptos y
autores.

\lastpgno=0 \thispgno=0 \rangefalse

\def\ndxline#1{\let\oldkey=\newkey \def\newkey{#1}\defline}
\def\defline#1#2#3#4{\ifnum#1=0 \else % not biblio
 \ifx\newkey\oldkey \thispgno=#4
  \ifnum\lastpgno=\thispgno \else \advance\lastpgno1
   \ifnum\lastpgno=\thispgno \rangetrue \else
    \ifrange \advance\lastpgno-1 --\number\lastpgno\fi
    \lastpgno=#4\rangefalse
    , \pdfgoto{\ifnum#1=1 {\bf#4}\else #4\fi}{#2}%
   \fi\fi
 \else % \newkey<>\oldkey
  \ifrange --\number\lastpgno\fi
  \bibbreak\hang\noindent \lastpgno=#4\rangefalse
  \ifcase#1 {\tt\newkey}\or{\bf\newkey}\or{\sc\newkey}\or{\rm\newkey}\else
   {\it\Red \newkey\Black}\fi
 ,\space\space
 \pdfgoto{\ifnum#1=1 {\bf#4}\else #4\fi}{#2}\fi\fi\ignorespaces}

\begindoublecolumns
\parindent=12pt \rightskip=0pt plus 4pc \hyphenpenalty=250
\input auxiliar.abc
 \ifrange --\number\lastpgno \fi
\enddoublecolumns

\vfill\break %%%%%%%%%%%%%%%%%%%%%%%%%%%%%%%%%%%

\setdest
\vbox to22pt{\centerline{\fontone Índice}\toc3{Índice}\vss}
\vskip 2pc plus1pc minus6pt

\def\tocline#1{\ifcase #1\let\next=\toclinezero \or
 \let\next=\toclineone \else \let\next=\toclineindex \fi \next}

\def\toclinezero#1#2#3#4{\bigbreak
 {\pdfcode \pdf@ut \stringactives \pdfoutline goto num #2
  count -\number\@utno {#1}\pdfendcode}%
 \line{\bf\pdfgoto{#1}{#2}\hfil}\ignorespaces}
\def\toclineone#1#2#3#4{\par
 {\pdfcode \pdf@ut \stringactives \pdfoutline goto num #2
  count -\number\@utno {\noexpand§#3 #1}\pdfendcode}%
 \contentsline{#1}{#2}{$\S#3$\quad}{#4}\ignorespaces}

\def\Referencias{Referencias}
\def\toclineindex#1#2#3#4{\par\def\1{#1}\ifx\1\Referencias\bigbreak\fi
 {\pdfcode \pdf@ut \stringactives \pdfoutline goto num #2
  count -\number\@utno {#1}\pdfendcode}%
 \contentsline{{\bf #1}}{#2}{}{#4}\ignorespaces}

\newdimen\tridig \setbox0=\hbox{$\S123$\quad}\tridig=\wd0

\def\contentsline#1#2#3#4{\setbox0\hbox to\tridig{\hfil#3}\setbox2\hbox{#1}%
 \dimen0=\hsize \advance\dimen0 by -\wd0
 \multiply\dimen0 by 8 \divide\dimen0 by 10
 \ifdim\dimen0>\wd2 \line{\box0 #1\tocleaders \pdfgoto{#4}{#2}}\else
  \line{\box0 \vtop{\hsize=\dimen0 \raggedright \normalbaselines
   \let\\=\ \noindent #1\strut}\tocleaders \pdfgoto{#4}{#2}}\fi}
\def\tocleaders{\leaders\hbox to\baselineskip{\hss\bf.\hss}\hfil}

\raggedbottom \openup3pt
\input auxiliar.toc

 \vfill % Colofón
 {\parindent=0pt \obeylines \normalbottom %\baselineskip=12pt
 Este libro ha sido compuesto por el autor
 utilizando el programa {\TeX}
 del Profesor Knuth (Stanford University).
 Los tipos son de la familia Computer Modern.
 Las figuras están descritas en {\logo METAFONT}.
 \par}

\let\vfill=\relax
\bye
