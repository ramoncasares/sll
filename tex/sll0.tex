% sll0.tex (RMCG20001212)

% front matter

% pdf info

\pdfcode
 \begingroup\corkfont
  \pdfinfo{
   /Author (© Ramón Casares 2002, 2012)
   /Title (Sobre la libertad)
   /Subject (Teoría de la subjetividad)
   /Creator (Licencia http://creativecommons.org/licenses/by-sa/3.0/)
   /Keywords (objeto problema significado símbolo sintaxis sujeto)}
 \endgroup
\pdfendcode

% [0] Portada

\pdflabel \toc0{Sobre la libertad}

\shipout\vbox{\background
\pdfWhite
\hbox{}
\vskip2in
\centerline{\ptitlefont Sobre}
\vskip1pc
\centerline{\ptitlefont la libertad}
\vskip10pc
\centerline{\psubtitlefont La teoría de la}
\vskip6pt
\centerline{\psubtitlefont subjetividad}
\vskip2pc
\centerline{\pauthorfont Ramón Casares}
\pdfBlack
}

% [1] Autor y título

\centerline{\fontone Ramón Casares}
\vskip1pc
\hrule height 1pt
\vskip1.5pc
\centerline{\fontzero Sobre la libertad}


\vfill\break % [2] Colección (en blanco)
\advance\pageno1

% [3] Autor, título, subtítulo

\centerline{\fontone Ramón Casares}
\vskip1pc
\hrule height 1pt
\vskip1.5pc
\centerline{\fontzero Sobre la libertad}
\vskip1pc
\centerline{\fontone Teoría de la subjetividad}

\vfill\break % [4] Créditos

\null

\def\URI#1#2{\leavevmode\pdfcode \pdfstartlink
  attr {/Border [0 0 0]}
  user {/Subtype /Link /A << /Type /Action /S /URI /URI (#2) >>}%
 \pdfendcode{\tt#1}\pdfcode\pdfendlink\pdfendcode}

\vfill
 {\sl Sobre la libertad}\par
 {\sl La teoría de la subjetividad}\par
 1ª edición (\todayiso)\par
 \null\par
 Publicado por \URI{www.ramoncasares.com}{http://www.ramoncasares.com}\par
 \copyright\ Ramón Casares, 2002, 2012\par
 Este libro queda liberado conforme a los términos de la licencia\par
 \ccbysa\ {\sf Creative Commons Attribution-ShareAlike 3.0},\par
 \URI{http://creativecommons.org/licenses/by-sa/3.0/}%
     {http://creativecommons.org/licenses/by-sa/3.0/}\par
 \null\par
 ISBN-13: 978-1-4536-xxxx-6\par
 ISBN-10: 1-4536-xxxx-6\par

\break % [5] Dedicatoria

\null
\vfil
\rightline{\em A mis padres}
\vfil
%\rightline{[Agradecimientos]}
% A Valentín Fernández Vidal
% A Nancy Konvalinka
\vfil
\break

\null\vfill\break % [6] Blanco

\endinput
