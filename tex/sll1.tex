% sll1.tex (RMCG19991123)

\Part Prolegómenos

\Section Propósito

El objeto de este ensayo es el ^{libre arbitrio}. Trata, pues, sobre la
^{libertad}, pero no sobre cualquiera, sino sobre el ^{libre albedrío}
según la ^{teoría de la subjetividad} presentada en el libro ^<El
problema aparente>, al que nos referiremos como {\sc epa}^(Casares1999).
Aun así, y dado que el propósito de este ensayo es facilitar la lectura
de aquel libro, no es preciso haberlo leído para entender este texto.

^<El problema aparente> plantea matemáticamente la cuestión
epistemológica_{epistemología} tal como se entiende desde ^[Descartes]:
qué puede llegar a conocer un sujeto que recibe datos crudos pero
valorados, unos le favorecen y otros no, y no tiene \latin{a priori}
ningún conocimiento adicional sobre su ^{entorno} exterior. Y una vez
planteado el ^{problema} matemáticamente, las propias ^{matemáticas} lo
resuelven.

Esa presentación de los argumentos pretende evitar las dificultades
filosóficas. De haberlas, éstas han de limitarse a cuestionar si el
problema matemático modela suficientemente la cuestión epistemológica, o
si la resolución encontrada, ya que no es la única posible, se ciñe a
los hechos. De manera que, dejando aparte los obstáculos técnicos, el
camino es filosóficamente llano.

Pero no lo es para todos. Hay quien encuentra inesperadas algunas de las
conclusiones o, aun peor, las considera absurdas. Se abren varias
posibilidades para quien opina así. Puede que, amparado en la regla de
reducción al ^{absurdo}, determine que, dado que las conclusiones de un
razonamiento bien desarrollado resultan absurdas, los postulados de
partida son falsos. Otra posibilidad es que califique la situación de
paradójica, al tener por ciertas las premisas, por correcta la
inferencia y por falsa su conclusión. Por último, puede terminar por
aceptar que, aunque inesperadas o sorprendentes, las conclusiones son
también verdaderas.

Para facilitar esta tercera posibilidad, razonaremos aquí en los dos
sentidos: en el de ^{entrada}, o sea, de fuera a dentro, y en el de
^{salida}, esto es, de dentro a fuera. Se trata, en definitiva, de
presentar las más diversas consecuencias de la teoría de la subjetividad
para mostrar todo su poder explicativo. De este modo, quien no esté
dispuesto a aceptarla, tendrá que sustituirla por otra teoría que tenga,
por lo menos, tanto alcance como ésta.


\Section Unos consejos

Este ensayo presenta muchos conceptos nuevos y, lo que es peor, muchos
de los conceptos cotidianos y fundamentales son interpretados de un modo
peculiar, o incluso extravagante_{extravagancia}. Además, y
exclusivamente por mi culpa, no repito ociosamente las explicaciones; si
repito es para introducir asuntos nuevos, por lo que la densidad
conceptual es grande. En mi descargo, no hacer repeticiones abrevia el
ensayo, que así no es largo, mientras que las repeticiones que yo no
hago puede hacerlas usted, querido lector, si quiere, y de este modo
tampoco las sufre, si no las precisa.

Pero, probablemente, la mayor complicación provenga de la variedad y
diversidad de las disciplinas que se ven involucradas:
 ^{ética}, ^{filosofía}, ^{epistemología},
 ^{lingüística}, ^{lógica}, ^{matemáticas},
 ^{computación}, ^{cognición}, ^{psicología},
 ^{neurología}, ^{biología} y ^{física}.
Nadie puede tener un conocimiento completo de todas ellas, de manera que
cada uno de ustedes tendrá una ^{perspectiva} diferente de la teoría,
que dependerá de cual sea su formación y su temperamento.

Ver en perspectiva, que es siempre ineludible, es más distorsionante en
este caso por dos razones. La primera, ya apuntada, es que siendo grande
el número de dimensiones de la ^{escultura}, el número de vistas
distintas es enorme, por lo que resulta más difícil integrarlas todas en
un único objeto coherente. La segunda, y más grave, es que el propio
escultor también tiene un saber limitado, razón por la cual su escultura
tendrá un punto de vista preferente, y por la cual, desde otras
posiciones, lo realizado podría, incluso, no coincidir con sus
intenciones. Así que pido disculpas por los errores que, si este
argumento es correcto, encontrarán en mayor grado aquellos de ustedes
con conocimientos más dispares a los míos. Y, consecuentemente, para no
viciar la investigación empírica de esta cuestión, no revelaré mis
aficiones.

\breakif2

En estas circunstancias no hay una fórmula mágica_{magia} que asegure la
comprensión de este ensayo que, no obstante, sería imposible si se
leyera con prejuicios. Mi consejo es, pues, amable lector, que tenga
^{paciencia}, que abra su entendimiento a lo nuevo, y, por favor,
suspenda su opinión hasta que comprenda completamente la teoría, porque
no importan tanto los detalles como el conjunto. Y déjese llevar por las
lucubraciones, aunque le parezcan del todo improbables, porque descubrir
consecuencias insospechadas es interesante y, ¡ojalá!, divertido.
Comencemos ya.


\Section Una diferencia pequeña

¿Qué es más fácil, distinguir a los gatos de los perros o calcular
raíces cuadradas?

Para una persona es más fácil distinguir a un ^{gato} de un ^{perro} que
hallar la ^{raíz cuadrada} de un número. Lo primero ni siquiera se
enseña en la escuela. Para un gato o para un perro, o incluso para un
^{ratón}, es también más sencillo distinguir a los gatos de los perros
que hacer raíces cuadradas, tarea que les resulta imposible.

A pesar de esta unanimidad, resulta que, desde el punto de vista de un
ingeniero_{ingeniería} encargado de diseñar una máquina que ejecute
tales tareas, calcular raíces cuadradas es mucho más sencillo que
reconocer gatos. Es más fácil construir una máquina calculadora que
extraiga raíces cuadradas que una máquina que distinga gatos y perros.
Dicho de otra manera: se precisan muchos más recursos de ^{computación}
para identificar gatos que para calcular raíces cuadradas.

Este error de apreciación es uno de los primeros y más interesantes
descubrimientos de la ^{inteligencia artificial}, que es el nombre que
se le da a una de las ramas de las nuevas ciencias de la ^{cognición}.
La consecuencia es inmediata: como resulta que un perro es tan capaz de
distinguir gatos como una persona, se deduce que la diferencia entre las
capacidades de computación de las personas y de los perros es en lo
poco, y no en lo mucho. Hay una diferencia pequeña con efectos
aparentemente muy grandes. Esta conclusión está de acuerdo con la teoría
de la ^{evolución} darviniana, por lo que tampoco debería sorprender.

En todo lo que sigue intentaremos descubrir en qué consiste esa pequeña
diferencia y, quizás, al insistir en la diferencia, parezca grande. No
lo es.


\Section ¡Abajo el objetivismo!

El regreso de ^[Descartes] a los primeros principios del saber es
inobjetable. Es difícil rebatir que el único conocimiento inmediato y
directo, lo único indudable, es el propio ^{yo}. Lo demás se conoce
indirectamente.

Pero abrimos los ojos y vemos nítidamente una ^{piedra}. Tenemos que
reflexionar para recordar las prescripciones descartianas, y aún así es
difícil dudar de la existencia de la piedra. Más reflexión puede servir
para que nos percatemos de que, en verdad, lo que captamos son unos
colores y unas luces que identificamos como una piedra, y no la piedra
misma. Entonces tomamos la piedra, la palpamos, y la ^{duda cartesiana}
se disipa por segunda vez. Un poco más de reflexión nos enseña que la
situación no ha variado en lo fundamental, ya que agarrar la piedra sólo
nos proporciona datos adicionales sobre la forma, el tamaño, el peso, la
rugosidad y la temperatura de lo tomado. La piedra sigue siendo el
resultado de una deducción realizada sobre unos datos.

Lo que confunde del proceso anterior es que la deducción, aun siendo la
parte computacionalmente más costosa, es inconsciente y automática. Nos
es más fácil deducir, que descubrir que estamos deduciendo, y de este
modo la deducción pasa inadvertida si no ponemos toda nuestra atención.
Y aún entonces, dado lo extraño de la situación, nos parece exagerado
dudar de la existencia de la piedra.

Nuestro cerebro realiza este imperceptible proceso de determinación de
objetos a causa de su aptitud para la supervivencia. Forma parte de
nuestra herencia genética y no tiene que ser aprendido. Es importante
observar que el proceso de objetivación es previo al proceso de
^{simbolización} que hace posible el ^{habla} y la ^{consciencia}
simbólica; y también el cálculo de raíces cuadradas. Es importante
porque explica que, para el substrato simbólico de nuestro pensamiento,
el objeto, en nuestro caso la piedra, es un dato y no el resultado de
una deducción. Y por lo tanto, aunque para el cerebro entero los datos
son los colores captados y las formas sentidas, para la consciencia
simbólica los datos son los objetos.
$$\hbox{Fenómeno} \underbrace{\longrightarrow \hbox{Objeto}
 \longrightarrow
 \overbrace{\hbox{\strut Palabra}}^{\hbox to 0pt{\hss
  Consciencia\hss}}}_{\hbox{Cerebro}}$$

\breakif2

Para ser coherentes con la conclusión anterior, hemos de abandonar
completamente la ^{ontología}. La existencia de los objetos es una
construcción del cerebro. Los objetos y todas y cada una de sus
propiedades dependen del sujeto que los percibe. El ^{subjetivismo} se
impone como la única alternativa posible.

Esta prueba del subjetivismo se basa en que, de los dos procesos
cognitivos considerados, el de objetivación es previo al de
simbolización. Quien no tenga esto por evidente, piense que no se puede
hablar de lo que aún no se ha pensado, y que por lo tanto, para poder
hablar de objetos, éstos han de ser previos al habla. Un caso particular
del anterior, pero muy a propósito, son las ilusiones_{ilusión}, que
acaecen cuando los procesos simbólicos conscientes descubren errores en
otros procesos cognitivos, necesariamente previos. Las ilusiones son
inquietantes porque nos revelan que lo que vemos puede no ser como lo
vemos. Y esto es lo que asevera el subjetivismo: que fuera no hay
objetos.

La teoría de la subjetividad afirma, pues, que de lo que no soy yo, que
podemos llamar ^{universo} exterior, sólo tenemos como verdadero un
torrente de datos crudos. Los datos que se le presentan como inmediatos
a nuestra consciencia simbólica son, ya, unos datos elaborados. La
preparación de estos datos sigue unas recetas que, por un lado, han
favorecido la supervivencia de aquéllos que nos han precedido, y por
otro, hacen objetos de las sensaciones. Y esto es todo.

Sé que, a pesar de su aparente lógica, todo esto es difícil de aceptar.
Exige entender que las cosas no son como se nos aparecen
conscientemente, no son obvias. Pero, aunque parezca que derribar la
teoría objetivista nos deja sin suelo que pisar, conviene reparar en dos
argumentos, uno teórico y otro práctico. A efectos prácticos podemos
seguir razonando como objetivistas, con la seguridad de que la
objetivación ha superado la prueba de funcionar millones de años sin
fallos catastróficos. A efectos teóricos, y si todo esto es correcto, el
suelo que nos proporciona la teoría objetivista es ilusorio, por lo que
más nos vale ser consecuentes con nuestros principios y adoptar sin
miedo el subjetivismo, si pretendemos entender cabalmente lo que es el
^{yo}, la ^{consciencia} y el ^{universo} ^{mundo}.

\vfill\break

\Section La realidad objetiva es subjetiva

¿Qué ves? Un niño columpiándose. No, diría un pintor
impresionista_{impresionismo}, se ven colores, manchas de colores. Es
como si tuviéramos unas gafas que añaden etiquetas_{etiqueta} a lo que
vemos. % [Portada: Un niño columpiándose]

Lo que está fuera es cambiante_{cambio}. Si nos mostraran por televisión
la imagen captada en nuestra retina, nos marearíamos. Porque el
^{mareo}, en condiciones no patológicas, siempre ocurre cuando
interviene un movimiento incontrolado; por ejemplo, el debido al oleaje
cuando estamos embarcados, o el del coche cuando no conducimos, y hasta
la imagen televisiva si otro maneja con excesiva ligereza el mando a
distancia. Esto significa que nos mareamos cuando nuestro sistema
estabilizador no tiene datos con los que adelantarse a las percepciones,
o sea, cuando no es posible estabilizar lo que se mira.

Vamos a distinguir la sensación de la percepción. Llamaremos
^{sensación} a la impresión sensorial, como, por ejemplo, la imagen
captada en la retina del ojo, y utilizaremos el verbo sentir únicamente
con este significado. La ^{percepción} es el proceso que toma la
sensación y produce las cosas ya estabilizadas y etiquetadas que
denominamos objetos_{objeto}. La sensación es cambiante, lo percibido
no. Utilizaremos frecuentemente `^{ver}' como sinónimo de `percibir',
aunque hay otras modalidades perceptivas como ^{oir} o ^{gustar}.

Llamamos realidad a las cosas que vemos, y no a las cambiantes
sensaciones que nos marean. De manera que la ^{realidad} es lo que
percibimos, no lo que sentimos. No es real la impresión en la retina,
que desconocemos, como lo prueba que ignoremos el punto ciego, véase
^[Resnikoff]^(Resnikoff1989). Es real la piedra que vemos, el objeto
visto, y, en consecuencia, la realidad de los objetos está elaborada.

No podemos evitar ver objetos, aunque éstos no existan fuera de nuestra
cabeza. No vemos el universo como es, o mejor, no vemos el universo como
lo sentimos, lo vemos como lo vemos. Ocurre que la realidad de los
objetos, o sea, la realidad objetiva, es una construcción del sujeto, o
sea, es subjetiva. Y, en resumen, la realidad objetiva es subjetiva.
$$\overbrace{\hbox{\strut Fenómeno}}^{\hbox{Universo}}
 \underbrace{\longrightarrow
  \overbrace{\hbox{\strut Objeto}}^{\hbox to0pt{\hss Realidad\hss}}
  \longrightarrow
  \overbrace{\hbox{\strut Palabra}}^{\hbox to 0pt{\hss
   Consciencia\hss}}}_{\hbox{Sujeto}}$$

\breakif2

\Section El sueño

Los sueños descubren las etiquetas. Tomado de un ^{sueño}: `Eras tú,
^[Piripili], aunque con el aspecto y la voz de tu madre'. Al etiquetar
mal, el sueño descubre las etiquetas, y que lo que importa son las
etiquetas y no la apariencia. Tampoco es exactamente la etiqueta lo que
importa. Lo que importa no es la ^{etiqueta}, ^[Piripili], sino su
significado, \meaning{tú}.

^[Eco]^(Eco1997) termina señalando el carácter surrealista_{surrealismo}
y onírico de los pasatiempos jeroglíficos, porque en ellos también se
confunde la ^{sensación} y la ^{palabra}. Nótese que en el sueño de
^[Piripili] y su madre, no hay manera de visualizar la escena, necesita
ser explicada con palabras; como si fuera un ^{jeroglífico}, hay que
etiquetar como ^[Piripili] lo que se muestra, a todos los efectos, como
su madre.

En los sueños no tengo ^{voluntad}, lo que contrasta con lo que ocurre
cuando estoy despierto y consciente. Me parece que la evolución, sin
presión adaptativa alguna en este punto, no se ha preocupado de
distinguir adecuadamente el papel de lo simbólico, me refiero a las
etiquetas, en el sueño, como sí ha tenido que hacerlo durante la
vigilia. Este aspecto inquietante de los sueños fue el que permitió a
^[Freud]^(Freud1900) descubrir el error por el cual el ^{sujeto} se
identifica con su ^{yo} consciente.

El ^{sujeto} no se ve como sujeto, sino como ^{yo}, lo que quiere decir
que el sujeto se identifica con su parte consciente y con su voluntad.
Este error de perspectiva explica por qué el sujeto entiende que el
objeto es algo externo, y no interno. Y es un error contumaz porque es
interesado: si el sujeto fuera idéntico a su voluntad, entonces no
tendría necesidad de morir_{muerte}.


\Section La realidad es involuntaria

Pero, así como la ^{voluntad} es consciente, o no es voluntad, en
cambio, el proceso de objetivación no es consciente, sino previo a la
^{consciencia} y automático. Como hemos visto en ^>¡Abajo el
objetivismo!>, el objeto es el resultado de un proceso cuyo diseño es
evolutivo, o, dicho de otro modo, el ^{programa} de la objetivación se
encuentra codificado en los genes. Por esto el objeto, aunque es
subjetivo, no está a merced de la voluntad del sujeto. De manera que,
sí, la realidad objetiva es subjetiva, pero involuntaria. La realidad es
involuntaria.

\breakif2

Hasta la ^{física}, cuando alcanza a describir objetos, es
necesariamente parte de la ^{psicología}. Es el caso de la mecánica
cuántica, que llega al límite del objeto. No puede ser de otro modo si
los objetos son productos de la cognición, o sea, subjetivos. Y, la
objetividad de la ciencia física, que parece elevarla por encima del
incierto y poco ecuánime ^{mundo} subjetivo, no se debe a la existencia
autónoma de los objetos, sino a que éstos están más allá del alcance
volitivo del sujeto.

\endinput
