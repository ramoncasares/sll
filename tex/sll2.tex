% sll2.tex (RMCG20000412)

\Part Entrada

\Section El adaptador

Hasta aquí hemos advertido que los objetos intervienen en los procesos
que determinan como vemos, es decir, en la ^{percepción}, y hemos
aceptado que tales procesos fueron diseñados por la ^{evolución}. Ahora
investigaremos por qué fue así, y para ello lo primero es dilucidar para
qué sirven los objetos.

Los ^|objeto|s simplifican la sensación. La cantidad de datos que captan
nuestros sentidos es enorme, demasiado grande para tratarla
completamente. Afortunadamente, el propósito de captar todos estos datos
es determinar qué es lo que conviene hacer en cada circunstancia, y esto
depende, generalmente, de unos pocos datos. La estrategia consiste,
pues, en determinar si de todos los datos recibidos se deduce que estas
pocas cosas están, o no, presentes.

Ahora, para centrar las explicaciones que siguen, vamos a establecer
algunas definiciones básicas. La ^{sensación} es la impresión del
fenómeno exterior sobre el ^{cuerpo}, sobre los sentidos, sentir es
recibir o captar la sensación, y percibir, o ver, es reconocer los
objetos presentes a partir de los signos_{signo}, o indicios, detectados
en la sensación. Así que denominamos ^|percepción| al proceso que
convierte al fenómeno en objeto ^{presente}, esto es, lo sentido en
percibido.

La percepción simplificadora ha sido explotada por la evolución desde
antiguo. De modo que los objetos median entre el fenómeno y la acción de
algunos seres vivos que llamaremos adaptadores_|adaptador|. Estos
adaptadores son animales que disponen de sistema nervioso. El ^{sistema
nervioso} tiene la tarea de determinar, en cada momento, de entre las
posibles acciones de su cuerpo, la acción actualmente ejecutada. Esta
determinación toma como datos los objetos presentes.
$$\hbox{Fenómeno}
  \underbrace{
   \longrightarrow
   \hbox{\strut Objeto}
   \longrightarrow}_{\hbox{\strut Adaptador}}
  \hbox{Acción}
$$

Es el caso de la ^{rana}, véase ^[McCulloch]^(Lettvin1959) \latin{et
alii}, que interpreta que cualquier punto oscuro de su campo visual que
se mueve con rapidez es una ^{mosca} que intentará comer. Esta
interpretación se ha mostrado útil y ha quedado preservada en el código
genético de las ranas.

Ese objeto que ve la rana, y que he llamado mosca, no coincide con
ninguno de los objetos de las personas. Es más que una mosca y menos que
un insecto. Tampoco se corresponde con ^{bicho}, que es repugnante a las
personas pero apetecible a las ranas. En puridad, ese objeto sólo existe
dentro del sistema nervioso de la rana, y si puede hacerse algún tipo de
correspondencia, aunque sea parcial, con nuestros bichos, es porque la
diferencia entre las ranas y nosotros no es tan grande como pensamos;
véase ^>Una diferencia pequeña>. La mosca de las ranas
existe_{existencia} porque le es útil a las ranas y el bicho de las
personas existe porque le es molesto a las personas.

Según estas ideas, la rana generaliza y usa universales, sin que para
ello precise de abstracciones_{abstracción} metafísicas_{metafísica}.
Esta capacidad de ^{generalización} se apoya en el proceso de percepción
que agrupa fenómenos distintos en un mismo objeto. Lo que ocurre es que
a la rana, para su supervivencia, le es suficiente distinguir a los
gatos de las moscas.


\Section El objeto

En el adaptador se encuentran, ya, las características primarias del
^{objeto}. El objeto se hace ^|presente| cuando el sistema nervioso
tiene signos_{signo} suficientes de que lo está. Estos signos provienen
directamente de la ^{sensación}, pero también pueden provenir de otros
objetos. Esto permite sacar provecho de la ^{contigüidad} de los
objetos, ya que algunos, como el ^{humo} y el ^{fuego}, suelen aparecer
juntos, y otros nunca se muestran a la vez. La presencia del objeto
puede provocar acciones y también, lo acabamos de ver, puede dormir o
despertar la presencia de otros objetos.

La presencia_{presente} del objeto no tiene sentido alguno fuera del
adaptador. La presencia del objeto es, simplemente, el resultado de
ciertos cálculos realizados por el sistema nervioso del adaptador que,
en todo caso, están avalados por su eficacia evolutiva. La realidad
objetiva presente es completamente inferencial, y está construida, en
parte, por información codificada genéticamente y, en parte, por
cálculos efectuados por el sistema nervioso, cálculos que hemos
denominado ^{percepción}.


\Section La realidad

En cuanto se hace presente un objeto, arranca el ^{proceso} asociado a
dicho objeto. Estos procesos tienen dos tipos de efectos: pueden influir
sobre otros objetos o pueden influir sobre el resto del cuerpo. La
influencia sobre otros objetos es también de dos clases: positiva si los
hace presentes o, por el contrario, negativa si los devuelve al estado
de latencia. Con esta terminología podemos afirmar que los objetos
constituyen una red de ^{procesos concurrentes}, véase
^[Pdp]^(Rumelhart1986), que denominamos ^|realidad|.
$$\hbox{Realidad} = \hbox{Red de objetos}$$

También distinguimos ^{acción} y ^{comportamiento}. Para simplificar las
cosas, puede usted asimilar el comportamiento de una computadora al
^{programa} que está ejecutando. Así, por ejemplo, la tecla que sirve
para señalar un fin de línea cuando se utiliza un programa de edición de
textos, puede ser la misma tecla que arranca el cálculo de una expresión
matemática en un programa de aritmética. La misma acción tiene una
respuesta diferente dependiendo del programa, que asimilamos a
comportamiento. Luego la descripción del comportamiento de la
computadora es el programa que está ejecutando. Usando otras palabras,
la ^{reacción} depende tanto de la acción como del ^{estado} de la
máquina. También el estado siguiente depende tanto de la acción como del
estado actual. Por ejemplo, la tecla que sirve en el editor de textos
para pasar al estado de mayúsculas, esto es, para que a partir de ese
momento las letras se escriban en mayúscula, lo que hace es pasar al
estado mayúsculas si estaba en el estado de minúsculas, y al de
minúsculas si estaba en el de mayúsculas.

Repito los aspectos básicos del adaptador, pero ahora con más precisión
utilizando la terminología que acabamos de introducir. La tarea del
sistema nervioso consiste en discernir, en la sensación, qué
comportamiento, o programa, conviene que ejecute el ^{cuerpo} en cada
instante, y para esto le basta determinar qué objetos están presentes.

Por ejemplo, según estas definiciones, los robotes_{robot} de
^[Brooks]^(Brooks1999), que acoplan la percepción al comportamiento, son
adaptadores.
$$\hbox{Percepción} \longrightarrow \hbox{Realidad presente}
   \longrightarrow \hbox{Comportamiento}$$


\Section El sustantivo y el verbo

Que el aparato perceptivo de la ^{rana} identifique como presentes, aquí
y ahora, los objetos ^{mosca}, ^{gato} y ^{agua}, se toma como
^{condición} al resolver el ^{problema} de qué ^{comportamiento}, de los
que el ^{cuerpo} es capaz, conviene en este momento, en este caso huir.
Ignora todo cuanto no está ^{presente}, e incluso parte de lo presente;
en nuestro ejemplo ignora la mosca porque prevalece la peligrosa
presencia del gato. La acción ejecutada en esta situación para huir
puede consistir en saltar al agua.

Para simplificar la explicación, hasta aquí sólo nos habíamos fijado en
un tipo de objetos que, como mosca, corresponden a
sustantivos_{sustantivo} y que denominaremos objetos nominales_{objeto
nominal}. Pero el sistema nervioso de los adaptadores también usa otro
tipo de objetos, como ^{huir}, que corresponden a verbos_{verbo}. Cuando
el ^{objeto verbal} huir se hace presente, el sistema nervioso de la
rana ejecuta procesos que emiten una serie bien sincronizada de órdenes
ejecutivas a varios músculos del cuerpo para efectuar adecuadamente el
salto.


\Section La realidad del adaptador es objetiva

Dos conclusiones sobre la ^{realidad} del ^{adaptador}:
\beginpoints
\point Lo que media entre el ^{fenómeno} y la ^{acción} de los
adaptadores es una ^{red de objetos} denominada realidad que simplifica
la ^{sensación} captada por los sentidos.
\point El adaptador vive siempre en la realidad ^{presente}, es decir, de
objetos presentes.
\endpoints


\Section El aprendiz

La ^{red de objetos} que constituye la ^{realidad} puede ser fija o
cambiante. Llamaremos ^{adaptador simple} a aquél cuya realidad está
fijada genéticamente y que puede adaptarse a las circunstancias
presentes, pero que no puede aprender a desenvolverse en situaciones
nuevas porque no es capaz de modificar su red de objetos.

En cambio, un ^|aprendiz| es un adaptador capaz de modificar su
realidad. Es decir, la red de objetos del aprendiz es plástica y puede
ser sintonizada con su ^{entorno} exterior. Denominamos ^{modelación} a
este proceso mediante el cual el aprendiz adecúa su realidad objetiva a
las circunstancias externas.

Volviendo al adaptador simple, éste dispone de una red de objetos
rígida. Se puede decir que su modelo del exterior es rígido, pero con un
matiz. Porque, dada una red rígida, no tiene utilidad que la red
calcule, como paso intermedio, el pronóstico del modelo, para que, a
partir de él, determine el comportamiento que debe ser ejecutado. En
este caso es más eficaz y rápido tener fijado de antemano el
^{comportamiento} apropiado a cada presente, de modo que, aunque haya
modelo, no hay ^{previsión}.

Dado el ^{oportunismo} de la ^{evolución}, los primeros aprendices
debieron de ser muy parecidos a los adaptadores. Podían tener un sistema
nervioso semejante al de un adaptador simple, pero con la posibilidad de
variar, aunque no mucho, su red de objetos. Esto podía ser suficiente
para que el aprendiz aprendiera a vivir en varios entornos no muy
diferentes entre sí. En estos primeros estadios el aprendiz tampoco
necesita calcular pronósticos.

Es distinta la situación del aprendiz cuando crece la plasticidad de su
modelación. El punto crítico tiene lugar cuando la variabilidad de los
modelos es tal que no resulta práctico usar comportamientos
predeterminados para cada posible red de objetos con cada posible
configuración de objetos presentes. Es decir, que cuando el número de
realidades y de presentes que el aprendiz es capaz de producir supera
cierto umbral, comienza a ser poco eficiente tener rígidamente
codificada la respuesta a cada uno. Es entonces cuando la evolución
obtiene ventaja si diseña aprendices que, para determinar qué
comportamiento ejecutar, prevén sus consecuencias. Estos aprendices son
capaces de simular_{simulación} internamente el resultado de comportarse
de varios modos ante la realidad presente, y sólo ejecutan actualmente
el comportamiento más favorable según la simulación. Recuerdo de nuevo
que `realidad ^{presente}' es una forma abreviada de decir `red de
objetos encontrada por la modelación, en la que están ahora presentes
los objetos que determina la percepción'. Al simular partiendo de la
realidad presente, el aprendiz se adentra en el ^{futuro}.


\Section La modelación

La ^|modelación| modifica la ^{red de objetos}, es decir, que si la red
de objetos está constituida por objetos_{objeto}, que son los nudos de
la red, unidos por relaciones positivas, que despiertan, y negativas,
que duermen, más o menos fuertes, entonces la modelación podrá variar
todos estos elementos. La modelación puede crear, eliminar, reforzar y
debilitar uniones, incluso, hasta hacerlas cambiar de sentido, esto es,
de adormecedoras a despertadoras y viceversa, y puede crear, eliminar,
unir y partir objetos.

El propósito de la modelación es que la red de objetos resultante
consiga prever_{previsión} exactamente cuáles serían las
reacciones_{reacción} del ^{entorno} exterior a las acciones_{acción}
del ^{aprendiz}, porque así la simulación será certera. Esta transacción
con el exterior no supone la ^{existencia} de objetos fuera del sistema
nervioso del aprendiz, basta con que la reacción del entorno exterior
coincida con la reacción anticipada por la red de objetos interna. Esta
conclusión corrobora una tesis aquí sostenida, a saber, que la
^{realidad} objetiva es subjetiva.


\Section La simulación

La ^|simulación| puede consistir en cerrar internamente el bucle que,
sin simulación, se cierra exteriormente a través del entorno. Es decir,
el aprendiz debe ser capaz de hacer que el ^{objeto verbal} presente, en
vez de provocar la ejecución directa de acciones sobre el exterior, se
reconduzca sobre la red de objetos que modela la realidad presente, para
prever la reacción del entorno, de manera que, solamente cuando el
pronóstico sea favorable, se ejecuten actualmente las acciones.
$$\hbox{Fenómeno}
   \underbrace{\strut
    \longrightarrow
    \onitself{\strut Objeto}
    \longrightarrow}_{\hbox{\strut Aprendiz}}
  \hbox{Acción}
$$

\kern-12pt

Estos aprendices capaces de prever necesitan un ^{sistema nervioso}
mucho más potente, computacionalmente, que los adaptadores. A efectos de
esta descripción somera, diremos que esta nueva función de simulación se
realiza en una parte del sistema nervioso denominada ^{cerebro}, que
tiene la complejidad suficiente.

Pero, ¡cuidado! Que los aprendices complejos puedan simular internamente
el resultado de sus acciones, no significa que siempre lo hagan. La
^{evolución} es oportunista, va modificando sobre lo existente y no
diseña desde el principio, de manera que en el aprendiz coexistirán los
comportamientos predeterminados con la simulación. El ^{oportunismo} es
una pauta constante de la evolución darviniana que, aunque no
mencionemos explícitamente, ha de tenerse siempre en consideración.


\Section La realidad del aprendiz es cambiante

Dos conclusiones sobre la realidad del aprendiz:
\beginpoints
\point El ^{aprendiz} cambia, además del ^{presente}, la propia
^{realidad}. Los objetos de la realidad del aprendiz, además de poder
hacerse presentes o ausentes, pueden ser modificados, creados y
eliminados.
\point El aprendiz prevé_{previsión}, o sea, ve_{ver} la realidad
futura_{futuro}.
\endpoints


\Section El conocedor

La ^{red de objetos} del ^{aprendiz} es modificable. Si llamo ^|mente| a
la parte del ^{cerebro} que se encarga de modificar la red de objetos, y
siempre que escribo `red de objetos' se puede leer `^{realidad}',
entonces tenemos, de nuevo, dos posibilidades: que la mente sea rígida o
que la mente sea modificable. Podemos llamar ^{aprendiz simple} a aquél
con mente rígida y ^|conocedor| al aprendiz con mente plástica.

Un aparte sobre la ^{plasticidad}. Puede parecer que cuanta más
modificabilidad mejor y que, llevado al extremo, lo mejor es que la
plasticidad sea completa. No es lo mejor porque el resultado sería
informe, sería el caos. A un cierto nivel ha de haber una capa rígida
que dé forma a las capas más moldeables. Puede usted iluminarse
estudiando el funcionamiento de las máquinas más plásticas, que son las
computadoras. Y me refiero, en concreto, al aspecto más interesante de
estas máquinas, a saber, cómo es posible construir una ^{computadora}
con puertas lógicas: ¿cómo, con unos elementos simplicísimos que siempre
hacen exactamente lo mismo, puede construirse una computadora que hace
cualquier cosa que se quiera? La respuesta nos espera en _>La máquina
universal de Turing> y, a un nivel más profundo, justo al final de
_>El logicismo>.

Volviendo al argumento principal, veíamos que un aprendiz simple dispone
de una red plástica de procesos para prever_{previsión}, pero que los
otros procesos mentales son rígidos. Son rígidos los procesos del
aprendiz simple que determinan cómo modificar la red de objetos para que
pronostique mejor, y también la ^{percepción} y los que determinan qué
^{comportamiento} conviene dada la realidad presente. En el aprendiz
simple es moldeable la realidad, pero son fijos los procesos que la usan
y, en consecuencia, la utilizan de una manera fija.

\breakif1

El conocedor, por contra, puede utilizar la red de objetos de distintas
maneras. Una manera consiste en usarla entera, como hace el aprendiz
simple, pero las otras maneras emplean solamente partes de la realidad.
Para conseguirlo es menester que otros procesos mentales del conocedor
tengan un acceso dinámico a la red de objetos y, a otro nivel, que la
presencia de los objetos sea controlable internamente, es decir, que la
determinación de qué objetos están presentes dependa de la percepción,
como en el aprendiz, pero también de otros procesos mentales. En
definitiva, con el conocedor reaparecen los mecanismos de la
^{atención}.


\Section El sentimiento

El cumplimiento de estos requisitos exige que los procesos que
determinan qué objetos están presentes sea de dos tipos. A la
percepción, heredada de los aprendices y que tiene su origen en el
fenómeno, se añade otra vía que es como una percepción interior, ya que
permite hacer presentes los objetos que interesan al propio conocedor, e
ignorar los que no interesan. Repito, si en el aprendiz simple la
presencia de los objetos era el resultado de la interacción entre la
^{sensación} recibida del exterior y la propia red de objetos, en el
conocedor intervienen además otros procesos mentales, que denominamos
sentimientos_{sentimiento} porque son sensaciones mentales.

Los ^|sentimiento|s son los estados que determinan cómo debe utilizarse
la ^{realidad} en cada momento y, de ese modo, modifican el ^{presente}.
Un conocedor sediento_{sed} empleará la realidad de un modo diferente
que uno satisfecho. Sus problemas, y por consiguiente las soluciones, o
comportamientos que le valen, son diferentes.

Los objetos adquieren ^{significado} cuando los sentimientos se difunden
por la ^{red de objetos}. Esta definición de significado es oscura, lo
reconozco, de modo que puede usted posponer su adhesión a ella, y
tratarla como un término técnico, hasta que veamos a donde nos lleva,
así que continuemos. Los sentimientos son los términos semánticos
primitivos. El significado ha de explicar por qué los conocedores actúan
de un modo y no de otro, por qué no les vale lo mismo un comportamiento
que otro, y esto, en último término, depende de los sentimientos.

Los significados primeros son los sentimientos, básicamente el
\meaning{^{placer}} y el \meaning{^{dolor}}. Todos los demás
significados se derivan de éstos. Es la evolución la que determina lo
que es el \meaning{placer} y el \meaning{dolor}. Para la propia
evolución los significados primeros son la \meaning{^{vida}} y la
\meaning{^{muerte}}. De modo que el \meaning{placer} y el
\meaning{dolor} son sucedáneos, descubiertos por la evolución, de la
\meaning{vida} y la \meaning{muerte}. Pero sólo los sujetos simbólicos
sabemos esto; los conocedores simples_{conocedor simple}, quiero decir
los que no son sujetos, no tienen un significado para la vida y la
muerte porque les basta con sus primitivos, el placer y el dolor.

Las acciones que el ^{conocedor} asocia a la obtención de placer o de
dolor, y los objetos verbales_{objeto verbal} presentes, que son los
responsables de dicha ^{acción}, obtienen así un significado segundo.
Los procesos causantes de que los objetos presentes lo estén, así como
los objetos que arrancaron esos procesos, tienen un significado tercero,
y así sucesivamente hasta que esta ola que actualiza significados llega
a los objetos nominales_{objeto nominal} percibidos que arrancaron tales
acciones.

De esta manera se difunden los significados por toda la red de objetos,
resultando que todos los objetos del conocedor tienen, en todo momento,
significado, y por esto decimos que la red de objetos del conocedor es
una ^{red semántica}. Resumidamente: la realidad del conocedor es
semántica porque tiene significados. Y los significados, que tiñen a los
objetos de sentimiento, son los que dirigen el comportamiento de los
conocedores hacia la vida y evitan la muerte.


\Section La emoción

El proceso de asociación de significados_{significado} a signos_{signo}
es muy general. Cuando el ^{perro} de ^[Pávlov]^(Pávlov1904), en su
estudio clásico del ^{reflejo condicionado}, produce saliva al oír la
campana, está dando el significado \meaning{comida} al toque de la
campana. La \meaning{comida}, por ser vital para la supervivencia,
tendrá un significado, dado directamente por la ^{evolución}, muy
cercano a \meaning{placer}. Comer es uno de los placeres de la vida.
Pero la situación es mucho más general. A todo cuanto ve el conocedor, a
cada objeto, ha de darle un significado. Esto es así porque la vista ni
alimenta ni mata. Quiero decir que si un ^{antílope} no da a ciertas
manchas que crecen en su campo visual el significado de \meaning{león
peligroso}, entonces el ^{león} que se acerca lo mata. Y tampoco
correría el león si no fuera porque da a lo que percibe el significado
de \meaning{antílope apetecible}.

Quiere esto decir que el ^{objeto} león del antílope debe tener asociado
el significado \meaning{peligroso}. La evolución ha encontrado útil
añadir a cada objeto un significado o, lo que es lo mismo, perpetuó las
especies de conocedores_{conocedor}, que aumentan la ^{red de objetos}
con significados que tiñen los objetos de deseos_{deseo}, apetitos,
utilidades, usos y necesidades que el ^{sistema emocional}, que
abreviaremos como ^|emoción|, deriva de los sentimientos.
$$\hbox{Emoción} = \hbox{Sistema emocional}$$

Hasta aquí los procesos descritos empezaban por el ^{fenómeno}, del que
la ^{percepción} extraía signos para hacer presentes_{presente} los
objetos que, a su vez, arrancaban las secuencias de órdenes ejecutivas
que determinaban el ^{comportamiento}. Un ^{proceso} que comience en un
^{sentimiento} nos servirá para distinguir ambas entradas a la ^{red de
objetos}. Por ejemplo, un sentimiento agudo de ^{sed} encontrará, en la
red de objetos, que es necesario arrancar la búsqueda de signos de
^{agua}. Esto, sin duda, está relacionado con la ^{atención}, que, si no
hay nada más urgente, ignorará cualquier objeto cuyo ^{significado} no
tenga relación con la sed.

Los conocedores, al disponer de significados, pueden utilizar la red de
objetos para resolver los problemas concretos, como el de apagar la sed,
que el sistema emocional prioriza. Tienen una doble entrada a la
^{realidad}, una es la ^{percepción} heredada de los aprendices y los
adaptadores y la otra la ^{emoción}, propia de los conocedores, que está
orientada a fines y que, usando la atención sobre los aspectos que más
interesan en cada momento, modifica el presente.
$$\hbox{Fenómeno}
   \underbrace{
    \longrightarrow
    \column{
     \onitself{\strut Objeto}\cr
     $\uparrow$\cr
     \hidewidth\strut Sentimiento\hidewidth\cr
     $\uparrow$\cr
     \strut Libido\cr}
    \longrightarrow}_{\strut\hbox{Conocedor}}
  \hbox{Acción}
$$

\kern-12pt

La raíz del ^{sistema emocional} es la ^{libido}, que encarna en el
^{cerebro} el ^{instinto de supervivencia}. Con el instinto de
supervivencia siempre presente y con la información
propioceptiva_{propiocepción} que recibe, la emoción prioriza, en cada
instante, los sentimientos. Es decir, utiliza directamente información
sobre su cuerpo para determinar cuál es el problema más urgente. El
sentimiento seleccionado es un significado primario que, utilizando los
significados difundidos por la red de objetos, consigue que el conocedor
atienda a los objetos cuyo significado interesa.


\Section El dolor

El ^{dolor} exige toda la ^{atención}, lo que prueba que es un
^{sentimiento} primario, que el \meaning{dolor} es puro ^{significado}.
Es una notable prueba medir la propia ^{voluntad} contra el dolor. Si la
^{jaqueca} es intensa, resulta imposible dirigir los pensamientos a algo
ajeno al propio dolor de cabeza.
 Ayer lo experimenté yo mismo. % [1999.12.16]

¡Qué poco vale todo cuando el dolor es grande! Porque, si el dolor es
grande, acapara todo el significado y lo demás resulta insignificante.


\Section El signo arbitrario

En el caso de los adaptadores_{adaptador}, los significados están
asociados genéticamente_{genética} a las percepciones_{percepción}, como
ocurre en el caso de las ranas_{rana}, para las que cualquier cosa
pequeña que se mueve con rapidez es una ^{mosca} comestible, sin que
pueda separar el ^{sustantivo} (mosca) del ^{adjetivo} (comestible), el
^{objeto} del significado. Así que los adaptadores no distinguen objetos
de significados, para ellos son lo mismo. Pero otras especies más
complejas son capaces de aprender a dar ^{significado} a objetos nuevos.
Por ejemplo, un ^{perro}, que según nuestra clasificación es un
^{conocedor}, es capaz de aprender a distinguir sus galletas
industriales preferidas de otros alimentos.

La ventaja del perro sobre la rana es que, para el perro, la relación
entre el ^{signo} y su significado es arbitraria, como mostró
^[Pávlov]^(Pávlov1904). Quiero decir que cualquier objeto, en principio,
puede tener cualquier significado. La naranja, que es una fruta sabrosa,
podría ser venenosa. Si lo fuera, el mismo signo tendría un significado
distinto.

Las especies que tienen los significados genéticamente determinados
nacen sabiendo, tienen ^{ciencia infusa}, lo que es sin duda una
ventaja, pero a costa de que su ^{entorno} no varíe. Porque si, por
ejemplo, aparece en su entorno una fruta semejante a la ^{naranja} pero
venenosa, morirán por no aprender la diferencia entre la naranja y la
pseudo-naranja. Si la naranja y la pseudo-naranja fueran
indistinguibles, o sea, si la mitad de las naranjas se hicieran de
repente venenosas, entonces lo que convendría sería no comer nada cuyo
aspecto fuese el de una naranja, de manera que si el conocedor es capaz
de cambiar el significado del objeto naranja de
\meaning{apetecible} a \meaning{venenoso}, reducirá sus riesgos.

\breakif1

El ^{comportamiento} de los conocedores depende del significado más que
del signo. La incorrecta asignación de significado a un objeto puede
resultar fatal. Sería, por ejemplo, el caso de asignar el significado de
la naranja a la pseudo-naranja venenosa. Lo que importa no es tanto lo
que percibimos como el significado que le damos a lo que percibimos.


\Section El adjetivo y el adverbio

La ^{red de objetos} del ^{conocedor} recibe datos del exterior y del
interior, esto es, del ^{fenómeno} y del ^{sentimiento}. Ambos,
^{sensación} y sentimiento, determinan la ^{realidad} ^{presente}; ya lo
hemos visto. La mera ampliación de las vías de acceso no hace necesaria
la modificación del mecanismo de ^{aprendizaje} que, por lo tanto, no
distinguirá a una de la otra, sino que tendrá a ambas en consideración
para modificar la realidad, que ahora se sintonizará tanto con el
exterior como con el interior. De este modo, aprovechando el mecanismo
de aprendizaje heredado de sus antepasados aprendices, el conocedor
puede aprender que las indistinguibles naranjas_{naranja} y
pseudo-naranjas son \meaning{venenosa}s en vez de \meaning{apetecible}s.

Eso no es todo. El aprendizaje modifica la realidad creando, variando y
eliminando las relaciones entre los distintos objetos y, también,
creando, modificando y eliminando los propios objetos. De manera que el
aprendizaje de los conocedores es capaz de crear objetos a partir de los
fenómenos externos y, también, a partir de los sentimientos internos.

Aparecen entonces otros tipos de objetos_{objeto} que corresponden a los
adjetivos_{adjetivo} y a los adverbios_{adverbio}. Un ^{objeto
adjetival} como \meaning{venenosa} será despertado por los objetos
nominales_{objeto nominal} no comestibles y vetará
comportamientos_{comportamiento} que causarían su ingestión. Los objetos
adverbiales_{objeto adverbial} hacen lo propio modificando o modulando a
los objetos verbales_{objeto verbal} que, al hacerse presentes, arrancan
los comportamientos del conocedor.


\Section El significado

El ^{comportamiento} de los conocedores_{conocedor} no depende
directamente de la ^{percepción}, sino del significado dado a la
percepción. El significado, a su vez, depende de la ^{emoción}, de modo
que el ^{significado} inmediato o primario es el ^{sentimiento}. Los
otros significados, no primarios, se encontrarán haciendo el recorrido
hacia atrás, esto es, primero los objetos verbales que, por estar
presentes, han arrancado el comportamiento actual, se contagian del
significado puro del sentimiento que han obtenido, después el
significado de los objetos presentes se va propagando epidémicamente a
aquellos otros objetos cuya presencia había causado la presencia de los
anteriores, y así sucesivamente hasta que la ola alcanza finalmente a
aquellos objetos nominales que la propia percepción había hecho
presentes. De este modo, todos los objetos_{objeto} del conocedor van
adquiriendo significado, y su ^{realidad} se convierte en una red con
significados, es decir, en una ^{red semántica}.

Ejemplo: Un ^{antílope} huye y consigue salvar su ^{vida} del ataque de
un predador. Al salvarse se sentirá aliviado,
\meaning{feliz}_{felicidad}, que es un significado primitivo. Se salvó
por ^{huir} ante la presencia de un peligroso ^{león}, y por esto la
línea de unión de los objetos huir, peligroso y león se refuerza para
que, también la próxima vez, pueda hacerse el recorrido de león a
peligroso y de peligroso a huir. Se puede decir de otra manera; para el
antílope, tras la exitosa huída, el león mantiene el significado de
animal peligroso, y peligroso el significado de aquello ante lo que es
preferible huir. Por el contrario, un comportamiento fallido,
\meaning{doloroso}_{dolor}, debilitaría, y hasta podría cambiar de
sentido, las relaciones existentes entre los objetos causantes del
comportamiento.
\begingroup
\abovedisplayskip=6pt
\belowdisplayskip=6pt
\abovedisplayshortskip=0pt
\belowdisplayshortskip=0pt
$$\hbox{Percepción}\longrightarrow
  \column{\strut Realidad\cr
               $\uparrow$\cr
  \hidewidth\strut Emoción\hidewidth\cr}
  \longrightarrow\hbox{Comportamiento}$$
\endgroup

Merced al significado, la realidad puede ser utilizada simultáneamente
por la ^{percepción}, que trae datos del exterior, y por la ^{emoción},
que usa información interna. El ^|significado| casa las condiciones
externas con las condiciones internas.
\begingroup
\abovedisplayskip=6pt
\belowdisplayskip=6pt
\abovedisplayshortskip=0pt
\belowdisplayshortskip=0pt
$$
 \hbox{Condiciones}
  \left\{ \vcenter{\nointerlineskip\halign{#\hfil\crcr
   Internas\cr
   $\quad\;\bigm\updownarrow\hbox{Significado}$\cr
   Externas\cr
  }}\right.
$$
\endgroup


\Section La realidad del conocedor es semántica

Dos conclusiones sobre la realidad del conocedor:
\beginpoints
\point La ^{realidad} del ^{conocedor} es ^{semántica}, tiene
^{significado}. El significado ordena el acceso a la realidad que se
disputan la ^{percepción} del exterior y la ^{emoción} interna.
\point La realidad del conocedor está situada_{situación}, está centrada,
o sea, es espacial_{espacio}, porque distingue lo interno de lo externo:
dentro y fuera, aquí y allí.
\endpoints


\Section La palabra

Supongamos que un antiguo antepasado prehistórico tuviese un tic que le
hacía exclamar `leo' cuando reconocía un ^{león}. Los otros miembros de
su tribu llegarían a aprender que el sonido `leo' pronunciado por él era
un indicio de que había un león presente en las proximidades. Este
aprendizaje en nada se diferencia de aquél estudiado por
^[Pávlov]^(Pávlov1904). Cualquier ^{signo} que ayuda a identificar un
^{objeto}, se asocia a dicho objeto. Éste es un proceder que hemos
adquirido evolutivamente y con el que nuestra habilidad para reconocer
objetos en los fenómenos mejora con la experiencia.

Otro comportamiento heredado, y que compartimos con los monos, consiste
en la ^{imitación}, sobre todo en la infancia, que nos permite evitar
tentativas infructuosas e intentar directamente las soluciones probadas
por la experiencia de nuestros mayores. Podemos así suponer que, en la
generación siguiente, todos los miembros de la tribu decían `leo' al
identificar un león. La ventaja, para la tribu, era que bastaba con que
uno de sus miembros viera un león, para que todos ellos lo percibieran
sin necesidad de captarlo en la retina, o sea, sin sentirlo visualmente.

Fue de este modo como la palabra `leo' pasó a significar \meaning{león}
en dicha tribu. Que fuera esa palabra, o cualquiera otra, es un hecho
fortuito, dado que el mecanismo de asociación de signos a objetos no
tiene otro requisito que el práctico, es decir, que permite cualquier
asociación de ^{nombre} a objeto siempre que resulte útil, como hemos
visto en ^>El signo arbitrario>.
$$
 \hbox{Fenómeno}
 \underbrace{\longrightarrow
  \column{
   \onitself{\strut Objeto$'$}\cr
   $\uparrow$\cr
   \hidewidth\hbox{\strut Sentimiento$'$}\hidewidth\cr
   $\uparrow$\cr
   \hbox{\strut Libido$'$}\cr}
  \longrightarrow}_{\strut\hbox{Conocedor$'$}}
 \hbox{Palabra}
 \underbrace{\longrightarrow
  \column{
   \onitself{\strut Objeto}\cr
   $\uparrow$\cr
   \hidewidth\hbox{\strut Sentimiento}\hidewidth\cr
   $\uparrow$\cr
   \hbox{\strut Libido}\cr}
  \longrightarrow}_{\strut\hbox{Conocedor}}
 \hbox{Acción}
$$

De momento, para nuestra tribu, la palabra `leo' es sólo un signo. Pero
es un signo peculiar porque quien pronuncia la palabra media entre el
^{fenómeno} y quien interpreta el signo. En el aprovechamiento de esta
situación nueva se encuentra el origen del hombre, que es el único
 ^{sujeto} vivo. % (¡Guauu! RMCG20000211).
A continuación desarrollaremos, precisamente, el proceso que lleva,
merced a la palabra, del conocedor mudo al sujeto simbólico.


\Section El lenguaje sígnico

Una única palabra como `leo', acompañada de una indicación de dirección
con el dedo, podía servir para provocar la huida_{huir} de todo el grupo
en la dirección contraria. Este uso de la palabra hablada como ^{signo}
es pues lo bastante útil como para que haya tenido un valor selectivo en
la ^{evolución} de nuestra especie. No es, todavía, un ^{lenguaje
simbólico}. La ^{palabra}, en este estadio preliminar del lenguaje, es
signo, lo cual quiere decir que para quien habla este ^{lenguaje
sígnico}, la palabra es otra propiedad, o indicio, del objeto. Así, la
palabra `leo' se considerará del mismo modo que el ^{color} del ^{león},
su aspecto o su ^{olor}.

^[Vygotsky]^(Vygotsky1934) observó que, para los niños, los nombres son
atributos o propiedades de las cosas, como lo es su color, y no
convenciones; esto es fácil de verificar, ¡y divertido! La lengua que
hablan los niños pequeños es, pues, un ejemplo de lenguaje sígnico, o
sea, no simbólico. Transportando esta prueba ontogenética obtenemos la
correspondiente prueba filogenética; dicho de otro modo, si cada persona
pasa, de niño, por este estadio, es lícito suponer que la especie pasó
también por él.

En un lenguaje sígnico se pueden dar nombres a los objetos que, como
hemos visto, pueden ser nominales_{objeto nominal}, adjetivales_{objeto
adjetival}, verbales_{objeto verbal} y adverbiales_{objeto adverbial}.
La palabra es un signo que hace ^{presente} el objeto al que se refiere.
Resumiendo lo dicho, el lenguaje sígnico se limita a añadir un atributo,
el ^{nombre}, a entidades proporcionadas por otros procesos cognitivos,
y la palabra es siempre una referencia a algo dado, a algo externo a
ella misma.

Las limitaciones del lenguaje sígnico son evidentes, basta tomar
cualquiera de las frases de este libro, sin ir más lejos ésta misma,
para descubrir lo que queda fuera de su alcance expresivo, porque la
^{autorreferencia} es imposible en un lenguaje sígnico. También son
imposibles las preguntas en un lenguaje sígnico.


\Section La verdad

La ^{palabra} es un ^{signo} peculiar porque permite a un ^{conocedor}
mediar entre el ^{fenómeno} y otro conocedor. Así pudo suceder que, en
aquella tribu en la que `leo' significaba \meaning{león}, alquien dijera
`leo' sin percibir un ^{león}, supongamos que por error. Como para los
demás miembros de la tribu, escuchar el sonido de la palabra `leo' era
otra de las maneras de detectar la presencia de un león, resultó que
para ellos el león estaba ^{presente}.

En un principio este error no parece ser ventajoso. Crea un nuevo
^{mundo} en el que el león está presente aquí y ahora aunque, en
^{verdad}, no lo está. Lo que importa es darse cuenta de que, con la
palabra, un conocedor tiene la posibilidad de influir directamente en la
^{realidad} de otro conocedor. Y seguro que la ^{evolución} sacó pronto
partido de esta extraordinaria potencia.

La palabra, al ser una propiedad mediada, sirve para traer el ^{objeto}
al presente, al aquí y ahora, aunque no lo esté, o siendo
escrupulosamente preciso, aunque sólo esté presente un signo del objeto,
su ^{nombre}. El uso de la palabra como signo tiene el poder de hacer
presente lo que, sin ella, no está presente. La palabra va más allá de
la ^{atención}, que modifica el presente; la palabra inventa el
presente. Y, con esta invención, aparece la verdad y la falsedad. Sin
palabras no hay mentiras_{mentira}.


\Section La comunicación

La ^{comunicación} entre conocedores se basa en la intromisión de un
conocedor, con la ^{palabra}, en el proceso perceptivo de otro. Para
sacar provecho de la situación fue necesaria, aún, una revolución más;
la denominaremos la revolución del ^{proto-sujeto}. Creo, sin embargo,
que una vez conseguida la mediación en la ^{percepción}, sólo era
cuestión de tiempo que la ^{evolución} encontrase la manera de
explotarla, porque para conseguirlo basta con poder prescindir de la
fuente de la percepción, esto es, del ^{fenómeno}.
$$\underbrace{
 \phantom{\longrightarrow}
 \column{
  \onitself{\strut Objeto$'$}\cr
  $\uparrow$\cr
  \hidewidth\strut Sentimiento$'$\hidewidth\cr
  $\uparrow$\cr
  \strut Libido$'$\cr}
 \longrightarrow}_{\strut\hbox{Proto-sujeto$'$}}
  \hbox{Palabra}
  \underbrace{\longrightarrow
   \column{
    \onitself{\strut Objeto}\cr
    $\uparrow$\cr
    \hidewidth\strut Sentimiento\hidewidth\cr
    $\uparrow$\cr
    \strut Libido\cr}
  \longrightarrow }_{\strut\hbox{Proto-sujeto}}
  \hbox{Acción}
$$

Y, una vez conseguido, si el proto-sujeto que habla gana la ^{atención}
del que oye, entonces la ^{acción} que ejecuta quien escucha tiene su
origen en la libido del que habla. Con la palabra se puede controlar el
^{comportamiento} del otro proto-sujeto.

La palabra ocupa en la comunicación el lugar que el fenómeno ocupa en la
percepción. La diferencia entre la percepción y la comunicación es que,
en la percepción, el fenómeno es el origen de los datos, mientras que,
en la comunicación, el origen no es la palabra, sino otro ^{objeto}. Que
sólo los subjetivistas observemos esta diferencia, prueba que la
simplificación de aplicar a la percepción el modelo de la comunicación
ha tenido históricamente un enorme éxito. El ^{objetivismo} postula que,
también en la percepción, existen objetos externos que son los causantes
de los fenómenos percibidos. Esta hipótesis es innecesaria, por lo que
su efecto sólo puede ser distorsionante.


\Section El símbolo

El caso es que la ^{palabra} pasa a desempeñar dos funciones: la
original de servir de ^{signo} de un fenómeno externo, aunque sea signo
obtenido por mediación de otro; y otra función nueva por la que la
palabra se refiere a un objeto de quien la pronuncia, y que no es signo
de un fenómeno externo, sino que es signo del propio objeto interior.
Diremos, en el primer caso, que la palabra es signo y, en el segundo,
que la palabra es ^|símbolo|. Ya en estas condiciones le conviene al
proto-sujeto distinguir ambos usos de la palabra, de modo que en un caso
podría decir `hay ^{agua}' y en el otro `quiero agua'.

Lo revolucionario consiste en que, con la palabra, el objeto, de ser un
mero constructo mental de uso propio, pasa a tener un carácter externo
de uso compartido y, sobre todo, se convierte en parte de una
^{realidad} ampliada. Pero esto requiere más explicaciones.


\Section El ensimismamiento

Visto que con la palabra un ^{proto-sujeto} puede controlar el
^{comportamiento} de otro proto-sujeto, basta con que se hable a sí
mismo para que controle su propio comportamiento. Esto puede parecer, en
un principio, superfluo, dado que el proto-sujeto ya controlaba su
comportamiento, ¿quién, si no? Sin embargo, en el proto-sujeto, como en
el ^{conocedor}, el comportamiento depende tanto de la ^{percepción}
como de la ^{emoción}. Hablándose a sí mismo, y dado que la palabra
pronunciada en voz alta se oye y se escucha, el ^{sistema emocional}
consigue ocupar efectivamente la percepción. Teniendo acaparadas las dos
vías de entrada a la ^{realidad}, la ^{libido} adquiere el dominio
completo de la ^{cognición}.

Así que el siguiente paso fue utilizar la ^{palabra} hablada para
escucharse a sí mismo. De este modo el nuevo sujeto conseguía para sí el
control que, también con la palabra, había conseguido el proto-sujeto
sobre el otro.
$$
 \vbox{\halign{\hfil#\hfil\cr\hfill$\swarrow$\cr Palabra\cr}}
 \underbrace{\longrightarrow
  \column{
   \onitself{\strut Objeto}\cr
   $\uparrow$\cr
   \hidewidth\hbox{\strut Sentimiento}\hidewidth\cr
   $\uparrow$\cr
   \hbox{\strut Libido}\cr}
  \longrightarrow}_{\strut\hbox{Sujeto}}
 \vbox{\halign{\hfil#\hfil\cr$\nwarrow$\hfill\cr Palabra\cr}}
$$
Al atender_{atención} a lo que uno mismo dice, el ^{sentimiento}
generado por la libido domina completamente la cognición. Ejemplo: un
niño se lastima y llora, pero no ve a su madre, de manera que al rato se
tranquiliza y sigue jugando. Cuando vuelve la madre, el niño pronuncia
la palabra `^{pupa}' y reanuda desconsoladamente el llanto. Atendiendo a
la palabra `pupa' la percepción apoya a la emoción, que así ocupa
completamente la cognición del niño.

Denominamos ^{ensimismamiento} a esta situación durante la cual el
^{sujeto} ha roto la conexión con el exterior, y que es una conversación
consigo mismo que le sirve para alcanzar el completo dominio de sí
propio. Dicho de otra manera: mientras que el ^{conocedor simple} está
necesariamente engranado al ^{exterior}, el sujeto puede, merced a la
^{palabra}, desencadenarse de él.


\setbox0=\hbox{$\underbrace{\phantom{\longrightarrow}%
   \subject
   \phantom{\longrightarrow}}_{\hbox{Sujeto}}$}
\dimen0=\hsize \advance\dimen0 by -\wd0 \advance\dimen0 by -1em
\dimen2=\wd0 \advance\dimen2 by 1em
\nointerlineskip
\hbox to0pt{\kern\dimen0\kern1em\vbox to0pt{\kern2pc\box0\vss}\hss}
\nointerlineskip

\Section El pensamiento

\begingroup\rightskip=\dimen2
Pronto se interiorizó la palabra. Llamaremos ^{idea} a la palabra
interiorizada. Llamamos ^|pensamiento|, o ^|reflexión|, al bucle que va
del objeto al objeto a través de la idea. Es ^{idear} pasar del objeto a
la idea, y ^{conceptuar} es volver de la idea al objeto. Si la
comunicación construye nuevos mundos_{mundo} y distintos
presentes_{presente} en los cerebros de otros, el pensamiento hace lo
mismo en el ^{cerebro} propio.
\par\endgroup

\breakif2

El pensamiento es el habla muda, interiorizada. Esto concuerda con
^[Vygotsky]^(Vygotsky1934), que observó, en los niños, que el habla es
anterior al razonamiento simbólico.
$$\hbox{Pensamiento} = \hbox{Habla muda}$$

Sólo tenemos una ^{boca}, quiero decir que sólo podemos decir una cosa
cada vez, y, tal vez por ello, nuestro pensamiento consciente es, como
el habla, discursivo y secuencial. Además, interesa que exista un
proceso único que dirija a los otros procesos, y cuando no ocurre esto,
como en el caso de la ^{esquizofrenia}, el ^{sujeto} tiende a
comportarse paradójicamente. Así se explica que el pensamiento sea
secuencial aunque tenga lugar sobre una maraña de procesos cognitivos
simultáneos, o sea, paralelos.


\Section La voluntad

Los sujetos filtran tres veces los datos que reciben, porque cada
entrada a la red de objetos que hemos venido denominando ^{realidad}
selecciona una parte.
$$\hbox{Fenómeno}
  \underbrace{\longrightarrow\subject
   \longrightarrow}_{\hbox{Sujeto}}
  \hbox{Acción}
\abovedisplayskip=\abovedisplayshortskip
\belowdisplayskip=\belowdisplayshortskip
$$

La ^{percepción} realiza el filtro más antiguo, heredado de los
adaptadores, y sólo deja los objetos presentes. Además, el ^{sistema
emocional}, o ^{emoción}, heredado de los conocedores, atiende
únicamente a los objetos presentes que le interesan al sujeto, los
objetos significativos. Y, el ^{pensamiento}, propio de los sujetos,
dispone de una vía por la que las ideas también condicionan la
^{realidad}. A esta última vía se le denomina ^|voluntad|.

Queda el ^{aprendizaje}, que no aportó filtro a la realidad, sino la
posibilidad de redefinir la propia realidad. De modo que si, en el corto
plazo, las tres entradas a la realidad filtran o seleccionan objetos, a
más largo plazo y merced al aprendizaje, contribuyen a modificar la
^{red de objetos} creando, destruyendo, uniendo, separando,
fortaleciendo o debilitando los objetos y las relaciones entre los
objetos.

De entre las entradas, el pensamiento se erigió en el controlador del
comportamiento al máximo nivel porque es la manera que la ^{libido} del
^{sujeto} encontró de ocupar su propia percepción para poder ampliar a
voluntad la ^{realidad}. Si el habla sirve para controlar a otro sujeto,
el pensamiento y la voluntad sirven para el ^{autocontrol}.

La atrofia o hipertrofia de cualquiera de las cuatro facultades que
controlan la red de objetos---percepción, aprendizaje, emoción o
pensamiento---produce un comportamiento disfuncional que estudia la
psicología patológica.


\Section La consciencia

La palabra puede ser dicha y oída. Por esto, el ^{pensamiento} va del
^{objeto} al objeto y es recursivo desde el principio, y de ahí que
también se llame reflexión. La ^{recursividad} hace posible la
^{introspección}. Veamos cómo.

Dado que los signos sirven para reconocer un objeto en un ^{fenómeno},
cuando la ^{palabra} es ^{símbolo}, esto es, cuando la palabra es
^{signo} de un objeto interior, entonces lo que hace es reconocer un
objeto en otro objeto. Y puesto que reconocer objetos es la manera,
diseñada por la evolución, de ^{ver} los fenómenos externos o, en
general, de percibirlos, al reconocer objetos en objetos los sujetos
tenemos la posibilidad de ver los objetos internos. Aquí debe observarse
con precisión la diferencia entre sentir y percibir, o ver; recuérdese
lo dicho en ^>La realidad objetiva es subjetiva>, y en ^>El adaptador>.
Los objetos que vemos son los que están presentes, de modo que los
sujetos vemos la ^{realidad} ^{presente}. Lo notable no es esto, sino
que los conocedores_{conocedor} simples no vean la realidad presente. Lo
sorprendente es que solamente los sujetos_{sujeto} veamos la realidad
presente. Los conocedores simples están en la realidad presente, pero no
la ven.

Esto vale tanto para la ^{idea}, que es la palabra muda, como para la
palabra dicha; si con la idea vemos nuestros propios objetos, con la
palabra dicha vemos los objetos de otro. Ocurre, sin embargo, que lo
dicho puede ser ^{mentira}. Esto sucede cuando el sujeto descubre que le
interesa que quien le escucha crea que piensa lo que no piensa. Menos
frecuente, pero más peligroso, es el ^{autoengaño}, que no estudiaremos
aquí a pesar de su enorme interés.

\breakif1

Entonces, si los adaptadores_{adaptador}, los aprendices_{aprendiz} y
los conocedores_{conocedor} disponen de unas gafas que añaden
etiquetas_{etiqueta} a lo que captan sus sentidos, y esto es ver, los
sujetos podemos ver, por añadidura, las propias etiquetas porque podemos
etiquetar las etiquetas. Las palabras son las etiquetas de los objetos,
que son las etiquetas de los fenómenos. Nosotros podemos, de este modo,
ver nuestros propios pensamientos. Esto no quiere decir que veamos todos
nuestros procesos cognitivos, pero aquéllos que sí vemos, son lo que
denominamos ^|consciencia|.

Igual que un ^{ojo} puede captarse a sí mismo en un ^{espejo}, así puede
verse un objeto a sí mismo reflejado en el pensamiento. Usando esta
analogía, por la que ya hemos hecho a `pensamiento' sinónimo de
`reflexión', podemos definir idea con exactitud, y con el permiso de
^[Platón]: una ^{idea} es un ^{objeto virtual}.
$$\hbox{Idea} = \hbox{Objeto virtual}$$


\Section El inconsciente

La ^{consciencia} es la parte de la ^{cognición} que podemos ^{ver}
gracias a su ^{simbolización}, pero hay otra parte que no podemos ver,
como descubrió ^[Freud]^(Freud1900). No podemos ver lo que ocurre antes
de la construcción de los objetos_{objeto}. Tampoco podemos ver lo que
sucede antes de la formación de los sentimientos_{sentimiento}. Es
seguro que hay muchos otros procesos cognitivos que tampoco vemos, dado
lo tardío de la simbolización en la ^{evolución} darviniana.

¿Por qué no se recuerda el primer año de vida? Porque sólo se recuerda
conscientemente lo simbolizado, es decir, lo hablado, lo pensado, lo
visto, pero nunca lo sentido y no visto. No se recuerda lo captado pero
no etiquetado, quizás, porque no es posible recuperarlo a la consciencia
sin su ^{etiqueta}. La consciencia y la ^{simbolización} son como las
dos caras de la misma moneda.

\breakif5

\Section La cosa y el concepto

En el ^{sujeto}, las ideas_{idea} permiten la ^{reflexión} que, tomadas
como datos, suponen ampliar la entrada y la salida de la ^{red de
objetos}. La red de objetos del sujeto puede producir comportamientos,
entre ellos el ^{habla}, y también pensamientos, o sea, habla muda.
Además, las ideas, como los fenómenos y los sentimientos, determinan qué
objetos están presentes y modifican la red de objetos. Así, la red de
objetos es el centro subjetivo de dos bucles: el nuevo ^{bucle teórico},
que denominamos reflexión o pensamiento; y el antiguo ^{bucle práctico},
que, ya desde el ^{adaptador}, pasa por la acción, el entorno exterior y
el fenómeno.

La palabra oída es un ^{fenómeno}, y como tal utiliza los mecanismos de
la ^{percepción}. Tampoco la palabra dicha se diferencia de otras
acciones que se ejecutan como resultado de los procesos cognitivos. Es
de suponer que la palabra interiorizada, la idea, utiliza los mismos
canales que la palabra dicha. El primer argumento en favor de esta
hipótesis es que el ^{pensamiento} es evolutivamente muy reciente, por
lo que poco ha podido diferenciarse del habla. El segundo es empírico:
cuando le hablan a alguien que está pensando en algo no relacionado con
lo que oye, no entiende lo que le dicen. Es típica la excusa `perdona,
pero estaba pensando en otra cosa', que demuestra que la palabra oída y
el pensamiento tienen una única vía de acceso al ^{presente} del sujeto.

Por lo tanto, aunque el bucle teórico y el práctico son distintos, su
tratamiento, por los procesos de ^{modelación} de la red de objetos, que
denominamos realidad, es el mismo. Este ^{oportunismo}, que es típico de
la ^{evolución} darviniana, permite que el pensamiento amplíe la
^{realidad}. Es decir, que los mismos procesos de ^{aprendizaje} que
determinan que \meaning{frutal} merece ser un objeto distinto de
\meaning{árbol}, también establecen que \meaning{par} merece ser un
objeto distinto de \meaning{número}.

Entre los objetos_{objeto} del sujeto, los hay construidos por la
^{percepción} y el ^{aprendizaje} a partir de los fenómenos
experimentados prácticamente, son los objetos nominales_{objeto
nominal}. Otros de los objetos primitivos, ya utilizados por los
adaptadores, son los verbales_{objeto verbal}, que controlan el
^{comportamiento}. Con la aparición del sistema emocional de los
conocedores, los objetos nominales y verbales adquirieron ^{significado}
y aparecieron los objetos adjetivales_{objeto adjetival} y
adverbiales_{objeto adverbial}, que debían su existencia a los
sentimientos. Y con el pensamiento del sujeto, todos los objetos
anteriores, nominales, verbales, adjetivales y adverbiales, que ya
tenían significado, adquirieron una carga conceptual, y además
aparecieron otros objetos cuyo existir es producto exclusivo de las
ideas del pensamiento. A los objetos nominales, verbales, adjetivales y
adverbiales los denominaremos cosas_|cosa|; a los otros, que son objetos
teóricos_{objeto teórico}, los denominaremos conceptos_|concepto|.
$$\hbox{Objeto}\cases{\hbox{Concepto}\cr\hbox{Cosa}}$$


\Section El mundo

Los conceptos son objetos peculiares porque su existir está fundado
sobre los propios objetos_{objeto}. Los objetos nominales_{objeto
nominal} existen por la ^{percepción}, los verbales_{objeto verbal} por
el ^{comportamiento}, los adjetivales_{objeto adjetival} por la
percepción y la ^{emoción}, y los adverbiales_{objeto adverbial} por la
emoción y el comportamiento, pero los conceptos existen por las
ideas_{idea} del pensamiento, que no son más que etiquetas de objetos,
así que los objetos conceptuales existen por los objetos. Las
cosas_{cosa} vienen dadas, y los conceptos_{concepto} son, eso,
concebidos a ^{voluntad}.

Por ejemplo, según estas definiciones, ^{piedra} es una cosa, y también
lo es rugosa, ^{huir} y ahora, aunque las palabras `piedra', `rugosa',
`huir' y `ahora' son conceptos, como aclararemos en la próxima sección,
_>La existencia y la referencia>.

Para asentar la diferencia entre las cosas reales, que vienen dadas, y
los conceptos teóricos, con los que ampliamos la realidad a voluntad, se
distingue la ^{teoría} de la ^{realidad}. Y se llama ^|mundo| a la
realidad ampliada que incluye tanto las cosas, que son los objetos que
conforman la realidad propiamente dicha, como los conceptos, que son los
objetos teóricos que el pensamiento concibe.
$$\hbox{Mundo}\cases{\hbox{Teoría}\cr\hbox{Realidad}}$$

Al hacer esta distinción se pierde la equivalencia estricta, establecida
en ^>La realidad>, entre la red de objetos y la realidad. La red de
objetos del sujeto se identifica con su mundo. Y como su mundo incluye a
la realidad propiamente dicha, resulta que la realidad es sólo una parte
de la red de objetos del sujeto.
$$\hbox{Mundo} \left\{ \vcenter{\def\:{\hskip0.65em}\halign{#\cr
    Teoría\:\hfil Conceptos\cr
    Realidad\:\hfil Cosas\cr} } \right\}
  \hbox{Red de objetos}$$

\breakif2

El ^{bucle teórico}, con no ser más que una tercera entrada y una
segunda salida de la ^{red de objetos}, complica enormemente la
realidad, y la amplía. El bucle teórico funciona como un ^{espejo} que
refleja los objetos sobre los propios objetos. Con esta ayuda, el
aparato destinado a ver es visto por sí mismo.
$$\hbox{Percepción}\longrightarrow
   \vbox{\everycr={}\tabskip=0pt \lineskip=0pt
    \halign{\hfil\strut#\hfil\cr
     \hidewidth Pensamiento\hidewidth\cr
     $\downarrow$\hfil$\uparrow$\cr
     \column{\strut Mundo\cr$\uparrow$\cr
      \hidewidth\strut Emoción\hidewidth\cr}\cr}}
  \longrightarrow\hbox{Comportamiento}$$

\kern-12pt

\Section La existencia y la referencia

Cuando se amplía la ^{red de objetos}, los nuevos objetos obtienen su
^{significado} por el proceso retrógrado ya visto; en ^>El sentimiento>,
y en ^>El significado>; que va desde los sentimientos a los
comportamientos y de los comportamientos a los objetos. Esto es general
y vale también si el nuevo objeto es un ^{concepto}. Con los conceptos
puede ocurrir, sin embargo, algo imposible con las cosas, a saber, que
no haya camino alguno con el que alcanzar el concepto desde los
sentimientos. La razón es simple, hay conceptos que no provocan
comportamiento alguno, sino reflexión. Estos conceptos puramente
teóricos que no originan acciones no pueden causar ^{dolor}, ni
^{placer}, y, por esta razón, no tienen significado.

La cuestión se complica porque la reflexión permite ver la propia red de
objetos. Es posible, de esta manera, que la palabra `^{agua}', que en un
principio era solamente uno de los signos del objeto agua, se haga ella
misma objeto. Porque es posible, podemos ver que la palabra `agua' tiene
dos sílabas, por ejemplo. Para distinguir el ^{objeto} agua, que es una
cosa, del objeto de la ^{palabra} `agua', que es un concepto, a este
último lo notaremos palabra `agua'. La cosa agua moja y, sin embargo, la
palabra `agua' tiene dos sílabas ¡y no moja!

En estas circunstancias decimos que la palabra `agua' se
refiere_{referencia} la cosa agua, o que la palabra `agua' toma el
significado de la cosa agua, o, resumidamente, que el agua
existe_{existencia}. Los dos objetos, la cosa agua y la palabra `agua',
están muy relacionados en la red de objetos, ya que si oímos la palabra
`agua', traemos directamente al presente la cosa agua, y si percibimos
signos que hacen presente la cosa agua, inmediatamente podemos
pronunciar la palabra `agua'; ambos objetos se hacen presentes
simultáneamente.

Hasta aquí la referencia es una operación binaria, porque hace uso de
dos objetos, uno de ellos una palabra objetivada. Pero esto no vale, en
general, para el lenguaje simbólico. Cada palabra es, efectivamente, un
objeto conceptual, o sea, un concepto, pero no es cierto que cada
concepto se refiera necesariamente a alguna cosa. Más adelante, en ^>La
paradoja>, hablaremos de las palabras sin referente ni significado.


\Section Primer atisbo de libertad

Al poder objetivar la ^{palabra}, que era ^{signo}, nos liberamos, de
una manera literal, del ^{presente}, del aquí y ahora; lo hemos visto en
^>El ensimismamiento>. La ^{libertad} no puede ser ^{pasado}, que ya
está acabado, ni presente, que no es sino un límite, pero tampoco es lo
mismo que el ^{futuro}. Los objetos pueden contener modelos
predictivos_{previsión}, como la campana del ^{perro} de ^[Pávlov] que
anuncia la llegada inminente, pero futura, de la ^{comida}. El futuro es
una condición necesaria de la libertad, pero no suficiente.

En cambio, la palabra hecha símbolo es una condición suficiente para la
libertad. Porque no hay ^{libertad} sin que haya varios mundos posibles,
y la palabra, al interferir entre el ^{fenómeno} y la ^{realidad},
construye esos otros mundos_{mundo}; véanse ^>La verdad>, y ^>El mundo>;
tampoco haría daño leer a ^[Goodman]^(Goodman1978).


\Section El lenguaje simbólico

La ^{palabra} fue primero ^{signo} y después ^{símbolo}. Esto se
corresponde con los dos usos de la palabra, a saber, como signo de un
^{fenómeno} exterior y como símbolo de un ^{objeto} interior. Y por esto
distinguimos dos estadios en el desarrollo del lenguaje: el ^{lenguaje
sígnico} que utilizaba aquella tribu en el que la palabra `leo' era un
signo más del ^{león}, visto en ^>La palabra>, y el ^{lenguaje
simbólico} en que hay palabras, como `verbo', que son signos de las
propias palabras.

La diferencia entre el lenguaje sígnico y el lenguaje simbólico es que
la ^{referencia} de las palabras de este último no tiene limitación
alguna. En concreto, una palabra simbólica puede referirse a otra
palabra y, aún más importante, una palabra puede no tener referente.

Según ^[Vygotsky]^(Vygotsky1934), el proceso de maduración intelectual
de los niños, que concluye aproximadamente sobre los doce años, consiste
en la simbolización de todo, incluido el propio proceso de
simbolización. Traducido a nuestra terminología, y si ^[Vygotsky] está
en lo cierto, el lenguaje simbólico se asienta alrededor de los doce
años.

Sobrepasar el estadio del lenguaje sígnico permite que las palabras
también indiquen los usos del objeto que sobrepasan su valor
referencial. Por ejemplo, no es lo mismo afirmar que el ^{agua} está
presente, `hay agua', que expresar el ^{deseo} de que el agua esté
presente, `quiero agua', o que preguntar en qué lugar hay agua, `¿dónde
hay agua?' Solamente el primer uso es sígnico, los otros son usos
simbólicos. Para distinguir qué uso se está haciendo, los sujetos han de
añadir, a la palabra que se refiere al agua, otras palabras o
modificadores. Algunas de estas palabras, como la palabra `dónde', no
tienen referente; diremos de ellas que tienen un valor puramente
sintáctico.

La ^{sintaxis} es lo que distingue al lenguaje simbólico del lenguaje
sígnico. Aunque su origen, dejar que la referencia no tenga límites, es
poco espectacular, su importancia es enorme. Por ejemplo, puesto que una
palabra simbólica, o ^{símbolo}, puede referirse a otra palabra, la
propia palabra se hace ^{objeto}. Y como objeto que es, el símbolo
admite distintos signos_{signo} para su identificación. Por esto la
palabra simbólica puede ser, además de dicha, escrita.


\Section La escritura

La palabra, para el ^{sujeto} simbólico, es un objeto conceptual. Todo
objeto puede ser reconocido por distintos signos; la palabra del sujeto
también. Por esta razón el sujeto puede utilizar palabras habladas,
escritas y, como mostró ^[Braille], hasta palpadas.

La ^{palabra escrita} tiene una característica que no tiene la ^{palabra
hablada}, perdura. Por esta razón puede utilizarse para dirigir, durante
más tiempo, la ^{atención}. ^[Sócrates], muerto hace dos mil quinientos
años, mantiene nuestra atención porque ^[Platón] escribió lo que su
maestro había dicho.


\Section La oración

En el ^{lenguaje sígnico} bastaba pronunciar una ^{palabra} para decir
que el ^{objeto} al que la palabra se refería estaba ^{presente}, porque
el único uso de la palabra era servir de ^{signo}. Pero cuando el
^{habla} empezó a transmitir otros aspectos de la ^{cognición}, su
^{expresividad} creció y lo hizo a costa del crecimiento de la unidad de
habla, que hasta entonces había sido la palabra, y que pasó a ser la
oración.

La ^|oración| es la unidad de habla, y consiste en una secuencia de
palabras. Que la oración sea una secuencia viene impuesto por la
limitación de no poder emitir más que un sonido vocal en cada momento.
Puede utilizarse la ^{entonación} para diferenciar distintos usos de una
misma palabra, y por ejemplo el ^{chino} la usa con profusión, pero este
método tiene sus limitaciones. Es así que la oración utiliza palabras,
modificadas o no, en forma de secuencias entonadas para intentar
expresar el estado cognitivo del sujeto.

Puesto que la oración es la unidad de habla, en principio, un
^{concepto} se expresa con, al menos, una oración. Pero el concepto se
puede idealizar y, tras la ^{reflexión}, puede quedar nombrado con una
única palabra cuya ^{definición} es el concepto al que da ^{nombre}.

De modo que la palabra `agua' que en el ^{lenguaje sígnico} servía para
decir que había ^{agua} presente, en el ^{lenguaje simbólico} se
convierte en la oración `hay agua', para diferenciarla de `quiero agua'
que expresa que mi sistema emocional ha determinado que tengo ^{sed} y
calcula que la cosa agua solucionaría el ^{problema}. Una oración
imperativa, como `tráeme agua', lo solucionaría definitivamente si
efectivamente consiguiera convencer a mi interlocutor, que al decir
`quizás haya agua' expresa su duda sobre la feliz ^{solución} del
problema. Si en vez de pedir directamente la solución, pido ayuda para
la ^{resolución} del problema, entonces he de plantearlo explícitamente
con una oración interrogativa, que en este caso podría ser `¿dónde hay
agua?' Cuando, por fin, encuentro agua, puedo exclamar `¡agua!', porque
he solucionado el problema. La oración exclamativa es un vestigio del
lenguaje sígnico.


\Section La sintaxis

En las oraciones anteriores, la ^{palabra} `agua' es la única que se
refiere_{referencia} a una ^{cosa}; las otras palabras, como `hay',
`quizás' o `dónde', son conceptos_{concepto} que no se refieren a cosa
alguna. Además de las palabras, la oración escrita utiliza algunos
signos, como los signos de ^{interrogación} o de ^{exclamación}, que
sirven para anotar la ^{entonación} especial que se emplea al decirla.
Por último, las palabras se ordenan en la oración atendiendo a su
clasificación, como verbo o sustantivo, porque, en ocasiones, el orden
sirve para distinguir los varios usos de las palabras. Todo cuanto se
relaciona con la oración, como tal oración, se denomina ^|sintaxis|, y
es diferente en cada ^{lengua}.

\breakif2

La ^{oración} expresa, parcialmente, el estado cognitivo del ^{sujeto}.
Pero, mientras el estado cognitivo del sujeto está compuesto por varios
procesos que actúan simultáneamente, es decir, en ^{paralelo} (véase
^>La realidad>), la oración es una única secuencia, o ^{serie}, de
palabras. Y hasta las palabras se construyen pronunciando
secuencialmente sonidos. De manera que ^{idear}, tal como se ha definido
en ^>El pensamiento>, es, básicamente, un proceso serializador, o
secuenciador, mientras que ^{conceptuar}, el proceso complementario, es
un proceso paralelizador. Decimos que el ^{motor sintáctico} ejecuta
estos dos procesos.


\Section El problema

El ^{símbolo} es un ^{signo} liberado. Liberar al signo de la
^{percepción} exterior e interior permite enunciar problemas y
resoluciones. Ésta es una consecuencia inesperada, como lo suelen ser
todos los descubrimientos de la evolución_{oportunismo}, pero de probado
valor (de momento) para la supervivencia. El lenguaje simbólico es capaz
de expresar problemas, `¿cómo puedo comerme una ^{nuez}?', y
resoluciones, `golpéala con una piedra hasta que se abra'. También
permite expresar soluciones, pero esto no es una novedad, ya que también
pueden comunicarse soluciones, como `¡huid!'_{huir}, en los lenguajes
sígnicos. Algunos graznidos de las aves que sirven para poner en fuga a
la bandada expresan, también, `¡huid!'; véase ^[Lorenz]^(Lorenz1949).
Para expresar soluciones no es precisa la sintaxis, como muestra el
ejemplo anterior.

Distinguimos la resolución de la solución de un ^{problema}. La
^{resolución} es el proceso que permite la ^{solución} del problema. Si
el problema es cómo comerse una nuez, golpearla con una piedra hasta
abrirla es la resolución, y la nuez abierta y comida es la solución.

Pero, ¿por qué el lenguaje simbólico es capaz de expresar problemas?
Ocurre que con el ^{lenguaje simbólico} vemos_{ver} nuestro
^{pensamiento} consciente, que es una parte del proceso cognitivo, y
resulta que el propósito del ^{cerebro} es solucionar problemas.
Recordemos que el propósito del ^{sistema nervioso} es determinar, dado
lo que percibe y su propio estado, qué comportamiento del cuerpo
conviene en cada momento; véase ^>El adaptador>. Luego, la razón por la
cual los lenguajes simbólicos permiten expresar problemas, soluciones y
resoluciones, es porque con el lenguaje simbólico se ve parcialmente la
maquinaria de plantear y resolver problemas que está en el cerebro.

En el ^{mundo} simbólico del sujeto cabe la ^{realidad} semántica
heredada del conocedor y mucho más. Caben los problemas, los
deseos_{deseo}, las dudas, las preguntas, y las resoluciones, los
algoritmos, las herramientas, los planes, y las soluciones, los
comportamientos, los procesos, las acciones. Los ingredientes de los que
está constituida la parte teórica del mundo, véase ^>El mundo>, son
éstos, y no son otros, porque en el lenguaje simbólico pueden ser
expresados los problemas, las resoluciones y las soluciones.

En un lenguaje simbólico se pueden plantear preguntas, como ¿por qué
moriré? o ¿qué es la ^{vida}?, que no existen fuera del mundo
sintáctico. Tampoco la ^{libertad}.


\Section El pronombre

Para expresar un ^{problema} han de poder expresarse sus dos
componentes, que son, como veremos en _>La teoría del problema>, la
^{libertad} y la ^{condición}. Las condiciones primarias las pone la
percepción, el comportamiento y la emoción. La ^{percepción} presenta
las condiciones externas, esto es, el estado del universo, y la
^{emoción} determina las condiciones internas, o sea, las necesidades y
los deseos_{deseo}. La otra condición es que el ^{comportamiento} que
el cuerpo haya de realizar para solucionar el problema sea uno de los
posibles.

El ^{lenguaje simbólico} usa palabras vacías, sin referente_{referencia}
ni ^{significado}, para expresar la libertad. No puede ser de otro modo
si esas palabras han de representar la libertad del problema. En la
oración interrogativa `¿qué hago?', la palabra `qué' es un ^{pronombre}
que no se refiere a ningún comportamiento concreto, que no tiene, por
tanto, significado alguno. Es necesario que no se refiera a nada, o no
habría manera de poder expresar el problema, que consiste, precisamente,
en que no se sabe qué es lo que me conviene hacer.

Que la palabra `yo' sea un pronombre, significa que se utiliza para
denotar la libertad de un problema. La libertad del ^{problema del
sujeto} se expresa con la palabra `^{yo}'.


\Section El artículo

En castellano el ^{artículo} sirve para determinar el ^{sustantivo},
esto es, para expresar si lo es enteramente o si debe ser tratado casi
como un ^{pronombre}. Así la frase `una piedra' expresa que la
referencia es indeterminada, aunque no tan completamente como si se
empleara un pronombre interrogativo, tal cual `qué'.


\Section La gramática

Existen varios tipos de oraciones_{oración, tipos}, que son diferentes
porque tienen su origen en momentos evolutivos distintos. En primer
lugar, ya las hemos visto (en _>La oración> y _>El problema>), se
encuentran las oraciones exclamativas que los sujetos heredamos de los
conocedores. Los resultados de la percepción se expresan con oraciones
enunciativas que describen el estado de las cosas. Para referirse
directamente a los comportamientos, debe utilizarse una oración
imperativa. Los sentimientos utilizan oraciones desiderativas para
expresar deseos_{deseo}, o enunciativas para sugerir que las necesidades
se imponen al sujeto como venidas de fuera de él. Por último y como
propias de los sujetos, están la oración dubitativa, que refleja un
distanciamiento entre el ^{pensamiento} y la ^{realidad} del sujeto, y
la oración interrogativa, que es la que mejor expresa la naturaleza
inquisitiva de la reflexión.

También los diferentes tipos de palabra se relacionan con la evolución
cognitiva. Los sustantivos_{sustantivo} provienen de los objetos
nominales_{objeto nominal} y los verbos_{verbo} de los objetos
verbales_{objeto verbal} que tienen su lejano origen en los adaptadores;
y los adjetivos_{adjetivo} y los adverbios_{adverbio} de los objetos
adjetivales_{objeto adjetival} y adverbiales_{objeto adverbial} de los
conocedores. Los pronombres_{pronombre}, y los artículos_{artículo},
aparecen cuando el sujeto quiere expresar problemas. Otros tipos de
palabras sirven para conformar la propia oración, y son en castellano
las conjunciones_{conjunción} y las preposiciones_{preposición}, que
pretenden expresar la ^{concurrencia} de la realidad, y del ^{mundo},
que el habla secuencial no puede alcanzar sin estos artificios.

Dada la naturaleza necesariamente recursiva_{recursividad} de la
reflexión, la sintaxis es también recursiva. Esto hace posible que una
oración contenga a otras oraciones, llamadas oraciones subordinadas, que
hacen las veces de sustantivos o adjetivos o adverbios.

No deben entenderse estas afirmaciones como estrictamente gramaticales.
Quiero decir que, en una oración como `quiero agua', el verbo gramatical
`quiero' actúa como adjetivo porque la oración completa equivale a las
frases `agua querida' y `agua deseable', en donde el verbo gramatical
`quiero' ha pasado a ser el adjetivo `deseable'. Del mismo modo, la
oración `te aseguro que tu hija te miente' es dubitativa porque tiene la
misma estructura cognitiva que `creo que tu hija te miente'; ambas
expresan una evaluación reflexiva de la realidad.


\Section Todo cambia

La diferencia que se establece entre la ^{permanencia} de la ^{cosa}
vista y el ^{cambio} del ^{comportamiento} permite distinguir entre
^{sustantivo} y ^{verbo}. Esta diferencia puede llegar a ser
convencional. Por ejemplo, `fuego' es un sustantivo, lo que significa
que es algo que permanece y sin embargo, como le gustaba observar a
^[Heráclito]^(Brun1965), el ^{fuego} es un proceso en continuo cambio.
Por otro lado, `arder' es un verbo y, por lo tanto, denota un cambio.
`Arder' y `fuego' son semánticamente sinónimos, y por esto la oración
`el fuego arde' es una ^{tautología}. No son redundantes porque `fuego'
puede ocupar la posición sintáctica del sujeto, y `arder' la del
predicado. Esto prueba que la diferencia entre la permanencia y el
cambio establecida discerniendo los sustantivos de los verbos puede ser
meramente gramatical, o sea, convencional, y por este motivo no sirve
para distinguir lo que permanece de lo que cambia.

``Todo cambia''
 ($\pi\acute\alpha\nu\tau\alpha \; \acute\rho\epsilon\tilde\iota$)
dijo ^[Heráclito]. El fuego y el ^{río} son los dos ejemplos
prototípicos de ^[Heráclito], pero a todo le acontece lo mismo, que
cambia aunque conserve su nombre. Por ejemplo, las personas envejecemos,
como el resto de los seres vivos, y ^{envejecer} es lo mismo que
^{arder}, es oxidarse, aunque sea más lentamente, es decir, que
envejecer es, meramente, un cambio menos perceptible que arder, al menos
para nosotros. ¿Y la ^{piedra}? También cambia si la podemos observar un
tiempo suficiente o lo bastante cerca, pero, aunque no cambiase, no la
veríamos dos veces con la misma luz y desde la misma perspectiva.
Construimos la piedra desde nuestras percepciones. Y, puesto que también
las cosas cambian, ¿puede concluirse, con ^[Heráclito], que nada
permanece?

Para aclarar definitivamente este asunto tenemos que volver a los
comienzos. Aunque el fuego cambia, mientras percibamos signos_{signo} de
fuego, el objeto fuego permanecerá presente. En este sentido, el fuego
se comporta perceptivamente como los otros objetos. Es decir, todos los
objetos permanecen presentes mientras la ^{percepción} detecta signos
suficientes de su presencia_{presente}. Y esto ocurre aunque los
estímulos sensoriales varíen de cada instante al siguiente. Hasta tal
punto es así, que las primeras etapas de la percepción ignoran lo que no
varía porque solamente atienden al cambio; véase
^[Resnikoff]\footnote{_(Resnikoff1989), $\S5.5$.}.

Así que, en la práctica, lo útil es decir de las cosas que cambian o que
permanecen si ello sirve para algo. Por esta razón, aunque solemos
hablar de ^{tierra} firme, sabemos que geológicamente es más preciso
hablar de la deriva de los continentes. ¿Cambia la tierra? No y sí,
depende de cual sea nuestro interés cuando lo decimos. Como dijo
^[Galileo] a propósito de la fijeza aparente de la tierra:
``\latin{Eppur si muove}''.


\Section Lo permanente es lo sintáctico

Pero dejemos las cuestiones prácticas, que a veces recomiendan hablar de
^{cambio} y otras de ^{permanencia}, y ataquemos la cuestión teórica que
debatieron ^[Heráclito] y ^[Parménides] al principio de la ^{filosofía}
griega. Porque hay un modo más radical de entender este asunto, y que
prefiero.

El ^{yo} es el arquetipo de la ^{existencia}---yo soy---porque, para
quienes creemos a ^[Descartes] en este asunto, es lo inmediato. De modo
que al existir de las cosas le aplicamos las cualidades que atribuimos
al existir del yo, y no a la inversa. Como el yo es lo inmediato, es
anterior a todo, incluso al tiempo. Y como el yo es anterior al tiempo,
suponemos que también las cosas existen fuera del tiempo, por sí mismas,
inalterables a la manera de ^[Parménides]. Pero el yo es libre y, por
esto, el yo es sintáctico; véase ^>El pronombre>. Así que proponemos una
generalización, lo permanente es lo sintáctico, cuya pertinencia
mostramos a continuación por su propio interés, y aunque su veracidad o
falsedad no afecta al núcleo de esta teoría.

Lo que no cambia es lo sintáctico. Fuera de la ^{sintaxis} todo es
cambio. Pero sólo desde la permanencia sintáctica se puede observar el
cambio. Otra vez, el simbolismo del sujeto le permite distanciarse del
cambio, en el que está, para observarlo, y esta observación abstracta
del cambio es lo que denominamos ^{tiempo}.

Una oración, como `el ^{perro} juega con una pelota', dice que el perro
está moviéndose, que cambia. Pero, también el perro de la oración `el
perro está quieto' cambia. No es que quien la pronuncia mienta
intencionadamente, al contrario, porque su propósito puede ser hacernos
notar que no molesta y, posiblemente, su movimiento sea imperceptible.
Como ya hemos visto en la sección anterior, _>Todo cambia>, puede ser
prácticamente interesante afirmar que `el perro está quieto'. Pero, ni
siquiera si el perro estuviera muerto, la frase `el perro está quieto'
sería completamente exacta para un físico, a no ser que estuviera a una
^{temperatura} de ^{cero} absoluto que, por otra parte, es inalcanzable.

La imposible quietud del perro contrasta con la permanencia de las
oraciones, como ``todo cambia'' que, contrariamente a lo que afirma, ha
permanecido inalterada desde que ^[Heráclito] la dijo. Además, cuando
una oración se refiere a un asunto sintáctico, como `el verbo de esta
oración es el anterior es', entonces sí que describe una permanencia.

Y, por si estos argumentos en favor de la igualdad entre sintaxis y
permanencia aún no fueran definitivamente convincentes, más adelante, en
_>La jerarquía de Chomsky>, veremos que las expresiones sintácticas de
los simbolismos han de analizarse avanzando y retrocediendo, esto es,
sin restricciones temporales.


\Section La definición

Tras la excursión por el cambio, hemos de retomar el camino, y lo
haremos en el lugar en donde se encuentran otras dos sendas, la de la
^{referencia} y la del ^{problema}, iniciadas, respectivamente, en ^>La
existencia y la referencia>, y ^>El problema>.

Los objetos pueden construirse a partir de cualquier expresión que el
lenguaje simbólico haga posible. Es decir, que merced a la ^{reflexión},
puede primero concebirse un objeto a partir de una oración, y puede
después usarse una palabra para referirse a dicho objeto, con lo que la
palabra última resume la oración original. Diremos que la oración
original es la ^|definición| de la palabra última. Merced a las
definiciones, el ^{lenguaje simbólico} es extensible_{extensibilidad}.

De este modo existen objetos construidos a partir de un ^{problema},
esto es, a partir de una oración interrogativa. Y también, aprovechando
la ^{recursividad} de la ^{sintaxis}, hay objetos construidos por el
expediente de ser las soluciones de un problema. A estos objetos los
denominamos abstractos_{objeto abstracto}. Todos los objetos abstractos
son conceptos, y no cosas, porque su construcción es teórica, y por esta
razón objeto abstracto es sinónimo de ^{concepto abstracto}.

La ^{abstracción} parece algo rebuscada, y lo es, pero, por esa misma
razón, resulta más sorprendente percatarse de que un concepto abstracto
es, simplemente, un objeto determinado por sus propiedades. Esto se
deduce, sin más, de que un problema es libertad y condición, y que la
solución del problema es aquel uso de la libertad que satisface la
condición; como veremos en _>La teoría del problema>. Luego, al decir
que me refiero a las soluciones del problema, quiero decir que me
refiero a todo cuanto cumple la condición del problema.

Así, por ejemplo, al hablar de los astros con luz propia estoy
proponiendo un problema cuya condición consiste en ser un cuerpo
celeste, porque esto es un astro, y en ser emisor de luz, que no
reflector, para referirme a todo cuanto tiene estas dos propiedades,
estar en el cielo y emitir luz. Una vez construido el concepto
abstracto, puedo darle el nombre de `estrella'. Con lo que concluimos:
la definición de ^{estrella} es astro con luz propia.

De manera que cada vez que definimos algo por sus propiedades,
utilizamos un objeto abstracto. ¿Hay algún otro modo de definir? No. Las
condiciones pueden provenir, según nuestro esquema, de la percepción,
del comportamiento, de la emoción y del pensamiento. De aquí que
tengamos cuatro tipos puros de definición: la definición descriptiva,
por las cualidades, cuando todas las condiciones provienen de la
percepción; la definición genética, cómo se hace, que limita los
comportamientos precisos para su obtención; la definición final, para
qué me sirve, si todas las condiciones atañen a su utilidad, y por tanto
se derivan del sistema emocional; y la definición teórica, que establece
condiciones provenientes de otras definiciones, y cuya recursividad es
producto del pensamiento. De modo que, merced a la definición teórica,
también pueden hacerse definiciones compuestas, cruzando los tipos
puros, si las propiedades son de tipos diferentes.


\Section La paradoja

Si el ^{problema} que define un ^{objeto abstracto} no tiene
^{solución}, entonces tenemos un ^{objeto paradójico}. Puesto que los
objetos paradójicos son un tipo de objetos abstractos, todos los objetos
paradójicos son conceptos, y no cosas, siendo ^{concepto paradójico}
sinónimo de objeto paradójico. Los conceptos paradójicos no tienen
referente_{referencia} ni, por consiguiente, ^{significado}. Por
ejemplo, como el problema de encontrar aquellas cosas redondas que sean
cuadradas no tiene solución, resulta que el objeto abstracto `cuadrado
redondo' es una ^{paradoja}.

Los conceptos abstractos son independientes de la ^{percepción}, del
^{comportamiento} y de la ^{emoción}, porque se pueden definir tanto si
tienen referente como si no lo tienen. Así `^{caballo}' puede ser
definido como el animal cuadrúpedo que cumple una serie de condiciones,
y `^{unicornio}' como el animal cuadrúpedo que cumple las condiciones
que definen al caballo y otra adicional, tener un ^{cuerno} en el centro
de su frente. Nótese que, así como el unicornio es, para nosotros,
paradójico, y decimos que no existe, en el caso del caballo, hay una
cosa caballo, que se hace presente cuando vemos un caballo, y un caballo
abstracto, que se hace presente cuando se cumplen las condiciones
enumeradas en su ^{definición}. La coincidencia de cosa y abstracción
suele ser intencional, pero no es siempre exitosa. Consulte
^[Eco]^(Eco1997) si precisa más detalles sobre las peculiaridades y
dificultades de la ^{referencia}, de la ^{definición} y de la
^{abstracción}. Y, de paso, que tengamos por verdadera la ^{oración} `el
unicornio no existe', que no atañe a lo real, prueba que la ^{verdad} es
la conformidad de la ^{expresión sintáctica} con el ^{mundo}, y no sólo
con la ^{realidad}.\strut

 \MT:def tribar(expr alpha) =
 \MT: pickup med_pen;
 \MT: save u, v; u = w/2; v = 12;
 \MT: z1 = (w/2,h/2) + u*(right rotated alpha);
 \MT: z2 = (w/2,h/2) + u*(right rotated (alpha+120));
 \MT: z3 = (w/2,h/2) + u*(right rotated (alpha-120));
 \MT: z1r = (w/2,h/2) + u*(right rotated (alpha+v));
 \MT: z2r = (w/2,h/2) + u*(right rotated (alpha+120+v));
 \MT: z3r = (w/2,h/2) + u*(right rotated (alpha-120+v));
 \MT: draw z1 .. z1r; draw z2 .. z2r; draw z3 .. z3r;
 \MT: draw z1 .. z3r; draw z2 .. z1r; draw z3 .. z2r;
 \MT: z1m = whatever[z1,z2r]; z1m = whatever[z1r,z3];
 \MT: z2m = whatever[z2,z3r]; z2m = whatever[z2r,z1];
 \MT: z3m = whatever[z3,z1r]; z3m = whatever[z3r,z2];
 \MT: draw z1 .. z2m; draw z2 .. z3m; draw z3 .. z1m;
 \MT: z1ra = z1 reflectedabout (z1r,z1m);
 \MT: z2ra = z2 reflectedabout (z2r,z2m);
 \MT: z3ra = z3 reflectedabout (z3r,z3m);
 \MT: draw z1m .. z1ra; draw z2m .. z2ra; draw z3m .. z3ra;
 \MT: z3x - z1x = whatever*(z3-z1m);
 \MT: z2x - z3x = whatever*(z2-z3m);
 \MT: z1x - z2x = whatever*(z1-z2m);
 \MT: z1ra - z1x = whatever*(z3-z1m);
 \MT: z3ra - z3x = whatever*(z2-z3m);
 \MT: z2ra - z2x = whatever*(z1-z2m);
 \MT: z1y = whatever[z1m,z3]; z1y = whatever[z1x,z2x];
 \MT: z2y = whatever[z2m,z1]; z2y = whatever[z2x,z3x];
 \MT: z3y = whatever[z3m,z2]; z3y = whatever[z3x,z1x];
 \MT: draw z1x .. z3y;
 \MT: draw z2x .. z1y;
 \MT: draw z3x .. z2y;
 \MT:enddef;

 \newdimen\tribar \tribar=75pt

 \MTbeginchar(\the\tribar,\the\tribar,0pt);
 \MT: tribar(-36);
 \MTendchar; \setbox0=\box\MTbox
 \MTbeginchar(\the\tribar,\the\tribar,0pt);
 \MT: tribar(-21);
 \MTendchar; \setbox2=\box\MTbox
 \MTbeginchar(\the\tribar,\the\tribar,0pt);
 \MT: tribar(-6);
 \MTendchar; \setbox4=\box\MTbox
 \MTbeginchar(\the\tribar,\the\tribar,0pt);
 \MT: tribar(9);
 \MTendchar; \setbox6=\box\MTbox
 \MTbeginchar(\the\tribar,\the\tribar,0pt);
 \MT: tribar(24);
 \MTendchar; \setbox8=\box\MTbox

 \dimen0=\hsize \advance\dimen0-\tribar

 \nointerlineskip\hbox to0pt{\kern\dimen0\vbox to0pt{\kern-9pt
 \box0\kern-10pt\box2\kern-9pt\box4\kern-9pt\box6\kern-12pt\box8\vss}\hss}%
 \nointerlineskip

\strut\parshape1 0pt \dimen0
Incluso algunas ilusiones_{ilusión} ópticas son paradojas, por ejemplo,
el tribar de ^[Penrose]. La representación plana del tribar es
perfectamente posible, como demuestran las figuras; que han sido
adaptadas de ^[Resnikoff]^(Resnikoff1989). Pero es paradójico el
concepto tribar definido como la solución del problema que consiste en
construir aquella cosa tridimensional cuyas representaciones
bidimensionales son las figuras planas del tribar.

\strut\parshape1 0pt \dimen0
La paradoja aparece, por consiguiente, cuando un ^{concepto} queda
encerrado en el ^{bucle teórico} sin ninguna posibilidad de alcanzar el
^{bucle práctico}. Por esta razón, los sistemas teóricos, con sus
definiciones teóricas, corren el riesgo de ser paradójicos. Piénsese,
por ejemplo, en el ^{flogisto} y en el ^{electrón}. De estos dos objetos
abstractos, el flogisto, véase ^[Kuhn]\footnote{_(Kuhn1970),
pá\-gi\-nas~99--100.}, al quedar invalidada la teoría ^{física} en la
que se sustentaba y que lo consideraba la cosa que explicaba los
fenómenos térmicos, es considerado inexistente, mientras que el
electrón, aunque se haya descubierto que puede comportarse como una
partícula y como una onda, es, hoy, tenido por existente. Según nuestra
jerga, el flogisto es paradójico porque no se tiene por causante de
efecto práctico alguno, mientras que el electrón no es paradójico, por
más que su comportamiento sea ininteligible, porque se considera que es
la explicación de ciertos fenómenos físicos, principalmente eléctricos.

Por otra parte, en un lenguaje simbólico es imposible eludir las
paradojas, porque merced a la abstracción puedo referirme a lo
irreferible, acabo de hacerlo, y definir lo indefinible, que es,
simplemente y por definición, aquello que no puede ser definido. La
imposibilidad de eliminar las paradojas es precisamente una de las
características de los lenguajes simbólicos, como quedará probado en
_>La paradoja reflexiva>.


\Section La herramienta

Al aparato simbólico del ^{sujeto} se le presenta ya dado el objetivo a
cumplir, como un sentimiento, por ejemplo de ^{sed}, y las condiciones
externas de satisfacción en las que se encuentra, a la manera de la
^{realidad} ^{presente}, en la que buscará signos de ^{agua}, de ríos o
manantiales. Es decir, se le presenta un ^{problema}, como ya se le
presentaba al ^{conocedor}. Lo novedoso del sujeto es que trata al
problema como objeto, como si fuera real, aunque no tenga significado. Y
es por este motivo, y porque también puede tratar a la resolución como
un objeto, por lo que puede abstraer, ^{razonar} e, incluso, construir
la resolución del problema.

Esto, que puede parecer muy teórico y poco práctico, explica por qué el
sujeto es capaz de diseñar herramientas, o vestirse, y los conocedores
simples no. Una ^{herramienta} es una resolución hecha cosa, de modo
que, antes de fabricarla, es necesario imaginarla, esto es, hay que
representársela internamente, y solamente un sujeto es capaz de
imaginarse, en su ^{lógica simbólica}, una resolución. Utilizo el
vocablo `representar', aunque me molesta su etimología objetivista.


\Section El sujeto

Al llegar al ^{sujeto} se desborda lo que puede ser dicho, así que es
mejor detenerse ahora haciendo, para terminar, un resumen de lo que más
nos importa.

Los sujetos usan, por razones de su historia evolutiva, dos tipos de
representaciones u objetos: las cosas y los conceptos. Las cosas_{cosa}
son los objetos antiguos construidos, como ya en el caso de los
conocedores, por la percepción, el aprendizaje y la emoción. Los
conceptos_{concepto} son los objetos nuevos que el pensamiento elabora
voluntariamente a partir de otros objetos, que pueden ser tanto cosas
como conceptos. Por esto el ^{simbolismo}, que es la lógica, o sistema
de representación, del sujeto, tiene dos capas: la ^{semántica}, o
lógica antigua, con las cosas reales_{realidad}, y la ^{sintaxis}, o
lógica nueva, con los conceptos teóricos_{teoría}.

La novedad del simbolismo es, pues, la nueva capa sintáctica que tiene
su origen en la ^{palabra}, que al interiorizarse es la ^{idea}, que al
hacerse ^{objeto} es el concepto. Nos interesa especialmente que la
^{sintaxis} hace posible la reflexión consciente y la representación
problemas; recordemos cómo.

Si ^{ver} el exterior consiste en reconocer objetos en los fenómenos
exteriores, entonces, cuando el sujeto reconoce objetos en los objetos
interiores, ve su propio interior. Por lo tanto, es la naturaleza
recursiva_{recursividad} del concepto la que permite a la cognición la
visión de la propia cognición, que por esto se puede denominar visión
reflejada o ^{reflexión}.

Los conceptos pueden referirse directa o indirectamente a cosas, de las
que toman sus significados, o pueden no tener ^{significado}. Estos
conceptos libres de significado, o sea, puramente sintácticos, son los
que permiten la representación de los problemas, porque pueden expresar
la ^{libertad}, o indeterminación, que todo ^{problema} plantea.


\Section El mundo del sujeto es simbólico

Dos conclusiones sobre el mundo del sujeto:
\beginpoints
\point El ^{mundo} del sujeto es simbólico, o sea, reflexivo, discursivo y
lingüístico, e incluye a la ^{realidad} semántica.
\point El sujeto, al considerar los problemas con sus soluciones y sus
resoluciones, evalúa distintos mundos posibles. El sujeto habita un
mundo de posibilidades. El sujeto es libre.
\endpoints


\endinput
