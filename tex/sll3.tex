% sll3.tex (RMCG20000320)

\Part Interludio

\Section El objeto concebido

Haremos aquí, antes de tomar el camino de vuelta, una pausa para
conciliar todo lo visto al entrar. Porque para preparar el viaje, antes
de tomar el camino de entrada, nos entretuvimos con unos prolegómenos en
los que se simplificaron en exceso algunos puntos. Y, aunque las dos
afirmaciones de los prolegómenos, la ^{realidad} objetiva es subjetiva y
la realidad es involuntaria, han quedado refrendadas en el camino de
entrada, ahora sabemos que, contra lo que afirmábamos entonces, hay
objetos que son posteriores a la ^{percepción}.

Sucede que, aquello que denominábamos ^{objeto} en los prolegómenos,
resulta ser, al verlo con más detalle, una ^{cosa}. Una cosa es un
objeto práctico, y por lo tanto ajeno a la ^{voluntad}. Pero hay otros
objetos, los conceptos_{concepto}, que son teóricos y voluntarios. Así
que no es correcto afirmar que todos los objetos son involuntarios y
anteriores a la palabra, ya que los hay voluntarios y creados con
palabras o, más exactamente, con ideas_{idea}.

Hemos de rectificar, pues, lo dicho en ^>La realidad es involuntaria>.
Resulta que solamente es involuntaria la realidad de las cosas, realidad
construida, según vimos en ^>La cosa y el concepto>, por la percepción,
el aprendizaje y la emoción, pero sin la intervención del pensamiento, y
que coincide con la realidad del sujeto, pero no con su mundo, que
también incluye conceptos teóricos voluntariamente generados.

Dado que en el ^{mundo} del sujeto hay objetos voluntarios e
involuntarios, una primera explicación puede establecer que los objetos
involuntarios son autónomos e independientes del sujeto, mientras que
los voluntarios no tienen existencia fuera del pensamiento del sujeto.
De este modo fundó el ^{objetivismo} la diferencia esencial entre el
mundo real de las cosas y el mundo teórico de los conceptos. Pero, como
hemos visto, la diferencia no es esencial, sino meramente circunstancial
o genética, ya que tiene su origen en la peculiar evolución del sistema
nervioso que hizo del hombre un sujeto.


\Section La contingencia

Aprovechemos la pausa para hacer otra salvedad. Hemos descrito el
camino de entrada como si cada paso fuese una consecuencia ineludible
del anterior, pero no es así. No debe pensarse, por ejemplo, que el
aprendiz es necesariamente seguido por el conocedor. Lo único que se
pide a un nuevo paso es que mejore a los ya dados en algún ^{nicho},
esto es, dadas ciertas condiciones que se cumplen en algún momento y
lugar. Si esto es así, es posible, pero no seguro, que la ^{evolución}
saque partido de la mejora en ese nicho. Por otra parte, de todas las
posibles secuencias evolutivas de la ^{cognición}, hemos intentado
describir la que lleva al \latin{homo sapiens}.

El resto de la pausa lo utilizaremos para reflexionar sobre la
explicación, deteniéndonos, sobre todo, en las dificultades que presenta
la ^{explicación de entrada} para, de ese modo, preparar el camino de
salida.


\Section ¡Abajo el materialismo!

Si una ^{explicación} lo es, no puede recurrir a ningún acto de ^{fe}.
Si un paso de la explicación, por pequeño que sea, precisa de la fe para
salvarlo, entonces es que no está explicado. Si un detalle de una
explicación, aunque sea minúsculo, es inexplicable, entonces la
explicación no es completa, y una explicación parcial no es una
explicación, sino un replanteamiento más preciso del ^{problema}. Todas
estas obviedades son, me parece, las que han llevado al ^{materialismo}
a una posición preponderante en las ciencias_{ciencia}.

Algunos científicos no son materialistas sólo porque la actual
especialización les permite pensar que, aunque en su propio campo de
investigación todo puede ser explicado, en otros campos hay fenómenos
imposibles de entender, ya sea porque ^{Dios} es inefable o porque la
persona es libre. Pero en las ciencias más básicas ni siquiera esto es
posible. Es así como, por ejemplo, ^[Hawking]^(Hawking1988) concluye
negando a Dios cualquier posibilidad de elegir y ^[Minsky]^(Minsky1985)
termina negando la ^{libertad} porque, alega, todo es ^{causa} y
^{azar}, como ya dijera ^[Monod]. En lo que sigue intentaré mostrar que,
a pesar de sus buenas intenciones, el materialismo no puede ser
correcto.

^[Descartes]^(Descartes1641), que recurrió a los principios, a lo claro
y distinto, planteó la cuestión correcta. ``Yo pienso, y por lo tanto
existo''. Es decir, lo primero es la palabra, la realidad viene después.
^[Descartes], además de plantear la cuestión correcta, la contestó
correctamente. Las cosas materiales, que existen realmente, pueden ser
descritas como máquinas_{máquina}, como relojes_{reloj} mecánicos, pero
precisamente el yo que piensa y que habla libremente no puede ser
asimilado a una máquina, como destacó ^[Chomsky]^(Chomsky1966).

Sobre los hombros de ^[Descartes], ^[Newton] pudo descubrir que el
universo era un enorme reloj de precisión. Pero aunque no quiso hacer
hipótesis, su reloj no era mecánico, al menos según las prescripciones
cartesianas. La ^{acción a distancia} sumió a la ^{materia} en grandes
dificultades ontológicas, y desde entonces es arduo definir qué es la
materia. A pesar de estos inconvenientes, la física newtoniana, capaz de
espectaculares predicciones_{predicción}, se impuso de tal manera que
finalmente se olvidaron las reticencias y en las ciencias se impuso
definitivamente el materialismo.

Pero, ¿qué es el materialismo? Quizás su eslogan más famoso es el ya
apuntado y que retoma un dicho del viejo ^[Demócrito]^(Monod1970):
``todo es azar y ^{necesidad}''. Debe observarse, sin embargo, que
ningún materialista lo admitía antes del advenimiento de la ^{mecánica
cuántica}, y que incluso ^[Einstein]^(Einstein1936), uno de sus
precursores, negó siempre el azar. Quiere esto decir que, en definitiva,
lo que los materialistas sostienen es que la explicación última es la
que proporciona la ciencia ^{física}. Así se puede entender que lo
material vaya cambiando a medida que la física evoluciona, incluyendo
primero la acción por contacto, luego los campos de energía y más tarde
el azar, por poner tres ejemplos. De manera que otro eslogan como `todo
es físico' o `el mundo es físico' puede ser más adecuado para el
materialismo.

El materialismo ordena por importancia las explicaciones y, a la vez,
las ciencias que las producen. La física, conforme al postulado
materialista, proporciona la explicación definitiva, por lo que es el
más fundamental, o más básico, o más importante (los ingleses dicen más
duro) de los saberes. A continuación aparece la ^{química}, y después
viene la ^{biología}. Por último, y únicamente cuando pueden reducirse a
la biología, a la química, y a la física, coloca el orden materialista a
la ^{psicología} y las ^{humanidades}. Como para el materialismo todo es
finalmente reducible a física, el ^{libre albedrío}, la ^{consciencia} y
el ^{yo} resultan ser meras ilusiones o maneras de hablar y, en
cualquier caso, sin influencia alguna sobre la realidad.

Todo esto es absurdo e insostenible. Porque las explicaciones, como la
propia explicación materialista, no son más que expresiones_{expresión
sintáctica} simbólicas. Y las expresiones simbólicas sólo tienen
significado para los sujetos_{sujeto} simbólicos, o sea, para los yoes.
Por esto, sólo los sujetos simbólicos, es decir, los yoes, están
interesados en producirlas. Más gráficamente. Imagine que desaparece la
humanidad de la tierra y que, por consiguiente, no queda yo alguno. No
habiendo quien interpretara las palabras, y dado que la relación entre
la secuencia de letras y el significado de la palabra es fruto de
convenciones establecidas entre sujetos, ¿cree que tendría algún efecto
que este papel tintado que está leyendo dijera lo que dice o significara
cualquier otra cosa? Luego el materialismo proporciona una explicación
que, si fuera coherente consigo mismo, no tendría significado alguno. El
materialismo es ^{absurdo}.

Quizás sospeche usted que la única alternativa al materialismo, que
viene a ser a la ^{ontología} lo que el monoteísmo es a la religión, es
el ^{dualismo}. No lo es. No lo es si se abandona el objetivismo y se
adopta el ^{subjetivismo}. El truco, que tampoco lo es, consiste en
percatarse de que, ya que los objetos son imágenes o representaciones
que no existen fuera de nuestra cabeza, tampoco es relevante que estén
formados con un único tipo de ^{sustancia} o con dos. Dicho de otra
manera: la ^{epistemología} es anterior a la ontología.

El ^{mundo} es simbólico, es decir, se compone de dos capas: la
^{sintaxis} en donde se encuentra el pensamiento racional con los
conceptos, y la ^{semántica}, que equiparamos a lo físico, con las cosas
reales. Esto lo alcanzó ya ^[Descartes], que en este punto no tuvo más
remedio que postular una \latin{^{res cogitans}} en oposición a la
\latin{^{res extensa}} para resolver, que no solucionar, el problema de
definir el mundo. Nosotros tenemos más suerte. ^[Turing] nos equipó con
herramientas que nos permiten solucionarlo. ^[Turing]^(Turing1936)
demostró en 1936 que puede construirse físicamente un simbolismo, o más
exactamente, un ^{motor sintáctico}. Y cada ^{computadora} es una prueba
palpable de la verdad de su demostración teórica.

\breakif1

Las dos proposiciones, `es posible construir físicamente un motor
sintáctico' y `la sintaxis puede tener efectos físicos reales', o sea,
puede tener significado, son lógicamente equivalentes. Ambas
proposiciones son equivalentes y sin embargo, dado que lo indudable es
el yo que está en la capa sintáctica, que es la capa del pensamiento
simbólico y el habla, la segunda es preferible.

Esta concepción del mundo rebate el postulado materialista, porque
muestra que no es cierto que todo sea físico. Por ejemplo, el yo no es
físico, es sintáctico, y es parte del mundo pudiendo incluso tener
efectos físicos reales, como acabamos de ver.

Estos razonamientos nos muestran, de paso, cómo la ^{filosofía} sigue
siempre a la ^{ingeniería}. Mientras ^[Descartes] contaba con el reloj
mecánico para imaginarse el mundo, nosotros contamos con la computadora,
y por esa razón podemos entenderlo de otro modo. Un modo que podría
denominarse simbólico o lingüístico.


\Section El mecanismo

El camino de entrada, por empezar con los objetos ya establecidos,
omitió el primer paso: el mecanismo.
$$\hbox{Mecanismo} \supset \hbox{Adaptador} \supset
  \hbox{Aprendiz} \supset \hbox{Conocedor} \supset \hbox{Sujeto}
\abovedisplayskip=9pt
\belowdisplayskip=9pt
$$

Un ^|mecanismo| es cualquier cosa que interactúa con su ^{entorno}, es
decir, su única característica es que tiene un ^{comportamiento}. Todos
los seres vivos, plantas_{planta} incluidas, son mecanismos.
$$\hbox{\strut Fenómeno}
  \underbrace{\strut \longrightarrow}_{\hidewidth
   \hbox{\strut Mecanismo}\hidewidth}
  \hbox{\strut Acción}
\abovedisplayskip=9pt plus 3pt
\belowdisplayskip=9pt plus 3pt
$$
También el reloj mecánico es un mecanismo. La ^{computadora}, ya lo
vimos en ^>La realidad>, es capaz de un comportamiento distinto por cada
^{programa} que es capaz de ejecutar y, por lo tanto, es capaz de imitar
a distintos mecanismos, por ejemplo al reloj, siendo la propia
computadora otro mecanismo.

A los otros jalones de la evolución epistemológica ya los hemos
presentado. Un ^{adaptador} es un mecanismo con dos partes, cuerpo y
sistema nervioso, siendo el sistema nervioso el que selecciona, en
función de la realidad objetiva presente, el comportamiento que el
cuerpo ejecuta. Un ^{aprendiz} es un adaptador capaz de sintonizar la
realidad a su entorno. Un ^{conocedor} es un aprendiz que puede utilizar
la realidad de varios modos que selecciona merced a una percepción
interior denominada emoción. Por fin, un ^{sujeto} es un conocedor que
dispone de lenguaje simbólico, lo que le permite ampliar la realidad,
merced al aprendizaje, más allá de la percepción y de la emoción, con
conceptos teóricos ideales.

En conclusión, un sujeto es un mecanismo con un sistema nervioso en el
que modela la realidad, realidad que puede utilizar de varios modos, y
que dispone de lenguaje simbólico. O sea, que un sujeto es un mecanismo
con una serie de características que lo distinguen de otros mecanismos
que no son sujetos.


\Section ¿Es libre el sujeto?

Si explicamos de fuera a dentro, el ^{sujeto} es un mecanismo y los
mecanismos no son libres. La ^{libertad} aparece en el sujeto de la
nada, como por magia, y esto no es aceptable en una explicación. La
secuencia discurre por las siguientes etapas.

¿Es libre un mecanismo? No, de ningún modo. El ^{mecanismo} es el
prototipo del determinismo. Para el mecanismo todo es ^{azar} y
^{necesidad}. ¿Tiene libertad un adaptador? No. Un ^{adaptador} no es
más que un mecanismo en que se han diferenciado dos partes: el cuerpo
que ejecuta los comportamientos y el sistema nervioso que selecciona el
comportamiento a ejecutar, y ambos son mecanismos. ¿Y el aprendiz?
Tampoco, porque el ^{aprendiz} sólo es un adaptador con un sistema
nervioso que se sintoniza mecánicamente a su entorno. Y el conocedor,
¿tiene libertad? No, ya que un ^{conocedor} no es otra cosa que un
aprendiz capaz de sentir las necesidades internas de su propio cuerpo
que, repito, es un mecanismo. Y el sujeto, ¿es libre?

¿Es libre el sujeto? Si nos fijamos en el ^{camino de entrada}, entonces
tampoco el sujeto deja de ser un mecanismo. Puede decirse que la
libertad aparece con el sujeto, pero entonces debería suponerse que la
libertad está, de algún modo, latente desde el mecanismo más simple que,
paradójicamente, es el prototipo del determinismo. Por otro lado, si en
vez de atacar esta cuestión de fuera a dentro, la encaramos en el otro
sentido, de dentro a fuera, la respuesta es la contraria. El sujeto es
libre por su propia naturaleza. El sujeto se ve como su ^{yo}, o sea,
libre para hacer según su ^{voluntad}.

Sucede, pues, que mirada la cuestión desde fuera se tiene una impresión
por completo diferente que si la miramos desde dentro. Esta situación no
es cómoda, de modo que, en lo que sigue, intentaremos reconciliar ambos
puntos de vista.


\Section El extraño

El camino de entrada, del mecanismo al sujeto, sigue a la ^{evolución}.
Ese sentido, de lo simple a lo complejo, puede que sea más explicativo,
pero no es el camino hecho, sino justamente el contrario al recorrido.
Porque lo primero es la ^{pregunta}. Sólo porque primero podemos
preguntar, podemos después responder. Hay ^{explicación} porque hay
cuestión, aunque, para poder preguntar y plantear problemas, sea precisa
mucha complicación. Tanta que sólo un sujeto puede preguntar.

Luego, aunque en la explicación de entrada el ^{lenguaje simbólico} es
lo último que aparece, resulta que es la condición indispensable, no ya
para empezar a explicar, sino incluso para comenzar a preguntar. Porque
tanto la pregunta como la propia explicación son expresiones del
lenguaje simbólico.

No me resisto a considerar dos posibilidades curiosas. Una posibilidad
es que no existieran los sujetos ni, en consecuencia, los lenguajes
simbólicos. En este caso no habría explicación de por qué no existía el
simbolismo ni de ningún otro asunto, ya que no habría ni explicaciones
ni preguntas. No habría explicación, ni nadie que la demandara, así que
tampoco habría tensión alguna.

Otra posibilidad es que haya efectivamente sujetos, y que sus
explicaciones del ^{mundo} consigan explicar todo cuanto ocurre, excepto
el propio lenguaje simbólico. Si éste fuera el caso, el ^{sujeto} se
vería a sí mismo excluido del mundo, como si no perteneciera a él. El
sujeto, viéndose ajeno a cuanto le rodeara, se sentiría ^{extraño}. En
resumen, si el sujeto no fuera capaz de explicar su propia naturaleza
simbólica, entonces se sentiría extraño al mundo y perplejo. Urge
explicar el lenguaje simbólico.


\Section La explicación material

En principio, una ^{explicación} puede ser llevada tan lejos como se
quiera. Basta preguntar el porqué de cada explicación, como suelen hacer
los niños cuando lo descubren. Pero esto es muy insatisfactorio, ya que
necesariamente se termina en un círculo vicioso de explicaciones, o se
llega a un punto en el que se reconoce que no hay una explicación
adecuada. Para evitarlo hay que convenir qué necesita de explicación, y
qué no la necesita. Esto es tan básico, que tal convenio suele ser
tácito.

\breakif2

Por ejemplo, la solución objetivista, que es la más extendida y natural,
establece que las cosas_{cosa} no tienen que ser explicadas, simplemente
son, y esto es suficiente, de manera que sólo los conceptos_{concepto}
han de ser explicados. Y explicar los conceptos consiste, para el
objetivismo, en hacer que los conceptos se adecúen a las cosas. Es
decir, el ^{objetivismo} subordina el ^{bucle teórico} al
práctico_{bucle práctico}. Esto es razonable, por los siguientes
argumentos. Las cosas son los objetos del sujeto que el conocedor simple
ya empleaba. El conocedor simple del que desciende el sujeto era viable,
como lo prueba la mera existencia de su descendiente. Así que es
prudente construir los nuevos conceptos sobre la sólida base de las
cosas. Y esto es, precisamente, lo que propone la ^{explicación
material} del objetivismo.

Como resultado de este análisis vemos que, para el objetivismo, la
explicación de un concepto consiste en su ^{cosificación}. Así, por
ejemplo, para explicar los fenómenos eléctricos se utiliza la cosa
llamada ^{electrón}, que ya no requiere explicaciones ulteriores aunque
su contradictoria naturaleza dual como onda y partícula sugiera lo
contrario. Cuando la física cuántica descubrió que todas las cosas
tienen una naturaleza dual, puso de manifiesto que el objetivismo tiene
limitaciones, aunque sea suficiente en la práctica, suficiencia avalada
por la existencia de los conocedores simples. Las limitaciones del
objetivismo resultan de subordinar las explicaciones a la ^{percepción},
al ^{aprendizaje} y a la ^{emoción} propios del ^{sujeto} humano.


\Section La explicación automática

Cuando la ^{física} se encontró con las llamadas paradojas cuánticas
resolvió superar la explicación material con la explicación automática.
La denominamos así, no porque obtenga automáticamente las explicaciones,
sino porque propone como ^{explicación} cualquier sistema de ecuaciones
que permita predecir_{predicción} mecánicamente el futuro del fenómeno
explicado, y estos sistemas se pueden modelar matemáticamente como
autómatas finitos. Así que, en el caso ideal, la ^{explicación
automática} proporciona un autómata cuya apariencia es indistinguible
del ^{fenómeno} explicado, siendo el ^{autómata} un ^{mecanismo}, pero
del que se desechan sus propiedades físicas y únicamente queda su
capacidad de tratar ^{datos}. De modo que, por definición, un autómata
es un mecanismo abstracto.

Para la explicación automática de la física actual vale cualquier
automatismo que permita predecir lo que sucedería en cada
caso,\vadjust{\null\vfil\break} aunque no se corresponda con ^{cosa}
alguna. En concreto, la explicación cuántica es un sistema de ecuaciones
que, particularizadas, consiguen predecir con una exactitud sin
precedentes el resultado de los experimentos; véase
^[Feynman]^(Feynman1985). La explicación última no es ya el ^{electrón},
sino las ecuaciones físicas. Para la explicación automática, el electrón
es una consecuencia del sistema de ecuaciones, y no al revés, como
ocurre en la explicación material, para la que el sistema de ecuaciones
es el resultado de describir el comportamiento del electrón.

La explicación automática mejora a la ^{explicación material} ya que no
da preferencia a las cosas sobre los conceptos, preferencia que no tiene
más razón que la contingente historia evolutiva del \latin{homo
sapiens}. Por contra, al conceder la primacía al automatismo previsor,
la explicación automática abandona el significado que sí estaba
naturalmente en las cosas de la explicación material. Recordemos que las
cosas reales siempre tienen un ^{significado} natural, pero que los
conceptos pueden no tener significado; según vimos en ^>La existencia y
la referencia>. Sucede, así, que la explicación automática que
proporciona la física cuántica es capaz de predecir con precisión y
exactitud, pero no significa nada. Para un objetivista esto es lo mismo
que no explicar, es preferir la descripción a la explicación. La
explicación automática no explica, sino describe.


\Section La explicación de entrada

La discusión entre ^[Einstein] y ^[Bohr]^(Murdoch1987) ha de entenderse
en el contexto de esta transición desde la postura ontológica propia de
la explicación material defendida por el primero a la posición
pragmática que propugna la explicación automática del segundo.

Desde otro punto de vista, la ^{explicación material} se completa con la
creencia en un ^{Dios} creador de todas las cosas y legislador del
universo, mientras que la ^{explicación automática} solamente precisa
del respaldo del legislador universal. Si en la explicación material las
leyes gobiernan todo cuanto ocurre, en la explicación automática rigen
aún más, porque en ella las leyes especifican todo cuanto acaece y todo
cuanto es. En ninguna de las dos cabe la ^{libertad}, ya que todo cuanto
acontece está gobernado por las ^{leyes universales} de la naturaleza.

Tanto las explicaciones materiales como las explicaciones automáticas
son explicaciones de entrada_{explicación de entrada}, porque ambas se
construyen con recursos externos al ^{yo}, como lo son las leyes y las
cosas. Y, puesto que en ninguna de las dos cabe la libertad, resulta que
ninguna alcanza a explicar el ^{sujeto}. Y, por lo tanto, ni la
explicación material, ni la explicación automática, explican el
^{lenguaje simbólico}, que es propio de los sujetos.  En estas tristes
circunstancias el simbolismo queda sin explicación y el sujeto
extrañado_{extraño}.


\Section Una mala jugada

Para superar este obstáculo, la ^{teoría de la subjetividad} propone
volver a comenzar desde el mismísimo principio. Esto es tanto como
volver al ``yo pienso'' de ^[Descartes] y dar por perdido todo el camino
hecho. La situación exige coraje y resolución, y por este motivo no he
anunciado antes el grave estado en el que nos encontramos. A estas
alturas, ya está usted muy lejos de la seguridad de su hogar, y no tiene
otro remedio que hacer el camino de vuelta, so pena de quedar
desamparado en su ^{perplejidad}. Lo siento, pero a veces hay que jugar
sucio.


\endinput
