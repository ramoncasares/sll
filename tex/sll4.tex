% sll4.tex (RMCG20000611)

\Part Salida

\Section ¿Qué soy yo?

El ^{camino de salida} va de dentro a fuera. Empieza en el ^{yo}, pero
¿qué soy yo?

Para ^[Descartes], el yo es lo cierto, lo indubitable, lo único que se
conoce con ^{certeza} absoluta, y que, por lo tanto, ha de servir para
conocer todo lo demás. Es decir, que yo soy, por definición, quien
define. Esta es, por supuesto, una definición circular que descubre,
simplemente, que el yo de ^[Descartes] es atómico, quiero decir, no
analizable.

Nuestro camino de salida parte del yo, con ^[Descartes], por lo que bien
podríamos consentir que el yo fuera un término primitivo no analizable.
Y, sin embargo, hay algunas cualidades que pueden predicarse del yo, y
otras que no. Por ejemplo, `yo soy libre' es perfectamente válido,
mientras que aplicar al yo cualidades físicas es más controvertido. En
principio se puede decir, y se dice, que `yo ^{peso} ochenta kilos',
pero, bien mirado, lo que pesa es mi ^{cuerpo}, no mi yo, lo que queda
probado cuando adelgazo diez kilogramos, porque sigo siendo el mismo yo,
aunque pese setenta kilos. Y, por supuesto, que yo pese ochenta
kilogramos no es un conocimiento que yo pueda alcanzar por simple
^{introspección}, así que no es un conocimiento indudable, así que no es
parte del yo.

Ya sabemos que este procedimiento de eliminación de las propiedades
físicas del yo concluye negando que el yo sea físico. Y la consecuencia
es que el yo no es explicable científicamente. Esta limitación de la
^{ciencia} actual es enojosa, y hay quien piensa, como
^[Minsky]^(Minsky1985), que el yo es una ^{ilusión} y que ^[Descartes]
había de estar necesariamente equivocado. Pero los argumentos de
^[Descartes] son, en este asunto, inatacables, hasta tal punto que ni
siquiera su descalificación como ilusorios los desacredita. Porque
aunque el yo fuera meramente ilusorio, la ciencia quedaría incompleta si
no explicara la naturaleza de tan repetida ilusión. Por mi parte, creo
que para superar la dificultad del yo es preciso ampliar el poder
explicativo de la ciencia, en vez de negar el hecho. Sigamos, pues.

Lo definitivo, en este camino de salida, es lo que introspectivamente
conocemos del yo. Introspectivamente, y así lo sanciona el
^<Diccionario> de la ^[Real Academia Española]^(RAE1970), el yo es la
``afirmación de conciencia_{consciencia} de la personalidad humana como
ser racional_{razón} y libre_{libertad}''. Para establecer el campo
semántico del yo, consultamos el ^<Diccionario ideológico> de
^[Casares]^(Casares1959), que toma la definición de la ^[Academia], y
remite al grupo analógico de la {\em conciencia}. En este grupo aparecen
seis secciones. Es en la primera en la que se encuentra {\em yo}, {\em
^{sujeto}} y {\em ^{persona}}, la segunda es encabezada por {\em
moralidad_{ética}}, la tercera incluye {\em percibir_{percepción}} y
{\em reflexionar_{reflexión}}, la cuarta {\em enconar}, la quinta {\em
voluntario_{voluntad}} y la última {\em interiormente}.

Toda teoría que explique el yo deberá construir un todo coherente que
asimile, de uno u otro modo, este amasijo de conceptos ciertamente
relacionados. Y en eso estamos.


\Section Yo soy libertad para no morir

El ^{sujeto} no se ve a sí mismo como sujeto, sino como su ^{yo}, es
decir, libre. Yo soy libre de decidir qué hacer.

Lo que se hace en cada momento es lo hecho, y en lo hecho no hay
libertad alguna. Por esto, la ^{libertad} no está en lo hecho, sino en
poder considerar distintas posibilidades de hacer, tantas cuantas la
imaginación del sujeto produzca. Un ^{esclavo} puede, así, ser tan libre
como su ^{amo}, aunque tenga que desechar inmediatamente muchas opciones
que su amo sí ha de evaluar. Lo grave es que, por una cuestión de mera
eficiencia mental, el esclavo terminará por no plantear, ¿para qué?, las
posibilidades socialmente imposibilitadas.

La libertad del sujeto no es completa porque está limitada, en última
instancia, por su propia ^{muerte}. Dicho más llanamente, los muertos no
toman decisiones, no son libres de actuar, por lo que la libertad de los
vivos sólo dura mientras lo son. Podemos decir, según estos
razonamientos, que el yo del sujeto es libre con la condición de que no
se muera. Y, con esta noticia, alcanzamos ya una definición concisa,
pero suficiente, del ^|yo|: yo soy libertad para no morir.

Pero sólo el desarrollo completo de la definición mostrará si es
suficiente y acertada, o si no lo es. Comenzaremos a investigar esta
cuestión afianzando la definición con un ejemplo, para luego proponer
una definición equivalente del yo.


\Section El sistema penal

Yo_{yo} soy ^{libertad} para no morir. El ^{sistema penal} hace un uso
{\em ajustado} de esta definición. El máximo ^{castigo} es la pena de
^{muerte}, y el siguiente la ^{cadena perpetua}, que aunque no elimina
la libertad completamente, sí que la mengua para siempre. Y los castigos
menores son penas de ^{cárcel} que privan temporal y parcialmente al
^{reo} de libertad.


\Section El problema del sujeto

En cuanto el ^{yo} es consciente de ser libre y mortal, es consciente de
un ^{problema}, ¿qué hacer para no morir_{muerte}?, que denominaremos el
^|problema del sujeto|.

La definición del yo propuesta, yo soy ^{libertad} para no morir, y el
problema del sujeto, ¿qué hacer para no morir?, son equivalentes.
Porque, en un sentido, si el problema está correctamente planteado es
porque tengo libertad para hacer y, en el otro sentido, si hay libertad
y condición, entonces se plantea inmediatamente qué hacer con la
libertad para cumplir la condición de no morir. $$\hbox{Yo} =
\hbox{Problema del sujeto}$$

En resumen, que si el yo es, por definición, libertad para no morir,
entonces el yo es, también, el problema del sujeto, ¿qué hacer para no
morir?, y, por consiguiente, la investigación del yo ha de proseguir con
el estudio de la ^{teoría del problema}, como nos aconseja sabiamente
^[Dewey]^(Dewey1941).
$$\hbox{Yo} \etapa \hbox{Problema}$$


\Section La teoría del problema

Todo ^|problema| es un compuesto de ^{libertad} y de ^{condición} (véase
{\sc epa}^(Casares1999)~$\S4$). Ha de haber posibilidades y libertad de
elegirlas, ya que si sólo hay ^{necesidad} y ^{fatalidad}, entonces no
hay problema ni decisión que tomar. Las distintas opciones posibles
podrán valer, o no, como solución del problema, de modo que en todo
problema existe cierta condición que determina si la opción vale, o no,
como solución del problema.
$$
\centerline{$\hbox{\rm Problema}
 \left\lbrace\vcenter{\hbox{Libertad}\hbox{Condición}}\right.$}
$$

Además, debería haber cierta ^{información} que ayudara a tomar la
decisión porque, de no haberla, el problema habría de ser resuelto al
^{azar}. Y sin embargo, un tipo de problema, que denominaremos aparente,
no aporta información alguna. Más adelante volveremos al problema
aparente porque se encuentra justo en el meollo de la cuestión, pero de
momento sólo nos fijaremos en que la información marca la diferencia
entre los dos tipos de problema: el ^{problema aparente}, sin
información, y el no aparente, con ella. Por ejemplo, el ^{problema del
sujeto} no es un problema aparente, ya que el ^{sujeto} dispone de una
ingente cantidad de información, consciente e inconsciente, sobre lo que
favorece la ^{vida} y atrasa la ^{muerte}. Ya veremos más adelante, en
_>El problema aparente>, _>El universo>, _>El conocimiento es
provisional>, _>El problema del aprendiz>, _>El conocimiento sintético a
priori> y _>La cultura>, de donde proviene toda esta información.

Es fundamental distinguir la solución de la resolución del problema.
Resolver es a buscar como solucionar a encontrar, y nótese que se puede
buscar lo que no existe. Así, la ^|resolución| es el proceso que intenta
alcanzar la solución del problema, mientras que la ^|solución| del
problema es un uso de la libertad que satisface la condición.
$$ \hbox{Problema} \longrightarrow
 \hbox{Resolución} \longrightarrow \hbox{Solución}$$

Se puede explicar con otra analogía. El problema queda definido por la
^{tensión} que existe entre dos opuestos: la libertad, ajena a cualquier
límite, y la condición, que es puro ^{límite}. Esta tensión es la causa
del proceso de resolución. Pero una vez cumplida la condición y
consumida la libertad, la solución aniquila el problema. La resolución
es, pues, un proceso de aniquilación que elimina tanto la libertad como
la condición del problema para producir la solución.

\bigskip
\centerline{$\underbrace{\vcenter{
  \halign{\strut\hfil\rm#\crcr Libertad\cr Condición\cr}}
    }_{\hbox{\rm\strut Problema}} \, \Bigr\rbrace
 \mathop{\hbox to 80pt{\rightarrowfill}}\limits^{\hbox{\rm Resolución}}
 \hbox{\rm Solución}$}
\bigskip

También pueden ser útiles un par de ejemplos matemáticos para distinguir
la resolución de la solución. En un problema de cálculo
aritmético_{aritmética} la solución es un ^{número} y la resolución un
^{algoritmo}, como, por ejemplo, el algoritmo de la división. Y en un
problema de ^{álgebra}, la resolución es una cadena de equivalencias que
transforman la expresión de partida, que representa al problema, en otra
expresión que, para que pueda ser aceptada como solución, ha de ser un
^{axioma} o, en su defecto, una expresión ya probada por un ^{teorema}
anterior.


\Section La lógica simbólica

Si queremos encontrar el ^{yo}, y el yo es un ^{problema}, entonces
hemos de investigar en qué medios pueden habitar los problemas. Esta
cuestión abre una de las áreas más importantes de la ^{teoría del
problema}, porque la emparentan con la lógica y el lenguaje merced a una
inesperada relación cuyos detalles se encuentran en \EPA5. Ocurre que,
si se impone como requisito al diseñar una lógica, la condición de que
puedan representarse en ella problemas, resoluciones y soluciones,
entonces se obtiene una ^{lógica simbólica}. Dicho de otro modo, en un
^{lenguaje simbólico} se pueden expresar problemas, resoluciones y
soluciones. Este descubrimiento esclarece la verdadera naturaleza de los
simbolismos, porque fundamenta la fuerte relación observada entre el
simbolismo y el yo, entendido éste como el problema del sujeto.

Puede usted intepretar, al efecto de seguir estas explicaciones, que
lógica y lenguaje son sinónimos, o considerar, más exactamente, que una
^|lógica| es un sistema de ^{representación}, mientras que un
^|lenguaje| es un sistema de ^{comunicación}. La relación entre ambos es
muy fuerte, ya que sólo se puede comunicar, o expresar, aquello que se
puede representar y, además, en nuestro caso, véase ^>El pensamiento>,
el ^{pensamiento} es el habla simbólica muda, o sea, que seguramente
nuestra lógica simbólica es un lenguaje simbólico interiorizado, pero es
mejor no distraerse ahora con estas disquisiciones.

Para hacernos una idea de por qué para expresar problemas es necesario
un lenguaje simbólico, retomaremos el ejemplo de ^>El pronombre>. Al
expresar un problema, como `¿qué hacer?', es preciso que el ^{pronombre}
interrogativo, `qué', no se refiera a ninguna cosa. Debe quedar
indefinido para que el problema lo sea. Si, por ejemplo, en el caso
propuesto, `qué' se refiriera a la acción de andar, y no a cualquiera
indeterminada, entonces la expresión `¿qué hacer?'\ significaría
\meaning{andar}, que no es ya un problema.

En la expresión de todo ^{problema} debe haber alguna ^{incógnita}, que
es una palabra sin ^{significado} o, más oportunamente, libre de
significado, que representa la libertad del problema. El resto de las
palabras expresarán las condiciones y las informaciones adicionales.
Esta es, por supuesto, una primera aproximación, porque muchas de las
condiciones que sí han de ser tenidas en cuenta, no tienen que ser
expresadas. En `¿qué hacer?', se trata de hacer, pero esta única
condición explícita no es, seguro, la única condición del problema. La
gravedad y la muerte son seguramente algunas de las condiciones tácitas.
Del mismo modo, podría prescindirse de la palabra que designa la
libertad, el pronombre interrogativo `qué', si el tono o el contexto
fueran suficientes para indicar que se trata de un problema.

Dejaremos, a pesar de su interés, estos asuntos de la ^{economía del
lenguaje} que, por otra parte, no cambian el hecho básico: para
representar la libertad hay que prescindir del significado. O, dicho de
otro modo, la ^{semántica} es insuficiente para representar la libertad,
lo que nos da la clave de la característica fundamental del simbolismo,
a saber, que todo simbolismo dispone de dos capas: la semántica y la
^{sintaxis}. Retomando el ejemplo, `qué' es un artefacto meramente
sintáctico.


\Section La semántica y la sintaxis

La imagen de un ^{simbolismo} dispuesto en capas explica que un
simbolismo puede construirse añadiendo una capa sintáctica sobre una
lógica semántica anterior, que es, seguramente, el proceso por el que se
elaboró el nuestro desde un lenguaje sígnico puramente semántico, según
vimos en ^>El lenguaje simbólico>.

Una manera de conseguir que la nueva capa sintáctica sea más capaz que
la capa semántica que la sostiene consiste en representar, en la capa
sintáctica, todos los objetos de la capa semántica y añadirle, además,
otros objetos, algunos de ellos libres de significado. En estas
circunstancias se distinguen dos tipos de objetos sintácticos: llamamos
^|cosa| a un objeto sintáctico que representa directamente a un objeto
semántico, y denominamos ^|concepto| al objeto que no es una cosa, es
decir, al objeto que no tiene un ^{significado} inmediato. Quiere esto
decir que, si el concepto tiene significado, éste se construye a partir
del significado de las cosas. O, dicho aún de otro modo, el concepto
tiene un significado elaborado, o no tiene significado.

Como la semántica es incapaz de representar la libertad, el ^{problema},
que aúna condición y libertad, solamente podrá expresarse
sintácticamente. Por esta razón el problema ha de ser un concepto y ha
de residir en la sintaxis.

La ^{libertad} está en los problemas, pero no puede estar en las
soluciones, porque las soluciones han de estar completamete
determinadas, sin ambigüedad, y sin ningún grado de libertad. Además,
las soluciones ya se representaban en la lógica semántica, véase ^>El
problema>, de modo que, en la sintaxis, las soluciones serán cosas.
Dicho de otra manera, las soluciones tienen significado. Pero,
obviamente, aunque todas las soluciones son cosas, no todas las cosas
son la solución del problema.

El proceso de resolución toma el problema expresado sintácticamente y
entrega la expresión sintáctica de su solución, que es una cosa. La
resolución es, pues, una transformación sintáctica. Y, por consiguiente,
la representación sintáctica de una ^{resolución} será un ^{algoritmo},
que es como denominamos a una expresión sintáctica que representa a una
transformación sintáctica. En definitiva, los algoritmos tienen que ser
conceptos sintácticos y recursivos_{recursividad}.

De manera que, mientras que el problema y la resolución han de ser
conceptos sintácticos y recursivos, la solución ha de ser una cosa
semántica; los detalles técnicos de esta conclusión se encuentran en
\EPA{5.9}. Y resulta así que la ^{solución} no puede quedarse solamente
en la sintaxis, tiene que transcenderla, o el simbolismo, convertido en
pura sintaxis y encerrado en sí mismo, sería inútil. Visto de este modo,
el proceso de resolución traslada el problema de la sintaxis a la
^{semántica} o, dicho de otro modo, la resolución busca el significado
del problema.

Lo novedoso de la ^{sintaxis} recursiva es que permite la completa
representación de problemas, resoluciones y soluciones, y por esto sirve
para resolver problemas. Un ejemplo: el diseño sistemático de
herramientas_{herramienta}, que son resoluciones hechas cosas, necesita
de una elaboración simbólica; véase ^>La herramienta>.


\Section La abstracción

La ^{sintaxis}, al no tener que dar un ^{significado} a cada ^{objeto},
tiene un mayor poder expresivo que la ^{semántica}. Ya hemos visto que,
por ejemplo, así como la sintáxis permite expresar problemas, no es
posible, en cambio, expresar semánticamente un problema, porque la
^{incógnita} del problema ha de estar necesariamente libre de cualquier
significado. Pero hay más.

Cuando un problema tiene una única solución, se puede emplear el
^{problema} para referirse a su solución que, según hemos visto en la
sección anterior, _>La semántica y la sintaxis>, coincide con su
significado. Este artilugio se denomina ^{perífrasis}, ya que el
problema se refiere, en primera instancia, a sí mismo, y sólo mediante
un rodeo a su solución. En principio, esta perífrasis puede no parecer
muy interesante, excepto para componer adivinanzas_{adivinanza} o
metáforas_{metáfora}. Sin embargo, la utilización perifrástica de
problemas para referirnos a sus soluciones, ya sea a todas ellas o a
cualquiera de ellas, es un procedimiento enormemente fructífero que se
denomina ^|abstracción|.

Porque, si queremos referirnos a todas las cosas que tienen determinada
forma y tales utilidades, lo que hacemos es construir un problema cuya
condición es la conjunción de las condiciones consistentes en tener esa
determinada forma y esas tales utilidades. De ese modo las soluciones
del problema construido coinciden con aquellas cosas a las que queremos
referirnos. Mirado con calma el artificio es sencillo, siempre que se
define algo por sus propiedades, se construye un problema cuya condición
es la conjunción de las propiedades. Esta es la manera en la que se
construyen los conceptos abstractos.

Puesto que tras cada concepto abstracto existe un problema, no puede
haber conceptos abstractos sin un ^{lenguaje simbólico} capaz de
expresar problemas.


\Section Yo estoy en la sintaxis

Para proseguir sin olvidar que toda esta apresurada investigación sobre
la ^{teoría del problema} tiene el propósito de dilucidar el yo, nos
detendremos a considerar algunas nuevas conclusiones que ya se le pueden
aplicar.

El ^{yo}, por ser ^{problema}, no es una ^{cosa}, sino un ^{concepto}.
Y, por ser concepto, el yo habita en la capa sintáctica, no en la
semántica. Esto puede no parecer muy sorprendente, aunque demuestra que
el yo está íntimamente ligado al ^{lenguaje simbólico} y que el yo no es
físico ni real. Pero, al aplicar la ^{abstracción} al problema del
sujeto, esto es, al considerar sus soluciones, caemos inmediatamente en
la cuenta de que el problema del sujeto no tiene solución. El ^{problema
del sujeto}, ¿qué hacer para no morir?, no tiene solución porque es
seguro que, haga lo que haga, yo moriré_{muerte}. Y esto sí que es
enigmático. El yo, que igualamos al problema del sujeto y que, por
abstracción, asimilamos a su solución, es ahora un ^{enigma}. ¿Qué es la
^{solución} de un problema sin solución?


\Section Yo soy paradójico

Una de las maneras de construir un concepto paradójico consiste en
referirse a las soluciones de un ^{problema} sin ^{solución}. Sucede,
pues, que el yo es paradójico, pero que no cunda el pánico. Yo soy
paradójico, sí, pero ¿qué es exactamente una paradoja?

Técnicamente, véase \EPA{5.7.1}, una ^|paradoja| es un ^{objeto}
sintáctico sin referente semántico, o sea, que una paradoja es un
concepto sin significado. La ^{libertad} incondicionada es, según esta
definición, una paradoja, y también es paradójica una expresión como
`esta frase es falsa', porque si es cierto lo que afirma, entonces es
falsa, pero si es falsa, entonces resulta que lo que afirma es cierto, y
entonces es falsa, y entonces cierta, y así por siempre sin alcanzar
nunca el ^{significado} final de la frase.

Para entender qué consecuencias tiene afirmar que el yo es paradójico
hay que hacer un par de deducciones.

Para la primera es preciso recordar:
\beginpoints
\point que el ^{yo} es equivalente al ^{problema del sujeto},
\point que un problema sirve para referirse, por ^{abstracción},
       a sus soluciones, y
\point que una solución es un significado del problema,
\par\noindent de donde se concluye que, si el yo es paradójico, es
porque el problema del sujeto no tiene solución. Nada nuevo---ya sé que
yo moriré---pero que ratifica los fundamentos del yo paradójico.
\endpoints

Para realizar la segunda deducción necesitamos una observación previa.
Sucede
\beginpoints
\point que en todo problema se manifiesta una ^{tensión} entre la
libertad y la condición que el proceso de resolución elimina,
aniquilando la libertad y la condición, y produciendo una solución,
\par\noindent o sea, en negativo, resulta
\point que, en todo problema sin solución, el proceso de resolución no
puede culminar en solución, manteniéndose la tensión entre la libertad y
la condición.
\par\noindent Esta es la razón, supongo, por la que las paradojas causan
desasosiego. Como un problema sin solución no puede dejar de ser
problema para ser solución, podemos afirmar que los problemas sin
solución son los únicos problemas necesariamente estables.
\endpoints

La segunda deducción es ahora sencilla. Dado
\beginpoints
\point que todos los problemas sin solución son estables, y
\point que el problema del sujeto no tiene solución,
\par\noindent se deduce que el problema del sujeto es estable y
que no se puede eliminar la libertad del yo.
\endpoints

En definitiva, el problema paradójico sin solución es el único sistema
capaz de confinar la libertad que, si no, se resolvería desapareciendo
aniquilada con la condición. El yo es problemático, y no puede dejar de
serlo, porque es paradójico. El yo paradójico mantiene la tensión entre
la libertad y la ^{muerte}.


\Section La inmortalidad

Sospecho que todas estas deducciones pueden confundir más que aclarar.
Porque se puede pensar, con razón, que, al morir_{muerte} el ^{sujeto},
también desaparece la ^{libertad}, y la ^{tensión}, y hasta el ^{yo}
paradójico. Esto es cierto, pero no significa que el ^{problema del
sujeto} tenga solución, sino justo lo contrario. Para solucionarlo, el
sujeto ha de conseguir la inmortalidad absoluta, sin condiciones, y lo
que estas deducciones afirman es que tal sujeto inmortal de necesidad,
sin problema vital, ni tendría yo, ni sería libre.

Siendo necesariamente inmortal, el sujeto no tendría ninguna
preocupación, dejaría de ser inquisitivo, ¿para qué?, su yo se
anquilosaría y simplemente viviría eternamente. Aunque no sé si podría
llamarse vivir_{vida} al vivir del sujeto inmortal de necesidad, ya que
su vivir no dependería de comer, ni de respirar, ni de ningún otro
condicionante. Si la ^{inmortalidad} no lo fuera de necesidad, si no
fuera incondicional, entonces el sujeto habría de mantener ciertas
condiciones de inmortalidad, como disponer de ^{alimento} y de ^{aire},
y el problema del sujeto mantendría su vigencia, aunque hubiera sido
parcialmente solucionado.


\Section Yo estoy vivo

La verdadera ^{solución} del ^{problema del sujeto} aniquilaría
efectivamente la ^{libertad} y la condición que constituyen el problema,
pero cumpliendo la condición. Esto es, la solución debería aniquilar el
problema del sujeto, o sea, debería aniquilar al yo, pero haciendo
inmortal al sujeto. Una vez solucionado el problema del sujeto, no
habría yo pero sí habría sujeto.

Esto explica que el ^{suicidio} no es la solución del problema del
sujeto, aunque efectivamente elimine el problema. Como vemos, para
percarse de que el suicidio no es solución, sino fracaso, es necesario
distinguir al yo del sujeto.

Estas reflexiones nos permiten poner, al fin, un pie fuera del yo
primigenio. El ^{yo} primitivo es la única ^{certeza}, pero no es lo
único que hay, ya que necesita, al menos, de un sujeto que lo soporte.
Para asentar este primer paso hacia afuera, hemos de estudiar al sujeto.
Y, para estudiar al sujeto a partir de su yo, tenemos un par de datos.

Uno es que el ^{sujeto} se identifica con su yo. Como el yo es un
^{problema}, la solución del problema ha de ser de la máxima importancia
para el sujeto. Y la solución, como sabemos, ha de cumplir la condición,
que es no morir. De manera que el mayor interés del sujeto es vivir: el
sujeto está vivo.

Además, sabemos que el sujeto ha de disponer de un ^{simbolismo} para
poder soportar el yo, que es sintáctico. Esto significa que cuenta con
una capa sintáctica que puede representar problemas, resoluciones y
soluciones, donde habita el yo, y una capa semántica en la que ejecuta
las soluciones que le sirven para no morir, esto es, para vivir: el yo
es una parte del sujeto.

En definitiva, el yo es parte de un sujeto que está vivo. Y si el yo es
parte de la vida, ha de ser porque el problema del sujeto es parte de un
problema más general, a saber, el ^{problema de la supervivencia}, que
es como nominamos al problema que define la ^{vida}.
$$\vbox{\halign{\hfil#&\quad\hfil#\hfil\quad&#\hfil\cr
 \vtop{\halign{\hfil#\hfil\cr Yo\cr $\parallel$\cr}}& %$\Updownarrow$\cr}}&
 $\subset$&
 \vtop{\halign{\hfil#\hfil\cr Vida\cr $\parallel$\cr}}\cr
 \strut Problema del sujeto&
 $\subset$&
 Problema de la supervivencia\cr}}$$


\Section El problema de la supervivencia

Así como del ^{problema del sujeto}, por ^{introspección}, pudimos
averiguar algo, del ^|problema de la supervivencia| nada sabemos, sino
que es un ^{problema}. Y ya que nada sabemos, nada supondremos.

Este no suponer es, concretamente, el fundamento de la ^{teoría de la
subjetividad}, y, además de ser razonable no suponer cuando nada se
sabe, también es consecuente, como mostrará su desarrollo. Lo repetiré
de otro modo, para recalcar su importancia. La vida es un problema, y no
es más que un problema. Toda la teoría de la subjetividad se deriva de
este postulado y, por eso, {\sc epa}^(Casares1999) se limita a plantear
el problema y a resolverlo.

De manera que, según la clasificación de problemas vista en ^>La teoría
del problema>, el problema de la supervivencia es un ^{problema
aparente}, porque no aporta ninguna ^{información}. El problema de la
supervivencia es el problema universal, aquél del que son parte todos
los demás problemas, y consiste sólo en ^{libertad} y en condición, ya
que es un problema aparente.
$$\hbox{Problema de la supervivencia} = \hbox{Problema aparente}$$

La naturaleza problemática del yo, que lo iguala al problema del sujeto,
nos hizo estudiar la teoría del problema, y ahora la naturaleza aparente
de la vida, que la iguala al problema de la supervivencia, nos lleva al
estudio del problema aparente.
$$\hbox{Vida} \etapa \hbox{Problema aparente}$$


\Section El problema aparente

Lo propio del ^|problema aparente| es que no aporta ^{información}. Nada
se sabe, ni sobre qué resoluciones son más propicias ni sobre la propia
condición del problema aparente. Es decir, no se conoce la condición,
que, en consecuencia, ni siquiera puede ser enunciada. Dicho de otro
modo, ante un problema aparente lo único lícito es intentar su
resolución. Por esto, el problema aparente es el problema puro, o
problema mínimo, o, por decirlo aun de otro modo, es el problema sin
información. Para ser exactos, la única información que aporta un
problema aparente es que es un problema, y que no es otra cosa.

Los problemas aparentes son tan peculiares que una primera impresión
puede inclinarnos a no prestarles la debida atención. Como no se dispone
de ninguna información, cualquier ^{resolución} es igualmente válida y,
en principio, parece que nada más puede ser dicho. Esto es cierto, pero
sólo en principio, porque hay maneras de ir más lejos.

Si sólo se dispone de una oportunidad para resolver un problema
aparente, entonces podemos elegir al ^{azar} la resolución, porque
ninguna otra elección es razonablemente mejor, ni tampoco peor.
Ejecutada la resolución puede ocurrir que solucionemos el problema, o
que no lo solucionemos. En cualquiera de los dos casos, ahora ya
disponemos de un bit de información sobre el problema, a saber, si la
resolución ejecutada lo ha solucionado, o no. Por lo tanto, si podemos
ejecutar otras resoluciones, e ir pasando la información obtenida de
unas resoluciones a otras, conseguimos una manera de resolverlo que es
mejor que el puro azar porque está más informada.

El proceso de ^{evolución} de ^[Darwin]^(Darwin1859) utiliza, contra el
problema aparente de la supervivencia, este método de repetición de
resoluciones. Básicamente, cada ser vivo_{vida} es un resolutor del
^{problema de la supervivencia} que, antes de fracasar y morir, hace
réplicas_{replicación}, solo o en pareja, y no siempre perfectas, que
incluyen información relativa a su manera de resolverlo. Para que, a su
vez, las réplicas efectúen réplicas, es preciso que las primeras venzan
a la ^{muerte}, al menos, hasta que vivan las nuevas réplicas; esta
criba se denomina ^{selección}. La distinción entre ^{solución} y
^{resolución} es aquí crucial, ya que todo ser vivo es mortal, o sea,
que no es una solución, pero sí es un resolutor.

El nombre del problema aparente tiene sentido porque, como vemos, lo
único que se conoce de él es su ^{apariencia}, esto es, su reacción
externa_{exterior} a nuestras acciones para intentar su resolución, y no
nos muestra nada de su ^{interior}. Es como si tratáramos de abrir una
^{caja fuerte} manipulando sus dispositivos externos pero sin
información alguna sobre el mecanismo de apertura. Esta manera de
hablar, sin embargo, puede inducir a error, porque da por supuesto que
el problema aparente tiene un interior que es el responsable de su
apariencia externa. Esta suposición, aunque parezca inevitable, es
ilícita y se denomina ^{logicismo}; lo veremos más adelante, en
_>El problema del aprendiz>.

El problema aparente es el que ^[Klir]^(Klir1969) denomina problema de
la ^{caja negra} puro.


\Section La evolución y la resolución

El ^{problema aparente} modela los aspectos
epistemológicos_{epistemología} de la ^{vida}. Esto quiere decir que no
toma en consideración lo que no afecta al ^{conocimiento}, por
importante que pueda ser para la propia vida. Por ejemplo, no considera
los detalles relativos a cómo se efectúan las réplicas_{replicación}.
Pero, a la vez, al definir la vida como problema aparente, la
generalizamos, ya que la vida no queda comprometida con la ^{química
orgánica} que la hace posible de la manera que la conocemos.

De modo que, si todo esto es correcto, ha de haber una correspondencia
entre la ^{resolución} teórica del problema aparente y la ^{evolución}
darviniana, en la que cada resolución teóricamente favorable se
corresponda con algún paso dado efectivamente por la vida. Así, por
ejemplo, si en la resolución del problema aparente se muestra que es
mejor hacer modelos_{modelación} del exterior que no hacerlos, entonces
hemos de concluir que el proceso de evolución favorecerá a aquellos
individuos que codifiquen genéticamente los mecanismos de formación de
modelos del entorno.
$$\hbox{Resolución del problema aparente} \iff
  \hbox{Evolución darviniana}$$

Si el ^{problema aparente} generaliza la ^{vida}, la ^{resolución} del
problema aparente generalizará la ^{evolución} darviniana. Trataremos
ahora de probar la validez de esta correspondencia entre la resolución
teórica del problema aparente y la evolución darviniana, pero para
probarla hemos de desarrollar una resolución teórica del problema
aparente. Lo primero que se necesita para desarrollar tal resolución
teórica es formalizar el problema aparente. El ^{problema aparente
formalizado} está aún más lejos de la vida a la que define que el
problema aparente, por lo que la formalización puede introducir
distorsiones en la definición. A pesar de este reparo, el siguiente paso
en nuestra salida desde el yo primigenio consistirá en formalizar el
problema aparente para atacarlo teóricamente.
$$\hbox{Problema aparente} \etapa
 \hbox{Problema aparente formalizado}$$


\Section El universo

Un problema aparente es un ^{problema} en el que no se tiene ninguna
^{información}, así lo presentamos, pero debemos concretarlo un poco
más. En un problema aparente hay ^{libertad} para actuar_{acción} y la
^{condición} de que las reacciones_{reacción} sean buenas, y no malas.
Es decir, la condición del problema aparente es la mínima posible y la
relación entre las acciones ejecutadas y las reacciones recibidas es
completamente desconocida. Denominaremos ^{entorno} o ^|universo| al
objeto hipotético que relaciona las acciones con las reacciones, y así
podemos rehacer la frase anterior: en el problema aparente el entorno o
universo es completamente desconocido, e incluso podría no existir.

Un problema aparente no se puede solucionar \latin{a priori}, o sea,
teóricamente, porque, ya lo hemos visto en _>El problema aparente>, en
principio y al no disponer de información, cualquier resolución es
igualmente razonable. Es decir, ante un problema aparente es imposible
diseñar una solución y argumentar razonablemente que lo es, porque no se
dispone de información para argumentar nada. Para obtener información
del problema aparente, también lo hemos visto ya, hay que enfrentarse
repetidamente a él.

¿Qué información segura puede obtenerse de un problema aparente? En
principio lo que obtenemos es la información segura de que ejecutada la
resolución que denominaremos $\Re$, después de la serie de resoluciones
$\Re_0, \Re_1, \ldots \Re_t$, se soluciona el problema, o no, según haya
sido el caso. Si repetimos la resolución $\Re$, sin embargo, no podemos
asegurar que el resultado se repita, porque ahora la serie de
resoluciones ya ejecutada no es $\Re_0, \Re_1, \ldots \Re_t$, sino
$\Re_0, \Re_1, \ldots \Re_t, \Re$. Es decir, que el entorno puede tener
^{memoria} y reaccionar de distinto modo a las mismas acciones
dependiendo de su estado. Y además, también podría ocurrir que la
relación entre las acciones ejecutadas y las reacciones recibidas
tuviera un componente aleatorio_{azar}, otra razón por la cual la
repetición de las acciones tampoco asegura una repetición de las
reacciones.

De modo que la información sobre el entorno obtenida al enfrentarse al
problema aparente toma la forma de un ^{autómata} finito y
probabilístico. Es un autómata, y no una ^{función}, porque el
\hbox{entorno} puede tener memoria. Finito, no por el universo, que
podría no serlo, sino por las limitaciones del propio aparato de
representación. Y probabilístico, otra vez, porque el entorno podría
serlo. El universo puede entonces ser cualquier autómata finito y
probabilístico o, al menos, con estas concreciones ya se puede
formalizar el problema aparente.

Llegados aquí, se impone otra parada para admirar los alrededores antes
de proseguir.


\Section El tiempo y el espacio

Para formalizar el problema aparente hemos introducido,
subrepticiamente, dos conceptos: el ^{tiempo} con la ^{memoria} y el
^{espacio} con la ^{acción} y la ^{reacción}.

Ya hemos visto que, para resolver el problema aparente mejor que al
azar, había que pasar ^{información} de las resoluciones ya hechas, o
sea, de las resoluciones pasadas, a las que se están intentando, o sea,
a las resoluciones presentes. Por esta razón parece necesaria una
primera distinción temporal entre el ^{pasado} y el ^{presente}.

La ^{acción} y la ^{reacción}, que conjuntamente denominaremos
^{interacción}, necesitan de un ^{dentro} y un ^{fuera} para distinguir
los dos sentidos, porque la acción va de dentro a fuera, sale, y la
reacción viene de fuera a dentro, entra. Por esta razón parece necesaria
una primera distinción espacial entre el dentro y el fuera, entre el
^{interior} y el ^{exterior}.

En la definición primitiva del problema aparente se usan conceptos como
^{libertad}, ^{condición} e ^{información}, pero no tiempo ni espacio.
Creo, sin embargo, que la concreción espacio-temporal del problema
aparente es la que mejor formaliza el problema de la supervivencia
porque permite enmarcar conceptos fundamentales como ^{cosa} y
^{muerte}, o ^{sustantivo} y ^{verbo}, con espacio y tiempo,
respectivamente. Además, el ^{significado} aparece para relacionar las
condiciones internas al resolutor con las que son exteriores a él. Y los
razonamientos que ocupan las dos próximas secciones fundamentan la
^{irreversibilidad} del tiempo y, al hacerlo, dan al ^{futuro} su
carácter abierto. Pero no hay que olvidar que, al menos en teoría, el
problema aparente podría concretarse de otras maneras.


\Section El conocimiento es provisional

Repetir una ^{resolución} que antes solucionó un problema aparente no
asegura que ahora lo solucione. Ya hemos visto que esto podría ocurrir
por estar ahora el ^{universo} en un estado_{memoria} no propicio o,
simplemente, porque el ^{azar} interviene en contra. Por más perspicaces
que seamos, nunca podemos estar seguros de acertar una ^{predicción}
ante un problema aparente. La ^{información} que puede obtenerse de un
problema aparente es provisional.

La información obtenida del ^{problema aparente} se corresponde, en la
^{evolución} darviniana, al ^{conocimiento} del ^{universo} que la
^{vida} puede alcanzar. En consecuencia, lo que esta propiedad abstracta
obtenida en la resolución teórica del problema aparente significa para
la vida es que el conocimiento es provisional, hipotético, tentativo, y
nunca es seguro.

No hay conocimiento que sea absolutamente cierto_{certeza}, y aunque el
^{sol} aparezca cada día por el este, y nuestra predicción de que mañana
nacerá el sol por el oriente haya acertado mil veces y más, aun así, no
podemos asegurar que ^{mañana} se levantará el sol por el este.

Incluso la información codificada genéticamente es provisional. Por esta
razón la información genética puede llegar a ser disfuncional causando,
si el caso es grave, la ^{extinción} de la especie. Y esa misma
conclusión también se aplica, porque las heredamos genéticamente, a
nuestra percepción, que determina las cosas que vemos, y a nuestras
emociones, que les dan significado. Olvidar esto causa el error que
denominamos ^{objetivismo} o, más en general, ^{logicismo}, que
presentaremos en _>El problema del aprendiz>.


\Section La vida es paradójica

No podemos asegurar que la ^{resolución} que ahora vamos a intentar
solucionará el ^{problema aparente} al que nos enfrentamos, como
acabamos de mostrar. Y ya que en ningún momento podemos asegurar la
solución, resulta que el problema aparente no tiene una solución
definitiva. Y si la solución de un problema no lo aniquila
definitivamente, entonces, en puridad, no es ^{solución}. Llegamos así a
la formulación más resumida de esta propiedad: los problemas aparentes
no tienen solución.

Quiere esto decir que todo problema aparente es paradójico_{paradoja} y
en consecuencia, según una conclusión ya alcanzada en ^>Yo soy
paradójico>, todo problema aparente es estable porque no puede dejar de
ser problema.

Transponiendo, otra vez, de la ^{teoría del problema} a la ^{vida},
concluimos que la vida es paradójica y, por eso, aun siendo ^{problema},
se mantiene irresoluble como tal. La vida es problemática, y además no
puede dejar de serlo, porque es paradójica.


\Section El álgebra automática

No presentaremos aquí ninguna formalización matemática del ^{problema
aparente}, porque es demasiado técnica para los fines más filosóficos de
este ensayo y porque, si está usted interesado, puede consultar
\EPA{1.4}, en donde encontrará una. Aun así, describiremos
suficientemente la formalización de {\sc epa} para que se capten los
aspectos epistemológicos más interesantes.

Puesto que formalizamos el universo como un ^{autómata} finito y
probabilístico, en ^>El universo>, el problema aparente formalizado
estará planteado en una lógica que permita representar autómatas finitos
y probabilísticos, además de problemas, resoluciones y soluciones.
Nuestra formalización usa el álgebra de autómatas, o ^|álgebra
automática|, que es una ^{lógica simbólica} y que permite representar
autómatas finitos, binarios, síncronos y probabilísticos (\EPA{A}).


\Section El autómata

Pero, ¿qué es un autómata? Un ^|autómata| toma ^{datos} del exterior y
produce datos que emite al exterior. Llamaremos, a los primeros, ^{datos
de entrada} y, a los segundos, ^{datos de salida}. Los datos de salida
dependen tanto de los datos de entrada como del ^{estado} del autómata.
Además, el autómata cambia de estado, y los cambios de estado también
dependen del estado y de los datos de entrada.

La anterior descripción de un autómata es algo teórica, de manera que
presentaremos un ejemplo que ayude a fijar el concepto. Podemos decir
que una ^{calculadora} con ^{memoria} es un autómata, ya que produce
datos, que son los números que muestra en su pantalla, y toma datos, que
son los números y las operaciones que tecleamos. Además, los resultados
dependen, en ocasiones, del contenido de la memoria, y el contenido de
la memoria de acumulación depende de lo que tecleemos y de su propio
contenido.

Hay más ejemplos. El ^{sistema nervioso} de un animal toma datos
del exterior a través de sus sentidos y produce datos que sus músculos y
glándulas transforman en acciones. Y además, los datos producidos
dependen tanto de la ^{percepción} como del estado interno del animal,
ya que no actúa del mismo modo sediento_{sed} que satisfecho.

El autómata es un modelo muy general, sobre todo si se tiene en cuenta
que un ^{autómata sin memoria} sigue siendo un autómata. Porque, bien
mirado, un autómata sin memoria es un autómata con un único estado, o
sea, simplemente un autómata que nunca cambia de estado. La ^{memoria}
de un autómata es la medida de su número de estados.

Es un modelo tan general que cualquier sistema de tratamiento de datos
puede verse como autómata, incluidas las computadoras. En concreto, las
computadoras más frecuentes en la actualidad, que se ajustan a la
arquitectura de _[von] ^[Neumann]^(Neumann1945), son autómatas
síncronos, finitos y binarios. Las redes de computadoras son también
autómatas, incluida la ^{Internet}, que no es síncrona porque no tiene
un ^{reloj} de referencia.

La ^{computadora} es especialmente importante porque, ignorando las
limitaciones físicas, es un autómata capaz de imitar a cualquier otro
autómata, esto es, se puede comportar como cualquier otro autómata si
tiene el ^{programa} adecuado. Técnicamente, la computadora es un
{\UP}, que presentaremos en _>La máquina universal de Turing>.

El ^{álgebra automática} emplea autómatas finitos, binarios, síncronos y
probabilísticos. Un autómata finito y binario_{autómata binario} emplea
una codificación de los datos en base a dos símbolos que
convencionalmente son el $1$ y el $0$. La codificación binaria es la más
simple y por esto la usamos sin que cause pérdida alguna de generalidad.
Un autómata finito y síncrono_{autómata síncrono} usa una señal de reloj
única como referencia que marca cuando ocurren todos los cambios de
valor de los datos de salida y de estado. La sincronía, que consiste en
considerar que todas las operaciones duran lo mismo, es conceptualmente
más sencilla que la asincronía, que exige tener en cuenta los diferentes
tiempos de ejecución de cada operación, y por esto la usamos sin que nos
cause, tampoco, pérdida alguna de generalidad. En lo que sigue no
volveremos a mencionar que los autómatas finitos a los que nos referimos
son binarios y síncronos, ya que no afecta a los resultados.

Visto lo visto, puede usted quedarse con el siguiente ardid para
identificar autómatas finitos y olvidarse del resto de lo dicho en esta
sección. Si puede programarse en una computadora, o sea, si una
computadora puede hacerlo, entonces existe un autómata finito que
también lo hace, y a la viceversa.


\Section El comportamiento

`Comportamiento' es una palabra muy genérica sobre la que ya hablamos en
^>La realidad>, pero que aquí utilizamos técnicamente, esto es, con una
definición precisa; véase \EPA{A.5.5}.

Decimos que dos autómatas tienen el mismo ^|comportamiento| si no es
posible distinguirlos desde el ^{exterior}. Interiormente pueden ser
diferentes, y por ejemplo uno puede utilizar más estados que el otro,
pero si manejando los ^{datos de entrada} y observando los de
salida_{datos de salida} es imposible distinguir a uno del otro,
entonces decimos que el comportamiento de ambos es idéntico. Por
consiguiente, si nos resulta indiferente la construcción interna del
autómata, como es el caso en esta investigación teórica, entonces
solamente nos interesa el comportamiento del autómata. Y, en conclusión,
consideraremos que `^{autómata}' y `comportamiento' son palabras
sinónimas, sabiendo que la diferencia entre ellas es de detalle y le
interesa a los ingenieros_{ingeniería}, pero no a nosotros, véase
\EPA{1.4.3}.

Hay, no obstante, un asunto que puede preocuparle. En algunos momentos
hablamos de autómatas capaces de varios comportamientos, lo que parece
un contrasentido. No hay truco, es posible, está demostrado
matemáticamente por ^[Turing]^(Turing1936), y la prueba palpable es que
las computadoras son capaces de varios comportamientos. La argucia
consiste en considerar que hay dos tipos de ^{datos de entrada}, los
ordinarios y otros, llamados ^{programa}, que especifican el
comportamiento. Porque, si tomamos un autómata~$\aut A$ así, y fijamos
el programa a determinado valor, y observamos los ^{datos ordinarios} y
los de salida, pero no el programa, entonces el autómata~$\aut A$ se
comportará como cierto autómata~$\aut B$, pero si fijamos el programa a
otro valor y lo observamos del mismo modo, entonces se comportará, en
general, como otro autómata~$\aut C$. El programa consigue, de esta
manera, que un autómata con programa imite a otros autómatas. Por
supuesto, si miramos todos los ^{datos}, incluido el programa, entonces
también el autómata con programa tiene un único comportamiento.


\endinput
