% sll6.tex (RMCG20000731)

\Part Conclusión

\Section El tao

% Ideograma de camino (tao) pequeño
\MTbeginchar(9pt,12pt,0pt);
 \MT: save border, ladderw, ladderh, gap;
 \MT: border = w/10; gap = h/20;
 \MT: ladderw = 3w/8; ladderh= h/2;
 \MT: % pickup thick_pen;
 \MT: pickup pencircle xscaled0.75pt yscaled0.4pt rotated 30;
 \MT: z1bl = (w/2,h/8); % One for the ladder
 \MT: z1tr - z1bl = (ladderw,ladderh);
 \MT: z1tl - z1bl = (0,ladderh);
 \MT: z1br - z1bl = (ladderw,0);
 \MT: draw z1tl .. z1bl; draw z1tr .. z1br;
 \MT: draw z1tl .. z1tr; draw z1bl .. z1br;
 \MT: z1mo = 2/3[z1bl,z1tl]; z1md = 2/3[z1br,z1tr] - (gap,0);
 \MT: z1no = 1/3[z1bl,z1tl]; z1nd = 1/3[z1br,z1tr] - (gap,0);
 \MT: draw z1mo .. z1md; draw z1no .. z1nd;
 \MT: z2o = (0,h/8); z2d = (w,0); % Two for the base
 \MT: x2m1 = w/4; x2m2 = 3w/4; y2m1 = y2o; y2m2 = y2d;
 \MT: draw z2o{right} .. z2m1 .. z2m2 .. z2d;
 \MT: y3ao = 4/3[y1bl,y1tl]; % Three for the crown
 \MT: x3ao = x1tl - border; z3ad = (w,y3ao);
 \MT: draw z3ao .. z3ad;
 \MT: z3eo = z1tl+(0,gap); z3em = 1/2[z3ao,z3ad]; z3ed = (w-border,h);
 \MT: draw z3eo .. z3em .. z3ed;
 \MT: z3io = (x3ao,y3ed); z3id = 1/2[z3ao,z3ad];
 \MT: draw z3io .. z3id;
 \MT: x4a = x4i; y4e = 1/2[y4i,y4a]; % Four for the circumflex
 \MT: z4e = 1/2[z3ao,z3ed] - (3w/8,gap);
 \MT: y4a - y4i = h/8; % circumflex height
 \MT: x4e - x4a = w/8; % circumflex width
 \MT: draw z4a -- z4e -- z4i;
 \MT: x5e = x1tl - w/8; % Five for the seven; seven x position
 \MT: x5a = x5u; x5e - x5a = w/5; % seven width
 \MT: x5i = 1/3[x5a,x5e]; x5o = 2/3[x5a,x5e];
 \MT: y5a = y1tl - gap; y5u = y1bl; % seven y position
 \MT: y5a = y5e; y5i = 2/3[y5u,y5a]; y5o = 1/3[y5u,y5a];
 \MT: draw z5a .. z5e; draw z5e .. z5i .. z5o .. z5u;
 \MT: %pickup thin_pen; % The bounding box
 \MT: %draw (0,0) -- (w,0) -- (w,h) -- (0,h) -- cycle;
\MTendchar;
\newbox\taobox \setbox\taobox=\hbox{\raise-1.5pt\box\MTbox}
\def\tao{\copy\taobox}

El ideograma ^{chino} \tao, que a veces se transcribe `^{tao}', otras
`^{dao}' o, incluso en ^{japonés}, `d\=o', significa ^{camino}, y
también método y ley. Es el concepto central del ^{taoísmo}^(LaoZi), del
que toma su nombre, y es fundamental en el ^{budismo}.

En general, la ^{filosofía} oriental es introspectiva, porque parte de
la observación del ^{yo} puro. Y, al atender únicamente al yo, en el
centro de todo encuentra la ^{nada} absoluta, es decir, la ^{libertad}
entera, sin límites, en la que incluso es posible la conciliación de los
opuestos, del sí y el no; véase ^>Yo soy paradójico>. De aquí la
importancia de la dialéctica del \latin{yin} y el \latin{yang} o, en el
budismo ^{zen}, de las paradojas; véase ^[Suzuki]^(Suzuki1934).

Aunque los desarrollos posteriores del budismo y del taoísmo difieren
del camino de salida de esta ^{teoría de la subjetividad}, el punto de
partida de los tres caminos es el mismo: yo. Pero, si sólo se ve la
libertad del yo, que es su voluntad, entonces es imposible salir de él.
Porque si todo es libertad, si la libertad es absoluta, entonces no
tiene límites, y no se puede salir de donde no hay límites. Basta
reconocer un límite para que la situación sea completamente diferente. Y
ya hemos visto a donde conduce definir al yo como libre para no morir.
% Ideograma de camino (tao)
\MTbeginchar(30pt,40pt,0pt);
 \MT: save border, ladderw, ladderh, gap;
 \MT: border = w/10; gap = h/20;
 \MT: ladderw = 3w/8; ladderh= h/2;
 \MT: % pickup thick_pen;
 \MT: pickup pencircle xscaled1.4pt yscaled0.7pt rotated 30;
 \MT: z1bl = (w/2,h/8); % One for the ladder
 \MT: z1tr - z1bl = (ladderw,ladderh);
 \MT: z1tl - z1bl = (0,ladderh);
 \MT: z1br - z1bl = (ladderw,0);
 \MT: draw z1tl .. z1bl; draw z1tr .. z1br;
 \MT: draw z1tl .. z1tr; draw z1bl .. z1br;
 \MT: z1mo = 2/3[z1bl,z1tl]; z1md = 2/3[z1br,z1tr] - (gap,0);
 \MT: z1no = 1/3[z1bl,z1tl]; z1nd = 1/3[z1br,z1tr] - (gap,0);
 \MT: draw z1mo .. z1md; draw z1no .. z1nd;
 \MT: z2o = (0,h/8); z2d = (w,0); % Two for the base
 \MT: x2m1 = w/4; x2m2 = 3w/4; y2m1 = y2o; y2m2 = y2d;
 \MT: draw z2o{right} .. z2m1 .. z2m2 .. z2d;
 \MT: y3ao = 4/3[y1bl,y1tl]; % Three for the crown
 \MT: x3ao = x1tl - border; z3ad = (w,y3ao);
 \MT: draw z3ao .. z3ad;
 \MT: z3eo = z1tl + (0,gap); z3em = 1/2[z3ao,z3ad]; z3ed = (w-border,h);
 \MT: draw z3eo .. z3em .. z3ed;
 \MT: z3io = (x3ao,y3ed); z3id = 1/2[z3ao,z3ad];
 \MT: draw z3io .. z3id;
 \MT: x4a = x4i; y4e = 1/2[y4i,y4a]; % Four for the circumflex
 \MT: z4e = 1/2[z3ao,z3ed] - (3w/8,gap);
 \MT: y4a - y4i = h/8; % circumflex height
 \MT: x4e - x4a = w/8; % circumflex width
 \MT: draw z4a -- z4e -- z4i;
 \MT: x5e = x1tl - w/8; % Five for the seven; seven x position
 \MT: x5a = x5u; x5e - x5a = w/5; % seven width
 \MT: x5i = 1/3[x5a,x5e]; x5o = 2/3[x5a,x5e];
 \MT: y5a = y1tl - gap; y5u = y1bl; % seven y position
 \MT: y5a = y5e; y5i = 2/3[y5u,y5a]; y5o = 1/3[y5u,y5a];
 \MT: draw z5a .. z5e; draw z5e .. z5i .. z5o .. z5u;
 \MT: %pickup thin_pen; % The bounding box
 \MT: %draw (0,0) -- (w,0) -- (w,h) -- (0,h) -- cycle;
\MTendchar;
$$\box\MTbox$$


\Section Los caminos

El ^{camino de salida}, que da prioridad a la ^{introspección}, es el
^{camino} ^{oriental}. Presupone al ^{sujeto} y sus escuelas
solipsistas_{solipsismo} soslayan el objeto. El ^{camino de entrada},
que prefiere el ^{fenómeno}, es el camino ^{occidental}. Da por supuesto
el ^{objeto} y sus versiones materialistas_{materialismo} prescinden del
sujeto.

Si las filosofías orientales no pueden salir de la ^{libertad} sin
límite que es el yo puro, las filosofías occidentales no pueden entrar
en la libertad del yo. Así, por ejemplo, la ^{filosofía} natural
occidental, o abreviadamente ^{ciencia}, explica exclusivamente los
fenómenos físicos que pueden ser repetidos en condiciones experimentales
y, de este modo, proscribe la libertad, aunque puede
predecir_{predicción} con precisión y provecho cuál será el resultado de
un experimento físico. Por esta causa, la ciencia actual no puede
explicar el yo.

El camino oriental es insuficiente y el camino occidental también. Uno
tiene libertad sin condición, y el otro tiene condición sin libertad. La
síntesis es un problema, porque problema es, justamente, la síntesis de
libertad y de condición. Por esto, solamente es completo el camino de
ida y vuelta, o sea, el ^{bucle subjetivo} visto en ^>El bucle
subjetivo>, que parte del ^{yo} y que sale de él para llegar al
^{problema aparente} de la supervivencia, que es la fuente del
significado, como precisaremos en _>El problema es la fuente del
significado>, y que, de regreso al yo, da significado a cada una de sus
etapas y, finalmente, también al propio yo.


\Section La explicación problemática

Para superar las dificultades explicativas de la ^{ciencia} actual,
conviene considerar su historia reciente. La ^{explicación material},
vista en ^>La explicación material>, que imperó en las ciencias hasta la
revolución cuántica, explica cosificando, esto es, explica hasta
alcanzar las cosas, porque las cosas tienen un significado natural;
véase ^>La cosa y el concepto>, y ^>La semántica y la sintaxis>. La
^{explicación automática}, al descubrir problemas en la cosificación,
véase ^>La explicación automática>, prescinde de los significados del
^{conocedor} e intenta valerse únicamente de las previsiones_{previsión}
del ^{aprendiz}. Este retroceso evita usar los inexactos y no razonados
modelos alcanzados durante la evolución, véase ^>El conocimiento es
provisional>, pero produce explicaciones sin significado, esto es,
descripciones. El verdadero progreso se consigue, no retrocediendo del
conocedor al aprendiz, sino avanzando al ^{sujeto}. Este progreso
necesita, pues, una teoría del sujeto. Y, esta ^{teoría de la
subjetividad}, vale.

Se está proponiendo que la ciencia deje la explicación automática, esto
es, que sus productos finales dejen de ser autómatas, o sea, sistemas de
datos que predicen_{predicción} mecánicamente, para adoptar la
^|explicación problemática|, y la ciencia produzca, en consecuencia,
problemas, resoluciones y soluciones. Las soluciones podrán ser
formalmente indistinguibles de los productos finales que hoy produce la
ciencia, esto es, autómatas, ya sea en la forma de sistemas de
ecuaciones diferenciales o en cualquiera otra, pero, al enmarcarse
explícitamente en un ^{problema}, que en último término ha de ser el
^{problema de la supervivencia}, alcanzarán un ^{significado} que hoy no
tienen.


\Section Las teorías científicas

Esta es, desde luego, una manera de superar el enorme obstáculo que
^[Gödel] descubrió en los fundamentos de las ^{matemáticas}, y que toda
la ^{ciencia} comparte.

Porque ^[Gödel]^(G\"odel1931) demostró, con su teorema de
^{indecidibilidad}, que en cualquier teoría que incluya a la
^{aritmética} existen proposiciones indecidibles, esto es, proposiciones
que la propia teoría no puede determinar si son verdaderas o si son
falsas, o sea, paradojas_{paradoja}. El corolario a la ^{física} es
inmediato: la física incluye a la aritmética y, por lo tanto, no puede
ser completa. Este mismo argumento se puede aplicar inmediatamente a
todas las teorías que utilizan la aritmética, y se puede generalizar,
siguiendo a ^[Turing]^(Turing1936), véase ^>La paradoja reflexiva>, a
todas las que tienen que ser expresadas en simbolismos recursivos, de
modo que todas las ciencias quedan afectadas. Pero el caso de la física
es especialmente sorprendente porque, a pesar de la famosa demostración
matemática del año 1931, todavía hay físicos importantes, como
^[Hawking]^(Hawking1988), que piensan que puede existir una ^{teoría
unificada} que describa completamente el universo.

Por otra parte, para la ^{explicación problemática} el ^{conocimiento}
científico queda subordinado a la vida. El teorema de indecidibilidad de
^[Gödel] no es más que la confirmación de que las matemáticas y las
ciencias heredan necesariamente la naturaleza paradójica de la ^{vida} y
del ^{yo}, vistas en ^>La vida es paradójica>, y en ^>Yo soy
paradójico>. Así que ya sabemos que es la naturaleza paradójica del yo,
de la vida, del conocimiento y de la ciencia, la que sostiene su
carácter problemático y los mantiene siempre inconclusos e incompletos,
es decir, abiertos, libres y vivos.

Es ^{absurdo} creer que todo cuanto sucede es expresable simbólicamente y
que tal expresión puede agotar completamente el ^{fenómeno}, porque está
probado matemáticamente que es imposible.


\Section El sujeto es libre

Para la ^{explicación automática}, el conjunto de ecuaciones
diferenciales que constituye la teoría es todo cuanto puede decirse
científicamente de los fenómenos que dicha teoría describe. La
explicación automática sólo sirve para el ^{camino de entrada}, es
decir, sirve para explicar el ^{fenómeno} exterior al ^{yo}, pero es
incapaz de hacer del yo interior un fenómeno. Porque la libertad queda
completamente excluida de la explicación automática y, como
consecuencia, la explicación automática no puede explicar el yo, que es
libre, conclusión ya alcanzada en ^>¿Es libre el sujeto?>.

Entonces nos preguntábamos si tiene ^{libertad} un ^{mecanismo}. La
respuesta era que no, que siendo el mecanismo el prototipo del
determinismo no había de tenerla. Tampoco es libre el ^{adaptador},
razonábamos, porque es un mecanismo. Y el ^{aprendiz}, que es un
adaptador, tampoco puede ser libre. No puede tener libertad el
^{conocedor}, cuya virtud es ser capaz de imitar al aprendiz, al
adaptador y al mecanismo. Y el ^{sujeto}, nos preguntábamos, ¿puede ser
libre un sujeto? Según la explicación automática del camino de entrada,
vista en ^>El mecanismo>, la respuesta era que no, porque el sujeto
sigue siendo un mecanismo, concretamente aquel mecanismo con un sistema
nervioso en el que modela la realidad, realidad que puede utilizar de
varios modos en función de sus sentimientos, y que dispone de una lógica
simbólica en la que puede razonar.

Para la ^{explicación problemática} la situación es muy diferente. Es
cierto que un mecanismo no tiene libertad, pero ocurre que un mecanismo
vivo no es un todo, sino una parte. Un mecanismo vivo es un resolutor
del ^{problema aparente} de la supervivencia, y no tiene sentido por sí
mismo fuera del problema, o sea, que ha de ser definido como parte del
problema. De manera que, aunque el mecanismo vivo sea impermeable a la
libertad del problema aparente, que lo es, forma parte de un sistema que
necesariamente la incluye. Y, en este sentido, la secuencia que va del
mecanismo al sujeto es un proceso en el que la libertad del problema
aparente de la supervivencia va empapando a sus resolutores. Con el
sujeto este proceso alcanza su culminación, porque el sujeto es capaz de
interiorizar el propio problema y, con el problema, su libertad. El
sujeto se apodera de la libertad del problema. El sujeto es libre.

El sujeto es libre, sí, pero la libertad no es absoluta, está
condicionada, o, dicho de otra manera, la libertad es una parte del
problema. Esto significa, por un lado, que el ^{futuro} no está
determinado, sino abierto, porque depende de decisiones tomadas
libremente, y, por el otro lado, que no todas las decisiones son
igualmente buenas o malas, es decir, significa que las decisiones libres
del sujeto tienen consecuencias diversas y, según vimos en ^>El
conocimiento es provisional>, nunca enteramente
predecibles_{predicción}. Si fuera de cualquier otro modo, la vida sería
más fácil.


\Section La consciencia y la autoconsciencia

Para la explicación problemática, el ^{sujeto} no es un ^{mecanismo},
sencillamente, porque el sujeto es libre y el mecanismo no. Añadir la
^{libertad} hace fácil lo que, sin ella, es imposible. Otro tanto ocurre
con la consciencia y con la autoconsciencia.

En el camino de entrada, concretamente en ^>La consciencia>, y señalando
la distinción hecha entre sentir y ^{ver}, definimos la consciencia como
aquello que el sujeto ve del mundo. En esta misma línea, la
autoconsciencia sería cómo se ve el sujeto a sí mismo. Y como la teoría
de la subjetividad nos descubre que el sujeto es un resolutor del
problema aparente de la supervivencia, podemos precisar más las
definiciones.

La ^|autoconsciencia| es la facultad merced a la cual un ^{resolutor} se
representa a sí mismo. Y la ^|consciencia| es la facultad por la cual un
resolutor se representa la situación completa, es decir, el ^{problema}
al que se enfrenta y a sí mismo como resolutor. De estas definiciones se
sigue que sólo los sujetos pueden tener consciencia y autoconsciencia,
porque son los únicos resolutores que, por ser su ^{lógica simbólica},
pueden representar tanto problemas como resoluciones. Dicho de otro
modo, para ver una situación entera es precisa cierta distancia, y, si se
trata de la propia situación, entonces se necesita un espejo, o sea,
reflexión, para obtener la distancia.

\breakif6

Luego el sujeto es consciente cuando se ve como problema, esto es,
cuando el sujeto se ve como su yo y se sabe libre. Y el sujeto es
autoconsciente cuando repara en que su problema es irresoluble, porque
es paradójico y no tiene solución, o sea, cuando el sujeto se sabe
mortal.
$$\vbox{\halign{&\hfil#\hfil\cr
  Yo &sé que &soy &libre &y & mortal\cr
  Yo &&soy &consciente &y & autoconsciente\cr}}$$

Estas son conclusiones fáciles para la ^{explicación problemática}, pero
imposibles para la ^{explicación automática} o para la ^{explicación
material}.


\Section El conocimiento es acumulativo

El ^{sujeto} se representa el problema al que se enfrenta y a sí mismo.
El problema al que se enfrenta el sujeto es un ^{problema aparente}, es
decir, es un problema que no admite una representación definitiva. Por
lo tanto, la representación que se hace el sujeto del problema aparente,
representación que denominamos ^{problema del sujeto}, es cambiante. Al
cambiar el problema del sujeto, cambiarán consecuentemente su resolución
y su solución. Esto, traducido de la resolución a la evolución, como ya
hicimos en ^>El conocimiento es provisional>, significa que todo el
conocimiento es provisional.

Y como ningún conocimiento es seguro, tampoco es seguro que lo nuevo sea
definitivamente superior a lo anterior, lo que aconseja no olvidar.
Seguramente por esto el ^{conocimiento} humano es acumulativo, y sólo se
olvida con el desuso y con el ^{tiempo}, a veces. He aquí la
^{explicación problemática} de por qué no podemos ^{olvidar}, sobre
todo, cuando queremos olvidar.


\Section La intención

Pueden hacerse muchas otras comparaciones entre la ^{explicación
problemática} y las otras explicaciones. Es usted libre de probar, con
aquel asunto que más le interese, qué explicación le parece más atinada.
Aquí continuaremos investigando el asunto más significativo: el
significado.

El significado aparece con el ^{conocedor}, capaz de usar la realidad de
varios modos, para elegir, de entre todos ellos, el modo que más le
interesa. Le sirve, en definitiva, para poder utilizar la realidad
conforme a sus propios intereses o, dicho de otro modo, el
^{significado} es el medio que emplean los conocedores para integrar los
condicionantes externos con las necesidades internas; véase ^>El
significado>.

Dado que los sujetos saben que se enfrentan a un problema, tienen un
propósito consciente, solucionar el problema, que dota de ^{intención} a
todo cuanto hacen conscientemente. No hacen por hacer, sino que tienen
un fin marcado por el problema que saben que afrontan, y por esto su
^{comportamiento} consciente es intencional.

De modo que, sobre el conocedor con sus significados, la evolución
diseñó el ^{sujeto} con sus intenciones. Y si los significados le
servían al conocedor para utilizar la realidad según sus propios
intereses, también le servirán al sujeto para hacer según sus
intenciones. La diferencia es que mientras el sujeto ve_{ver} los
significados, o sea, es consciente de los significados, y, por esto,
puede utilizarlos con intención, el conocedor usa los significados, pero
no puede verlos. Las intenciones del sujeto son conscientes, es decir,
expresables, y los intereses del conocedor no.


\Section El homúnculo

Como señala ^[Searle]^(Searle1992), si se estudia la cognición del
^{hombre} rechazando el ^{significado} y la ^{intención}, resulta que
todos los cálculos realizados mecánicamente por el ^{sistema nervioso}
han de servir para que finalmente un ^{homúnculo} tome las decisiones
conscientes. Queda explicar la cognición de este homúnculo que, para
tomar decisiones conscientes, ha de utilizar significados que encaucen
sus intenciones. Pero entonces vale tanto explicar la cognición del
homúnculo como la del propio hombre, por lo que podemos ahorrarnos el
homúnculo que nada explica. Si está usted interesado en conocer otras
opiniones sobre esta dificultad, puede consultar, entre otros,
 ^[Dennett]^(Dennett1991),
 ^[Edelman]^(Edelman1992), o
 ^[Crick]^(Crick1994).

^[Searle] está en lo cierto y solamente caben dos soluciones. Una es
negar la ^{consciencia}, es decir, calificarla de ilusoria. Pero aquí
preferimos no rechazar ni el significado, ni la intención. Lo que sucede
es que, aunque no haya un homúnculo en el cerebro, hay un ^{espejo};
según vimos en ^>La consciencia>.


\Section La teoría de la subjetividad

Como vamos comprobando, en la explicación problemática podemos utilizar
lícitamente propósitos, fines, intenciones y significados. Justo lo
contrario es lo que ocurre con las explicaciones automáticas, que no
admiten razonamientos finales, o teleológicos_{teleología}, y que, en
consecuencia, no pueden albergar de modo alguno los significados. La
incoherencia de una explicación sin significado es patente, lo que
justifica que las explicaciones automáticas convivan malamente con las
explicaciones materiales que utilizan los significados naturales, esto
es, las cosas y sus significados.

Tomemos el ejemplo del ^{electrón}. Aunque la ortodoxia física nos
suministre como toda explicación la ecuación de onda que describe su
comportamiento, porque así se evitan las paradojas_{paradoja}, el mero
hecho de que el electrón sea descrito con su propia ecuación, descubre
que se le sigue tratando como una cosa o, más exactamente, siguiendo a
^[Bohm]\footnote{_(Bohm1951), página~118.}, como un objeto que no es una
partícula clásica ni una onda, sino que presenta, según las
circunstancias, las propiedades de una o de otra.

Forzando la situación, sólo la ecuación de onda del ^{universo} evita
completamente las paradojas. Pero, si la ^{física} hubiera de limitarse
a tal ecuación, sería completamente inútil, dada la imposibilidad de
plantearla y, mucho menos, de resolverla, aunque la solución esté,
literalmente, a la vista, porque basta abrir los ojos para que se
muestre en todo su esplendor.

La ^{teoría de la subjetividad} desenmaraña la situación introduciendo
al ^{sujeto}. Los hombres somos sujetos, pero no unos cualesquiera, sino
que los humanos somos los sujetos encontrados por la evolución entre los
primates. Y, como sujetos que somos, los significados_{significado} nos
sirven para amigar las condiciones exteriores con los apetitos
interiores. Además, por herencia genética, nuestros significados se
acoplan sin esfuerzo a las cosas_{cosa} y, con más dificultad, a los
conceptos_{concepto}. Por esta razón seguimos aferrándonos al electrón
como cosa. Necesitamos que las teorías se refieran a algún objeto para
poder colgarle significados, porque es así, y no de otro modo, como
comprendemos.

\breakif3

Lo que, al final, nos descubre la teoría de la subjetividad, es que las
ecuaciones dependen de cómo estamos construidos los sujetos, porque,
hasta el existir de los hipotéticos objetos que llamamos electrones,
depende del sujeto que les da significado. La ^|teoría de la
subjetividad| es una teoría subjetivista del sujeto.


\Section El conocimiento sintético a priori

Aunque, según la ^{teoría de la subjetividad}, el sujeto construye el
^{mundo}, no lo diseña sobre una hoja de papel en blanco. Veamos por
qué.

El ^{problema de la supervivencia}, que define la ^{vida}, es un
^{problema aparente}, o, dicho de otro modo, la vida no tuvo ninguna
^{información} inicial sobre lo que le convenía y lo que no; véase ^>El
problema de la supervivencia>. Pero ahora la vida es un conjunto de
seres vivos, y cada ser vivo es un resolutor del problema de la
supervivencia que ha recibido ^{información} de otros resolutores que lo
antecedieron.

De esta manera, el ^{aprendiz}, capaz de hacer modelos del ^{entorno},
cuenta con un ^{conocimiento} \latin{a priori} del exterior. Porque su
herencia ^{genética}, aquélla por la cual es un aprendiz y no otra cosa,
es también información sobre el exterior. Lo informa, en el más puro
sentido etimológico, de que la modelación del entorno exterior ayuda a
sobrevivir. Se muestra así que las ^{categorías} apriorísticas, que
^[Kant]\footnote{_(Kant1787), página 168 de la edición de 1787.}, por
anteceder a ^[Darwin], pensaba necesarias, son contingentes y fruto de
la ^{evolución} darviniana.

En nuestro propio caso, el ^{motor sintáctico} que nos hace
sujetos_{sujeto}, véase ^>La razón>, es apriorístico, o sea,
que los sujetos humanos lo heredamos genéticamente.

Para salvar al ^{empirismo} basta que la hoja en blanco de
^[Locke]^(Locke1690) se refiera al conocimiento que tenía la vida al
aparecer, y no al que tiene un hombre cuando nace. En nuestra jerga: el
problema de la supervivencia es un problema aparente, pero el ^{problema
del sujeto} no.


\Section El logicismo

Nos parece, como a ^[Locke], que somos capaces de pensar cualquier cosa,
por distante que sea de lo que experimentamos habitualmente. Pero
cuidado, ya que esto es engañoso por dos razones.
\beginpoints
\point No se puede ^{imaginar} lo que no se puede imaginar, y por lo
tanto, cualquiera que sea nuestra limitación imaginativa, pensaremos que
todo es imaginable, ya que, repito, no es posible imaginar lo
inimaginable. En términos lógicos, no podemos representarnos lo que no
es representable en nuestra ^{lógica interna}, como nos descubrió
^[Wittgenstein]^(Wittgenstein1922).

\point Nuestro simbolismo no trabaja directamente sobre los datos
captados por nuestros sentidos, sino sobre datos elaborados en forma de
cosas_{cosa} con ^{significado}. Es decir, que con respecto a los datos
sobre los que trabaja el aparato simbólico, las cosas con significado,
nuestro conocimiento está genéticamente codificado.

\noindent Desarrollaremos el segundo punto en la sección siguiente,
_>El objetivismo>, y en ésta el primero.
\endpoints

Al plantear el ^{problema aparente} en el ^{álgebra automática}, véase
^>El problema aparente formalizado>, escribimos que el {\universo} podía
ser cualquiera, $\forall\aut U$. Con esta proclamación expresábamos que
en el problema aparente no se tiene, \latin{a priori}, ^{información}
alguna sobre lo externo. Pero aunque lo proclamemos, nuestra propia
limitación impone que seamos incapaces de considerar aquellos universos
que somos incapaces de imaginar, ¿cómo podría ser de otro modo? Esto
sucede, también, en el planteamiento formal del problema aparente, ya
que el {\universo} no puede ser cualquier cosa, sino que, dado que
empleamos el álgebra automática como lógica, ha de ser un autómata
finito, $\forall\aut U \in \{ \aut A \}$.

Por falta de imaginación aportamos información, incluso, cuando
pretendemos no aportarla. Esto es lo que llamo ^|logicismo|; véase ^>El
problema del aprendiz>. Cualquier formalización del problema aparente
peca necesariamente de logicismo; véase \EPA{4.5}, en donde lo
denominaba ^{esencialismo}.

Llamando ^{lógica externa} a la lógica en la que se representa un
problema aparente, podemos extraer algunas consecuencias del logicismo.
La lógica externa ha de ser tal que permita representar el propio
problema aparente y, también, sus posibles resoluciones y soluciones,
véase ^>El álgebra automática>, que son exactamente las mismas
condiciones que se requieren de la lógica interna del sujeto, véase ^>El
sujeto simbólico>, y ^>La razón>. Esto implica que la lógica externa de
un determinado problema aparente siempre vale como lógica interna de un
sujeto de ese mismo problema, y viceversa, de modo que ambas lógicas
pueden ser iguales. Así que, cuando un sujeto se plantea su propia
situación problemática, lo mejor que puede hacer es postular, como
lógica externa, su propia lógica interna, puesto que no puede imaginarse
otra más expresiva. A propósito, la demostración de que la
\Mental sintaxis recursiva$L_{\syn U}$ es la más expresiva, vista en
^>La expresividad>, nos descubre que ésa es precisamente nuestra
sintaxis, conclusión conocida como tesis de ^[Church]-^[Turing], véase
^[Hofstadter]^(Hofstadter1979).


\Section El objetivismo

El ^{logicismo} no es, en principio, ni bueno ni malo; es lo que ocurre
y, además, es inevitable. Lo que sí es pernicioso es no percatarse de su
influencia. El ^|objetivismo| es la forma que toma este error entre
quienes no entienden que la ^{realidad} objetiva y semántica es la
manera específica en la que nosotros vemos_{ver} lo externo, sino que
piensan que es así; creen que las cosas que vemos son como las vemos.

Por ejemplo, la falta de imaginación nos induce a pensar que las
cosas_{cosa} existen_{existencia} por sí mismas, es decir, que la
realidad de las cosas es exterior e independiente del sujeto. Y aunque
las repercusiones prácticas del objetivismo pueden ser despreciables, la
consecuencia filosófica de esta falacia es anteponer la ^{ontología} a
la ^{epistemología}. Aquí preferimos hacer de la teoría del conocimiento
la base de la filosofía y desvincular a la ontología de la ^{filosofía}
para adscribirla a la ^{psicología}. Otro ejemplo son las llamadas
paradojas_{paradoja} cuánticas, como la dualidad onda-corpúsculo del
^{electrón}, que igualmente se deben a la necesidad que tenemos las
personas, para entender lo externo, de cosificarlo, convirtiéndolo en
objetos con significado. Ya pensaba ^[Bohr]^(Bohr1929) que sería preciso
renunciar a la representación intuitiva de los fenómenos atómicos.

La necesidad de cosificar_{cosificación} para entender es una necesidad
de los hombres, pero no de los sujetos en general, y tiene su raíz en la
^{percepción}, el ^{aprendizaje} y la ^{emoción}, que construyen las
cosas como objetos con ^{significado}. Retomando la analogía de la
percepción con unas gafas que añaden las etiquetas_{etiqueta}, dejada en
^>La realidad objetiva es subjetiva>, podemos argumentar que, para
recomponer lo que hay más allá de las gafas, es preciso contrarrestar el
efecto distorsionante de las lentes. Es decir, que sabiendo que la
percepción, el aprendizaje y la emoción son las causas de la distorsión,
estamos en disposición de reinterpretar las paradojas cuánticas. La
teoría de la subjetividad nos ofrece esta posibilidad, que supone
supeditar la ^{física} a la ^{psicología}.


\Section Yo y la realidad

Toda expresión sintáctica necesita de un ^{sujeto} para tener
^{significado}; aún más, sólo los sujetos emplean expresiones
sintácticas. Luego no tiene sentido hablar de verdades absolutas, que
habrían de ser expresiones sintácticas cuyo significado no dependiese de
ningún sujeto, ni de explicaciones objetivas, que serían explicaciones
fundadas sobre verdades absolutas. La ^{verdad} y la ^{explicación} son
necesariamente subjetivas.

Lo inmediato es lo subjetivo. Lo objetivo es una construcción. Luego no
es posible explicar lo subjetivo desde lo objetivo, sino que lo
subjetivo explica lo objetivo. Pero el sujeto no se ve a sí mismo como
sujeto, sino como ^{yo}. Se ve libre para resolver problemas y, por lo
tanto, se ve habitante de la ^{sintaxis}. Yo estoy en la sintaxis. Por
esto la realidad semántica se le aparece al sujeto como dada e
independiente de sí, cuando depende del sujeto hasta el punto de que es
construida por él mismo.

Por fin, tres resúmenes que desmontan el ^{objetivismo}:
\beginpoints
\point Es cierto que yo estoy fuera de la ^{realidad} de las
cosas, pero también que yo y las cosas reales estamos en el sujeto;
 véase ^>La cosa y el concepto>.
\point En el ^{mundo} del sujeto cabe la realidad y quepo yo;
 véase ^>El mundo>.
\point Yo y la realidad estamos en capas distintas del mismo sujeto;
 véase ^>Las capas>.
\endpoints

\breakif6

\Section Hay significados erróneos

En el caso de la teoría física que sirve para predecir_{predicción} el
comportamiento del ^{electrón}, ni la teoría por sí misma, ni el
electrón, que es un ^{concepto} y no una ^{cosa}, tienen ^{significado}.
El significado se lo da el ^{sujeto} que elabora o interpreta esa
teoría, y no otra, porque es la que le permite construir los
dispositivos electrónicos que, en último término, le ayudan a vivir.
Todos los significados tienen su origen en el ^{problema aparente} de la
supervivencia, y por esto no hay significado fuera del bucle subjetivo;
visto en ^>El bucle subjetivo>.

Pero las personas también hacemos por el simple gusto de hacer, incluso,
a veces, con riesgo de la propia vida. Olvidemos las actividades
arriesgadas porque, a pesar de su vistosidad, no pueden ser tan
peligrosas si nos atenemos a sus exiguas estadísticas de mortalidad. Aun
así, es cierto que podemos contemplar una puesta de sol, o ^{cantar},
por el simple gusto de hacerlo, y no por eso consideramos que sean
actividades sin sentido o sin significado. Tengo por seguro que, a pesar
de su apariencia, también estas actividades, incluso las peligrosas,
contribuyen a la mera supervivencia. Nuestra historia evolutiva nos ha
construido de un determinado modo, y estamos preparados para soportar
cierto rango de ^{relajación} y de ^{tensión} que nos han servido para
sobrevivir como especie. Cuando nuestro ^{entorno} no nos proporciona
las dosis adecuadas de relajación o de tensión, hemos de obtenerlas por
otros medios.

Y es que, aunque no podamos justificar racionalmente por qué nos gusta
la ^{música}, si nos resulta placentera, entonces tiene significado.
Ocurre lo mismo con un ^{dolor} desproporcionado al peligro que señala.
Si es un dolor, por definición, tiene significado; véase ^>El
sentimiento>. Puede ocurrir que, por un azar de nuestra historia
evolutiva, un determinado dolor sea ahora contraproducente, porque su
intensidad puede forzarnos a atenderlo a él y, en consecuencia,
desatendamos otros dolores menos molestos pero de más peligro.

Esto es un ejemplo, no más, de que también la ^{información} genética es
provisional, como vimos en ^>El conocimiento es provisional>, pero lo
interesante del caso es que sólo un sujeto, al relacionar todos estos
datos con el ^{problema de la supervivencia}, puede concluir que,
incluso los significados naturales más primitivos, pueden ser erróneos.
Para un conocedor no hay significados erróneos, ni puede haberlos,
porque, aunque usa significados, no puede verlos_{ver} como el sujeto,
ni, en consecuencia, plantearse si son adecuados o no.

\breakif1

Es ciertamente interesante saber que los significados naturales pueden
ser erróneos, pero complica enormemente la situación. Porque falla la
simple regla que establece que todo significado natural es
necesariamente correcto. Las personas somos unos sujetos
azarosamente_{azar} diseñados y, por esta razón, las cosas no son como
se nos aparecen, como adelantamos en ^>¡Abajo el objetivismo!>.


\Section El problema es la fuente del significado

Los objetivistas dicen que la ^{piedra} es una cosa que
existe_{existencia} por sí misma y que, por lo tanto, no requiere
explicación alguna, sino que, simplemente, la piedra es. Por otro lado,
de ^{Dios}, y de ^{patria}, y de cualquier otro concepto, exigen una
explicación para aceptarlos como existentes; véase ^>La explicación
material>. Para el objetivismo la ^{cosa} tiene ^{significado}, y al
^{concepto} hay que buscárselo.

Para nosotros la diferencia entre la piedra y la patria consiste en que
son objetos que provienen de fases distintas de la ^{evolución}, de modo
que son imágenes o representaciones que se encuentran en capas
diferentes del ^{mundo}; véase ^>El mundo>, ^>Las capas>, y ^>Yo y la
realidad>. Pero, como ambos son objetos, y en esto no difieren, ambos
han de ser explicados, aunque cada uno resulte tener una explicación
diferente, y solamente por esto distinguimos las cosas, como la piedra,
de los conceptos, como la patria. Que la piedra sea un objeto construido
por la ^{percepción}, el ^{aprendizaje} y la ^{emoción}, pero al margen
de nuestra ^{voluntad}, y que la patria sea un objeto teórico y
voluntario, es circunstancial y no altera fundamentalmente la situación.

Tanto a las cosas como a los conceptos hay que buscarles el significado,
que sigue siendo la manera de casar las condiciones externas con las
condiciones internas al resolutor del problema de la supervivencia;
véanse ^>El significado>, y ^>La inteligencia>. Y como los significados
naturales de las cosas pueden ser erróneos, tal como mostramos en la
sección anterior, _>Hay significados erróneos>, los significados
definitivos han de obtenerse en última instancia del problema de la
supervivencia, que permite calibrar absolutamente la importancia de las
distintas condiciones, porque el objetivo final de todos sus resolutores
es, precisamente, su solución.
$$\hbox{Problema}
   \llave{Libertad\cr
    Condición$\,
     \left\{ \vcenter{\nointerlineskip\halign{#\hfil\crcr
      Interna\cr
      $\quad\;\bigm\updownarrow\hbox{Significado}$\cr
      Externa\cr
     }}\right.$
    \cr}
$$

\breakif6


\Section Los límites del saber

El ^{significado} amiga los deseos_{deseo} con la ^{percepción} de los
resolutores mentales. Los resolutores mentales, que son tanto los
conocedores como los sujetos, tienen en cuenta la ^{geometría}
espacial_{espacio} de las condiciones del problema que afrontan, esto
es, distinguen si las condiciones son internas o externas. Puede
entonces decirse que el significado es la geometría que adopta el
sistema formado por el problema y su resolutor; véase ^>El tiempo y el
espacio>. No hay significado sin problema. En nuestro propio caso, el
problema aparente de la supervivencia es la fuente primigenia de los
significados. La ^{vida} y la ^{muerte} son los límites del significado,
no nos es posible ir más allá.

Esta es una consecuencia de nuestra teoría ^{semántica} de la
subjetividad, conforme a la cual, y como hemos visto en ^>Hay
significados erróneos>, la ecuación de onda que describe al ^{electrón}
según la ciencia ^{física} tiene significado porque, en último término,
nos permite construir dispositivos electrónicos que facilitan nuestra
vida.

No es posible explicar el sentido de la vida, sino que es la vida la que
nos da sentido. Las explicaciones pueden extenderse hasta alcanzar la
supervivencia, pero no más lejos. No tiene sentido ninguna teoría
metatanática, esto es, que trascienda la muerte, o la vida.


\Section Los límites de la comunicación

Si la ^|comunicación| es la transferencia de significados_{significado},
entonces la ^{teoría de la subjetividad} también establece límites a la
comunicación: dado que el ^{problema} aporta los significados, solamente
puede haber comunicación entre los resolutores del mismo problema. Este
resultado tiene varias consecuencias, ya que, al poner límites, queda
necesariamente definido un dentro y un fuera, un posible y un imposible.

Como, en principio, los resolutores del mismo problema pueden compartir
significados, y como todos los seres vivos_{vida} somos resolutores del
problema de la supervivencia, ocurre que los seres vivos podemos
comunicarnos. Por esta razón entendemos a las plantas_{planta} que, aun
siendo meros mecanismos, buscan la luz del ^{sol}. También podríamos
interpretar que la ^{luna} busca la trayectoria más cómoda para girar
alrededor de la ^{tierra}, pero este modo de hablar, que da ánimo a lo
que no está vivo, es siempre figurado y no funciona bien, porque
mientras que la planta morirá si no encuentra la ansiada luz, la luna
está exenta de tales contingencias.

\breakif1

Y, por contra, como solamente pueden compartir significados los
resolutores del mismo problema, ocurre que no podemos comunicarnos con
nada que no esté vivo. La búsqueda de ^{inteligencia} extraterrena puede
toparse, por esta causa, con una dificultad insalvable. Porque, para que
podamos dar significado a una regularidad, ésta ha de considerarse una
regularidad simbólica, y no meramente física. Y, para que una
regularidad se considere simbólica, ha de suponerse que, bajo la
sintaxis manifiesta, subyace una ^{intención} semántica. Me explico
mejor con un ejemplo: la regularidad de los cristales de ^{cuarzo} puede
interpretarse como el resultado de la resolución de un complicado
problema tridimensional de minimización de energía, pero, como el
problema resuelto no es el de la supervivencia, no le atribuimos una
intención y, como consecuencia, no podemos comunicarnos con las rocas.

Otro asunto sería el de un ^{robot} construido por un ingeniero, que
aunque no estuviera vivo según la definición tradicional de vida, ligada
a la química orgánica del carbono, en cambio, si se diseñase con el
propósito de sobrevivir, entonces sí que formaría parte de la vida
definida a la manera de esta teoría de la subjetividad, o sea, definida
como problema aparente; véase ^>El problema de la supervivencia>. Este
robot podría ser un ^{adaptador}, como lo es el ^{termostato} que se
diseña con el propósito de mantener la ^{temperatura}, pero entonces no
podría ser inteligente y la comunicación con él sería pobre. Muy
distinto sería el caso de un robot ^{sujeto}, que sí podrá ser más
inteligente y racional que una persona; asunto al que volveremos en _>El
sucesor del hombre>.


\Section La burbuja semántica

Todos los seres vivos_{vida}, por compartir un mismo ^{problema},
constituimos un universo semántico o, dicho más humildemente, una
^{burbuja semántica}. Esto quiere decir que, por un lado, no podemos
comunicarnos con lo que no está vivo y, por el otro, que es posible, en
principio, la ^{comunicación} entre todos los seres vivos. Pero,
mientras los conocedores pueden cambiar los significados de los objetos
y los sujetos podemos, además, ^{ver} los significados, los aprendices y
los adaptadores no distinguen los objetos de los significados, y los
mecanismos ni siquiera usan objetos. De manera que la comunicación más
rica, y la única capaz de transmitir problemas y resoluciones sin
restricción, es la comunicación simbólica entre sujetos.


\Section La colonia mental

El hombre, \latin{homo sapiens}, es genéticamente muy semejante al
^{chimpancé}, \latin{pan troglodytes}, y, sin embargo, había en 1980,
según ^[Ayala]^(Ayala1980), unos cien mil chimpancés y unos cuatro mil
quinientos millones de hombres, es decir, cuarenta y cinco mil por
cada chimpancé. Sucede, pues, que una pequeña diferencia ^{genética}
provoca una enorme ventaja evolutiva, como ya adelantamos en ^>Una
diferencia pequeña>.

Sostenemos que la ^{simbolización} marca dicha ventaja, y que es la
característica más recientemente adquirida. Si aceptamos que la
capacidad de simbolización es la más recientemente adquirida, resulta
que las otras características cognitivas las compartimos con otras
especies animales. Así que, por ejemplo, la ^{percepción} que hace
objetos_{objeto} y el ^{sistema emocional} que hace
significados_{significado} son empleados también por otras especies
animales.

La simbolización hace posible el lenguaje con ^{sintaxis recursiva}, que
es único entre los hombres. La simbolización permite que las
asociaciones humanas superen el millón de miembros y lo convierte en el
único mamífero que, como las hormigas_{hormiga}, forma
colonias_{colonia}. Porque el lenguaje con sintaxis recursiva, que nos
hace capaces de comunicar y compartir problemas, resoluciones y
soluciones, nos permite alcanzar una ^{especialización} mental
equivalente a la que las hormigas consiguen en la capa corporal; véase
^>Las capas>. Es curioso que, según ^[Hölldobler] y
^[Wilson]^(H\"olldobler1994), el peso de todas las hormigas coincida
aproximadamente con el peso de todos los humanos.

La simbolización nos distingue como especie y nos aventaja con respecto
a las demás. Que corresponda al último paso evolutivo explica que sea
única. Que permita la formación de colonias explica su ventaja para
sobrevivir.


\Section La cultura

El ^{simbolismo} permite establecer cualquier ^{convención} sintáctica,
y conviene la convención que mejor haga al caso en cada momento. Porque
siendo la sintaxis recursiva_{recursividad}, la maquinaria sintáctica es
completamente flexible y, cualquiera que sea la modelación sintáctica
precisa, es posible definirla, como vimos en ^>La máquina universal de
Turing>. En particular, se puede partir de una sintaxis establecida y
extenderla_{extensibilidad} para cubrir otros propósitos, como es el
caso de la ^{jerga} matemática, de la científica, o de cualquiera otra.
De manera que la sintaxis es convencional, esto es, los objetos
sintácticos pueden tener cualquier significado, o no tenerlo, y pueden
referirse a cualquier objeto, que también puede ser sintáctico, de modo
que una expresión sintáctica puede, incluso, referirse a sí misma.

Todo esto era sabido, pero no parecía que la diferencia esencial entre
nuestra especie y las demás fuera que somos convencionales. Y, sin
embargo, lo es. La razón, ya lo sabemos, es que, con la ^{sintaxis
recursiva}, el simbolismo permite la representación y la expresión de
problemas, resoluciones y soluciones. Esto es crucial porque los seres
vivos somos resolutores del ^{problema aparente} de la supervivencia, de
modo que los únicos seres vivos capaces de representarse la situación
tal cual es, incluidos ellos mismos, son aquéllos cuya lógica es
simbólica, esto es, somos nosotros los sujetos. Sólo los sujetos somos
conscientes de la situación problemática en la que vivimos.

Dicho de otra manera, los hombres podemos expresar y comunicar una parte
de nuestros procesos cognitivos, parte en la que está incluida nuestra
^{realidad} y también nuestros problemas y nuestras resoluciones. Puede
decirse que nuestro ^{pensamiento} consciente es transparente o, más
exactamente, transparentable_{transparentabilidad}. La consecuencia es
que unos hombres pueden aprovecharse del conocimiento y del saber de
otros, vivos o muertos, de tal modo que las resoluciones encontradas por
uno, si son beneficiosas, pueden ser empleadas por cualquier otro
hombre. Este proceso de ^{cognición} compartida, o distribuida, que se
conoce con el nombre de ^|cultura|, es el que permite formar
colonias_{colonia} humanas de millones de miembros y el que, en
definitiva, ha procurado al \latin{homo sapiens} un desarrollo sin
precedentes en la ^{evolución} durante los últimos treinta mil años;
véase ^[Harris]^(Harris1989). Puede que parezca mucho tiempo, pero no lo
es comparado con los más de tres mil millones de años que han pasado
desde la aparición de la ^{vida}. Es una explosión que merece más
explicaciones.


\Section La evolución darviniana

La primera estrategia empleada por la ^{vida} para resolver el
^{problema aparente} ha consistido en aplicar el proceso de reproducción
imperfecta y ^{selección} natural descubierto por
^[Darwin]^(Darwin1859); véase ^>El problema aparente>. Este es un
proceso a dos niveles: el nivel superior produce los resolutores y el
inferior determina la solución, en forma de comportamiento, a aplicar;
véase ^>La doble resolución>. Ahora sabemos que ésta es la forma de
resolver los problemas que emplean los conocedores; véase ^>El conocedor
formal>. Además, como elige los resolutores por prueba y error, la
evolución darviniana funciona como un conocedor que tantea_{tantear}.
Por esto, la ^{inteligencia darviniana} es tentativa, y no es semántica.

Mientras la ^{evolución} produjo mecanismos, adaptadores y aprendices,
mantuvo los dos niveles, pero cuando la evolución darviniana, que
funciona ella misma como un conocedor que tantea, comenzó a generar
conocedores con ^{inteligencia} semántica, el proceso adquirió un nivel
adicional. Porque la dotación genética, al construir un ^{conocedor}, ya
no determina completamente la resolución a emplear, que puede ser
cualquiera de las resoluciones de las que es capaz la ^{mente} del
conocedor construido. Es decir, en la determinación de la resolución
actualmente aplicada interviene la inteligencia darviniana, que
selecciona un conocedor, y la inteligencia semántica del conocedor
seleccionado, que elige una resolución.

La inteligencia semántica del conocedor ha de considerar tanto la
situación externa como la interna, y, dentro de ésta, tanto la corporal
como la mental. Con esta consideración de la propia situación mental, y
recordemos que la mente es el resolutor múltiple del conocedor, el
conocedor se entromete por primera vez en la resolución del ^{problema
de la supervivencia} que, hasta la aparición de los conocedores, era del
dominio exclusivo de la evolución darviniana. Y, para realizar su parte
de la resolución del problema aparente, el conocedor ha de tratar con
deseos_{deseo} y sentimientos, o sea, con significados_{significado}.


\Section La evolución cognitiva

Con los conocedores, la tarea de determinar qué resolución emplear queda
dividida entre la ^{inteligencia} de la ^{evolución darviniana}, que
selecciona al conocedor, y la inteligencia propia del ^{conocedor}
seleccionado, pero ninguna de estas dos inteligencias es capaz de
prever_{previsión} el efecto de las resoluciones, es decir, ninguna de
ellas es racional_{razón}. Cuando la evolución darviniana comienza a
generar sujetos, que en su lógica simbólica con ^{sintaxis recursiva}
pueden representarse problemas, resoluciones y soluciones, véase ^>El
algoritmo>, la propia evolución sufre un cambio cualitativo. Porque el
^{sujeto}, aun siendo resultado del proceso de evolución darviniana, lo
supera.

El sujeto interioriza completamente el proceso de resolución del
^{problema de la supervivencia}, según vimos en ^>Los niveles>. Al
hacerlo, el sujeto puede plantearse en apenas unos instantes distintas
resoluciones que la evolución darviniana tardaría generaciones en
intentar. Con los sujetos el proceso de evolución se acelera o, como
decíamos, explota porque se hace innecesaria la construcción física del
resolutor. Es por esto por lo que distinguimos la ^{evolución física}, o
darviniana, de la ^{evolución cognitiva} que solamente acaece en la
razón de los sujetos.

Y así como los distintos resolutores del problema de la supervivencia
encontrados por la evolución darviniana son físicamente distintos, las
resoluciones de la evolución cognitiva son representaciones en la
sintaxis recursiva de la lógica simbólica del sujeto y no pueden
distinguirse física o perceptiblemente. No pueden distinguirse porque
las resoluciones del sujeto son expresiones sintácticas y, como tales,
son convenciones que, por sí mismas, no tienen ^{significado}.

Así que tenemos, por un lado, la ^{evolución darviniana}, o física, que
funciona como un conocedor que tantea_{tantear} porque es capaz de
diversas resoluciones que selecciona por un procedimiento de prueba y
error, y, por el otro lado, la evolución cognitiva, que se vale de
sujetos que razonan porque pueden representarse varias posibles
resoluciones y sus consecuencias y, de este modo, pueden prever los
resultados de las distintas maneras de resolver. La evolución darviniana
es inteligente, pero tentativa, y la evolución cognitiva es racional.


\Section La técnica

El hombre pudo aclimatarse instantáneamente al frío. No fue necesario
que los individuos más peludos sobrevivieran mejor y dejaran una
descendencia mayor para que, al cabo de muchas generaciones, la
población fuera mayormente peluda. Bastó utilizar la piel de otros
animales ya aclimatados para vestirse_{vestido} y abrigarse. Vestirse
para abrigarse parece sencillo, pero ninguna especie animal se viste,
excepto nosotros. La razón es que somos los únicos sujetos vivos; pero
detengámonos para apreciar mejor los detalles.

Vale decir que solamente los sujetos extendemos la ^{evolución física}
con una ^{evolución cognitiva} que, en este caso, en vez de dotarnos de
pelo, nos viste. Esta explicación, que ya es muy general, se puede
generalizar considerando que todas las herramientas_{herramienta}, los
utensilios y los artefactos que fabricamos son, como la vestimenta,
^{prótesis} producidas por la evolución cognitiva que completan nuestro
cuerpo físico.

La explicación anterior no aclara por qué la evolución cognitiva nos
permite a los sujetos fabricar herramientas. Es porque las herramientas
son resoluciones hechas cosas, de modo que, para poder fabricarlas, hay
que imaginarlas, esto es, es menester representárselas internamente, y
solamente los sujetos disponemos de una lógica que permite la
representación de resoluciones; como ya vimos en ^>La herramienta>.
Fabricamos útiles porque nuestra lógica es simbólica.

La ^{lógica simbólica}, con la que interiorizamos la resolución de los
problemas, es la responsable de la ^|técnica|, que definimos como la
disposición física y, sobre todo, mental que nos permite fabricar
herramientas. Las herramientas, por ser resoluciones hechas cosas, son
los rasgos perceptibles, o físicos, de la evolución cognitiva. Estas
mismas herramientas son, también, los aspectos más visibles de la
cultura, porque la ^{cultura} es la transmisión de resoluciones entre
sujetos usando lenguajes simbólicos. Por esto es adecuado referirse a la
evolución cognitiva como ^{evolución técnica} o ^{evolución cultural}.


\Section El control del entorno

Construir artefactos puede verse, o bien como la fabricación de
^{prótesis} que extienden el cuerpo, o bien como la modificación del
^{entorno} para acomodarlo al cuerpo. Si la confección del ^{vestido} se
ajusta mejor a la primera perspectiva, la construcción de una ^{casa}
parece mejor descrita de la segunda manera. En último término se trata
de dos formas de encarar el mismo hecho: la ^{evolución cognitiva} actúa
fuera del ^{cuerpo}.

De hecho, hasta ahora la evolución cognitiva sólo actuaba fuera del
cuerpo. O, al menos, solamente había actuado indirectamente sobre los
cuerpos mediante la ^{selección artificial} de animales domésticos y de
plantas agrícolas, que ^[Darwin]^(Darwin1859) empleó como primer
argumento de su teoría. Hoy ya no es así.

Los sujetos podemos ^{ver} la situación completa porque podemos
representarnos el problema de la supervivencia y su resolución, que es
la vida en la que estamos incluidos nosotros mismos. Por esto, los
sujetos tenemos ^{consciencia} de nuestra posición dentro del todo. Y,
también por esto, la evolución cognitiva supera e incluye a la
^{evolución física}. En palabras llanas: el hombre puede intervenir en
los procesos de la evolución darviniana, tanto en el de selección como
en el de reproducción genética, y modificarlos.


\Section El único sujeto vivo

Una ^{sintaxis recursiva} es un sistema extensible de convenciones que
sirve para resolver problemas porque permite la expresión de problemas,
resoluciones y soluciones. Por ser convencionales, los símbolos
sintácticos están vacíos de ^{significado} y son los problemas los que
aportan los significados. El ^{simbolismo}, con semántica y sintaxis
recursiva, fue diseñado por la ^{evolución} darviniana porque la ^{vida}
es un ^{problema aparente}. El ^{sujeto} es el resolutor del problema
aparente de la supervivencia cuya lógica es simbólica. Y el único sujeto
vivo es el ^{hombre}. Por ahora.


\Section El sucesor del hombre

La resolución teórica del ^{problema aparente} nos descubre que la
^{evolución física}, o darviniana, y la ^{evolución cognitiva}, o
cultural, son dos etapas del mismo proceso; véase ^[Elias]^(Elias1989).
Y que, mientras el funcionamiento de la evolución darviniana es
oportunista y tentativo, esto es, utiliza el método de tanteo, también
llamado de prueba y error, propio de los conocedores que
tantean_{tantear}, la evolución cognitiva, en cambio, es simbólica y
razonada, como corresponde a los sujetos, y no es oportunista sino
finalista o teleológica_{teleología}.

Un ejemplo del ^{oportunismo} de la evolución darviniana es la aparición
del ^{sistema nervioso} que permitió el paso del ^{mecanismo}, capaz de
un único ^{comportamiento}, al ^{adaptador}, capaz de varios. Porque la
causa que provocó su aparición no fue, posiblemente, ésa, sino que el
sistema nervioso permitía transmitir datos a mayor distancia, y de ese
modo pudieron construirse organismos con mayores cuerpos y, aun así, con
un comportamiento unitario y coordinado.

Esto sugiere que el hombre puede mejorar el diseño oportunista del
proceso darviniano y que el sucesor del \latin{homo sapiens} será un
producto de la ^{ingeniería} genética diseñado por él mismo, aunque no
necesariamente basado en la ^{química orgánica}. Cabe, por supuesto, la
posibilidad de que el \latin{homo sapiens} se extinga_{extinción} sin
descendencia. O, aun peor, que el propio éxito de nuestra especie, o de
su sucesora, resulte en ^{plaga} y termine con la ^{vida} toda.

\breakif4


\Section La ética

El único resolutor del ^{problema aparente} que puede ser
autoconsciente_{autoconsciencia} es un sujeto, porque sólo un sujeto
puede representarse, en su razón simbólica, resolutores. Así que sólo el
sujeto puede representarse a sí mismo. Y, también por disponer de una
^{lógica simbólica}, el sujeto es el único resolutor capaz de
representarse el problema al que se enfrenta. Sólo un sujeto puede ser
consciente_{consciencia} porque sólo un sujeto puede ^{ver}
completamente la situación en la que se encuentra, incluido el problema
de la supervivencia y él mismo; según hemos visto en ^>La consciencia y
la autoconsciencia>.

Al representarse el problema que afronta, el ^{sujeto} consciente llega
al origen de los significados_{significado} y, al representarse a sí
mismo, el sujeto autoconsciente alcanza su propio sentido como
resolutor. En consecuencia, todo sujeto consciente y autoconsciente es
responsable de usar su libertad conforme a su propio sentido. Ésta es la
responsabilidad ^|ética| del sujeto, que se sabe libre y mortal, o sea,
vivo. Llamamos ^|persona| al sujeto con responsabilidad ética.

De estas definiciones se siguen varias consecuencias. El problema ético,
¿qué se debe hacer?, coincide con el ^{problema del sujeto}, ¿qué hacer
para no morir?, ya que ambos apuntan al ^{problema de la supervivencia},
de modo que la ^{epistemología} y la ética son uno. No es, pues, casual
que el desarrollo de la evolución resolutiva alcance a hacer al sujeto
responsable del futuro. Es decir, que el \latin{homo sapiens}, por ser
el único sujeto vivo, es la consciencia de la vida y está en sus manos
su futuro y el de toda la vida. ¡Enorme responsabilidad la del hombre!


\Section La ética y la epistemología son uno

La ^{ética} y la ^{epistemología} son uno. Sorprende que mientras
^[Sócrates] coincide con este resultado en su diálogo con
^<Protágoras>^(Platón-IV), en el que iguala
 virtud
  (areté, $\grave\alpha\rho\varepsilon\tau\acute\eta$) y
 conocimiento
  (episteme, $\grave\varepsilon\pi\iota\sigma\tau\acute\eta\mu\eta$),
en cambio ^[Kant] hubo de escribir su ^<Crítica de la razón
práctica>^(Kant1788) porque no fue capaz de incluir a la ética en la
^<Crítica de la razón pura>^(Kant1787). La causa de este fracaso se debe
a que ^[Kant] tomó como paradigma del conocimiento la ^{física} de
^[Newton], que aquí hemos clasificado como ^{explicación material}, y
que excluye a la ^{libertad}; véase ^>La explicación de entrada>.
Sorprende porque la ^{revolución copernicana} de ^[Kant] enderezó el
rumbo que la epistemología había extraviado por seguir a la ^{ontología}
según el dictado de ^[Sócrates]. Irónicamente, para ^[Sócrates] lo
primero, aun antes que la ontología, era la ética, de manera que su
propósito era soslayar la epistemología subjetivista de los sofistas,
que juzgaba de éticamente nociva.


\Section El adoctrinamiento

El control del ^{entorno} incluye, también, el control de otros seres
vivos, aunque el grado de manipulación depende del tipo de resolutor que
se pretenda dominar. Si se trata de controlar un mecanismo, capaz de un
único comportamiento, entonces la única posibilidad es aprovechar
ese comportamiento o no aprovecharlo; podemos, si queremos, cultivar
trigo. En el caso de los adaptadores, con varios comportamientos, si nos
interesa uno de ellos, podemos provocarlo interfiriendo en su
percepción; podemos hacer volar a una mosca que está posada acercando
nuestra mano. Pero, dada la rigidez genética de sus comportamientos, no
es posible amaestrar ni mecanismos, ni adaptadores.

Para ^{amaestrar} animales es preciso que sean aprendices, y sólo si es
posible modelar su ^{realidad} conforme a nuestros intereses. Así, por
ejemplo, pudo ^[Lorenz]^(Lorenz1949) ser {\em realmente} la madre para
unos gansos_{ganso}.

La domesticación_{domesticar} de animales es posible si éstos son
conocedores, porque se puede influir en su asignación de significados.
Se puede conseguir que un perro dé al sonido de una campana el
^{significado} \meaning{comida}, o que traiga las ^{zapatillas}.

Pero el mayor control posible es el que se puede alcanzar sobre un
^{sujeto}. El máximo dominio se obtiene amañando el problema del sujeto,
que es su propio ^{yo}, porque un sujeto así adoctrinado_{adoctrinar}
utilizará toda su potencia de resolución, y toda su ^{libertad}, y todo
su yo, para lograr su fin. Incluso literalmente, porque un sujeto puede
suicidarse conscientemente si determina que el ^{suicidio} es la
solución.


\Section El suicidio

Un ^{conocedor} puede matarse si asigna fatalmente un significado
inadecuado a un signo mortal, como vimos en ^>El signo arbitrario>. Pero
sería impropio decir que el conocedor se suicida, porque no tiene la
^{intención} de morir.

De manera que sólo los sujetos podemos suicidarnos. El ^{suicidio} es
contradictorio con la naturaleza viva del ^{sujeto}, pero, precisamente
por ser suceso tan extremado, sirve para exhibir la gran flexibilidad
del sujeto, y su enorme peligro.


\Section La vida

El ^{problema de la supervivencia}, abreviadamente vida, nos da sentido
y significados_{significado} porque los hombres somos sujetos vivos.
Vivos porque somos resolutores, precisamente, del problema aparente de
la supervivencia, y no de otro. Y sujetos porque, enfrentados a un
problema aparente, somos resolutores capaces de plantearnos el problema
al que nos enfrentamos en nuestra ^{lógica simbólica} con ^{sintaxis}
recursiva. En resumen: por ser sujetos vivos somos parte de la vida y
somos responsables de su futuro, aunque no le seamos imprescindibles.

Y la ^|vida| es un ^{problema aparente}. A esta definición de vida, que
es más un postulado que una definición, véase _>El bucle subjetivo>,
sólo se puede agregar, o bien información redundante, como que de la
vida nada se sabe excepto que es un problema, o bien información
circunstancial relativa a su resolución, por ejemplo información
histórica sobre la ^{evolución} darviniana de las especies aprovechando
ciertos procesos que estudia la ^{química orgánica}.

Y el problema aparente, ya lo hemos visto, es solamente ^{libertad} y
condición. La condición es la que distingue la vida de la muerte. La
libertad es \dots, aquí quería llegar yo.


\Section La libertad

La ^|libertad| es una de las dos partes que constituyen todo
^{problema}. No hay problema sin libertad, ni libertad sin problema.
Porque sin libertad hay necesidad y acaso ^{azar}; hay ^{fatalidad} mas
no hay problema. Pero tampoco hay libertad sin límite, sin condición, y
libertad con condición es problema. Si la libertad fuera completa, no
habría ^{deseo} sino satisfacción. Y, en habiendo total satisfacción, no
podría haber problema, que es su contrario. La libertad y el problema
son inseparables.

El problema es libertad y condición. Y cuando el resolutor del problema
es complejo, ha de emplear significados que integren las condiciones
externas con sus necesidades o condiciones internas. Por esta razón no
hay ^{significado} sin problema, y como tampoco hay problema sin
libertad, resulta que para que haya significado ha de haber libertad. El
significado es imposible sin la libertad.

Pero el significado está con las condiciones y la libertad es,
precisamente, la otra parte del problema. La libertad es un concepto
necesariamente, aún más, tautológicamente libre de significado. Por esta
razón la ^{semántica} es insuficiente y es preciso un ^{simbolismo}, con
semántica y ^{sintaxis} recursiva, para poder representar la libertad.
La libertad es un concepto sintáctico. La libertad y la lógica simbólica
son inseparables.

Dada la esencia problemática de la ^{vida}, la libertad es parte
inseparable de la vida que nos da sentido y que es la fuente de todo
significado. El problema aparente de la supervivencia es el problema, y
todos los demás problemas se derivan de él, son subproblemas de él. Como
la libertad es limitada y sólo habita en problemas, resulta que toda la
libertad se deriva del problema de la supervivencia. La libertad y la
vida son inseparables.

La resolución del problema de la supervivencia es un proceso evolutivo
que culmina en el ^{sujeto}, que es un resolutor capaz de representarse
el problema afrontado y a sí mismo. El sujeto dispone de una ^{lógica
simbólica} que le permite ser consciente del problema que afronta y
autoconsciente de ser él mismo un resolutor. El sujeto se sabe vivo, o
sea, el sujeto se sabe libre y mortal. El sujeto se define en relación
al problema que le da significado: yo soy libertad para no morir. La
libertad y el sujeto son inseparables.

El hombre, \latin{homo sapiens}, es el único sujeto vivo. Saberse libre
y parte de la vida impone al sujeto responsabilidades hacia la vida, es
decir, hace ^{persona} al sujeto. Sobre todo porque, dada su posibilidad
de superar la evolución darviniana, la persona es libre para modificar
drásticamente las condiciones de la vida, y con ellas el problema de la
supervivencia completo. La libertad y la ^{ética} son inseparables.


\Section La ciencia subjetiva

La filosofía natural occidental, o ^{ciencia}, no encuentra acomodo a la
^{libertad} y, por esta razón, no puede estudiar propiamente al hombre.
Y, por la misma causa, tampoco puede comprender la ^{ética}, ni el
^{sujeto}, ni la ^{vida}, ni el ^{simbolismo}, ni el ^{significado}, ni
el ^{problema}. Es necesaria una ciencia subjetiva. Lo que proponemos,
con esta ^{teoría de la subjetividad}, es superar la ^{explicación
material} y la ^{explicación automática} con la ^{explicación
problemática}.

\breakif2

La dificultad de la ciencia surge porque se limita a estudiar la
^{realidad} física, en la que no cabe la libertad. Lo físico y real es
lo que la lógica antigua del sujeto, por otro nombre ^{semántica}, es
capaz de representar. Por esto, las cosas que vemos merced a la
percepción son físicas y reales. La realidad tiene a su favor que
acumula la experiencia de millones de años. Pero, así como tener dos
pies o cinco dedos no supone que el exterior sea, de algún modo, bípedo
ni pentadactilar, que veamos cosas reales tampoco supone que el exterior
sea, de algún modo, real. En los tres casos la situación presente
depende de decisiones azarosas, pero reforzadas por su éxito inicial y
entonces establecidas irreversiblemente, que la evolución tomó hace
millones de años en circunstancias que, probablemente, ya no están
vigentes.

Lo teórico tiene, en comparación con lo real, muy poca experiencia,
apenas unos miles de años. Aún así, la ^{sintaxis}, que es la lógica
nueva del sujeto en la que se expresan las teorías y sus conceptos, es
la manera que la evolución ha encontrado para revisar y ampliar la
realidad más allá de la semántica. Además, recordemos, lo peculiar del
hombre es, precisamente, la sintaxis recursiva que completa su lógica
simbólica para que pueda representar problemas_{problema}, resoluciones
y soluciones.

La restricción que limita la ciencia al estudio de la realidad física se
queda sin fundamento, ya que tanto las cosas reales como los conceptos
teóricos son representaciones, y su diferencia es meramente histórica.
Vemos las cosas reales y no vemos los conceptos teóricos a causa de
nuestra constitución cognitiva, que es el resultado de nuestra historia
evolutiva. Solamente en una ciencia subjetiva que estudie el mundo
completo, es decir, tanto las cosas como los conceptos, puede caber la
^{libertad}. Porque, repito, estamos construidos de tal modo que no
podemos ver la libertad, y por esto decimos que la libertad es un
concepto, y no una cosa.

Una consecuencia de que la libertad no sea real es que, aunque
construyamos un ^{robot} libre, que ha de ser un resolutor simbólico, o
sea, un sujeto enfrentado a un ^{problema aparente}, no podremos ver en
donde hemos puesto la libertad, sencillamente porque la libertad es un
concepto y los conceptos son inasequibles para la ^{percepción}. Y, por
la misma causa, tampoco encontrarán los neurólogos_{neurología} la
libertad en el ^{cerebro} humano; no es que no esté, es que no se ve.
Sin embargo, en estos casos, la importancia de los conceptos invisibles
es mayor que la importancia de las cosas visibles, porque no es posible
entender a un sujeto sin comprender que es libre.


\Section La emancipación del sujeto

Como señala bellamente ^[Thiebaut]^(Thiebaut1990), la mayor revolución
de la ^{historia} es la ^{emancipación} del yo.
^[Descartes]^(Descartes1641) marca en la ^{filosofía} el momento en el
que el yo empieza a independizarse, proceso que, a pesar del tiempo
transcurrido, todavía está sin rematar. Porque la concepción del mundo
como un ^{mecanismo} que se rige por ^{leyes universales} remite a la
autoridad de ^{Dios}. Luego el ^{sujeto} sólo dejará de estar sujeto
cuando reconozca que es libre porque el ^{mundo} es un problema, como
propugna la ciencia subjetiva, y no un orden impuesto.


\Section ¡Arriba el subjetivismo!

La ^{historia} de lo ocurrido hasta ahora se puede compendiar en cuatro
jalones, que comienzan en el decimoséptimo.

\beginpoints
\siglo xvii ^[Descartes] estableció en el siglo {\sc xvii} los
fundamentos de la ^{filosofía} moderna: lo único indudable es el ^{yo}.
Además señaló la naturaleza mutuamente irreductible de la ^{realidad}
frente a la ^{libertad} del yo, que resolvió con un ^{dualismo}
ontológico.
\siglo xviii ^[Kant] advirtió que, previo al entender, es preciso un
aparato para entender, que aquí denominamos ^{lógica}. Pero la lógica
kantiana, aunque era capaz de representar la realidad, no podía
representar la libertad.
\siglo xix ^[Darwin] postuló que el ^{hombre}, \latin{homo sapiens}, es un
producto de la ^{evolución} de las especies. En consecuencia, también lo
ha de ser su lógica y su yo. E incluso su libertad. Desde entonces no se
puede entender al hombre sin entender la vida.
\siglo xx ^[Turing] inventó, siguiendo a ^[Gödel], el ^{motor sintáctico},
dilucidando, de paso, los simbolismos. Un ^{simbolismo} es una ^{lógica
gramatical} y recursiva, o sea, una lógica de ^{expresividad} máxima y
dividida en dos capas: ^{semántica} y ^{sintaxis}.

\vfill\break

\siglo xxi Y ahora nos queda la tarea de integrar los
descubrimientos de los cuatro siglos anteriores.

\noindent
Para ello sólo hay que sustituir la lógica de ^[Kant] por un simbolismo,
con lo que el dualismo ontológico de ^[Descartes] se transforma en un
dualismo lógico. No hay dos sustancias, sino dos tipos de
representaciones lógicas: los objetos_{objeto} semánticos, que son las
cosas_{cosa} reales que se ven sin necesidad de pensar, y los objetos
sintácticos, que son los conceptos_{concepto} teóricos que hay que
pensar pero que no se ven.
\endpoints

La explicación del dualismo lógico es histórica y contingente, es decir,
es darviniana. Así que, para adecuarnos a ^[Darwin], hemos de mostrar
que el simbolismo mejora las posibilidades de supervivencia. Y esto es
así si postulamos que la ^{vida} es un ^{problema aparente}, o sea,
exclusivamente libertad y la condición de no morir. Porque, para que un
resolutor de un problema aparente pueda comprender la situación
completa, que incluye al problema con su libertad y a él mismo como
resolución, su lógica ha de ser simbólica.

Un sujeto simbólico así definido se verá a sí mismo enfrentado al
^{problema de la supervivencia}, es decir, libre aunque con la condición
de no morir. Y su libertad será tan genuina como la del problema de la
supervivencia. O sea, completamente genuina si aceptamos que la vida es
problemática y absoluta.


\Section La libertad nunca es completa

Admitir que la ^{libertad} es un concepto científico básico exige
admitir que el ^{problema} también lo es. Y, una vez bajo la disciplina
de la ^{ciencia}, la libertad, con cuya infinitud podían soñar los
románticos, queda necesariamente limitada. Limitación que se contagia al
^{significado} y, desde éste, al conocimiento.

El ^{conocimiento} no es absoluto, depende del ^{sujeto} que, a su vez,
toma los significados del problema del que es resolutor. El límite de la
libertad, el límite del significado y el límite del conocimiento es el
mismo, a saber, el ^{problema aparente} de la supervivencia, esto es, la
vida. La ^{vida} es una burbuja de conocimiento y de libertad. La
^{muerte} no tiene significado.

Al final, ni siquiera la libertad es un ^{concepto transcendente}.
Ningún concepto transciende la muerte. La libertad es, sin embargo, uno
de los conceptos fundamentales, porque me define. Yo soy libertad para
no morir.


\Section ¿Por qué buscamos la libertad?

¿Por qué buscamos la ^{libertad}, si nos supone tener problemas? Porque
nos gusta avanzar en la resolución de los problemas y, todavía más,
solucionarlos. Nos hace felices porque estamos diseñados para solucionar
problemas. Pero, para solucionar un problema, es menester haber un
problema. Por esta razón somos curiosos, inquisitivos, y, por esa misma
causa, buscamos la libertad que implica tener problemas y muchas maneras
distintas de resolverlos, y huimos del tedio que supone la actuación
mecánica y repetitiva, que puede ser efectiva, pero que nunca es
problemática.


\Section El conocimiento no es absoluto

La naturaleza problemática y paradójica de la ^{vida} y del ^{sujeto}
limitan el ^{conocimiento}. Esto puede parecer un inconveniente, pero
creer lo contrario es un error. Y, por otra parte, que el conocimiento
no es absoluto, sino que depende del sujeto que, a su vez, no es más que
un resolutor de un problema aparente, generaliza otros dos principios
científicos: que el ^{espacio} no es absoluto, propuesto por ^[Galileo],
y que el ^{tiempo} no es absoluto, establecido por ^[Einstein].

Además, entender que la vida es un ^{problema aparente} permite
integrar la ^{evolución} darviniana y la cultural_{evolución cultural}
en un único proceso en cuya bisagra está el sujeto. Sujeto que queda
definido, y disfruta de tan ventajosa posición, porque su lógica es
simbólica_{lógica simbólica}, de suerte que es consciente del problema
que enfrenta y autoconsciente_{autoconsciencia} de ser un resolutor del
mismo. Es decir, que la naturaleza problemática de la vida explica por
qué el lenguaje simbólico, la ^{cultura}, la ^{técnica}, la
^{consciencia}, la ^{ética} y la ^{libertad} coinciden en el hombre, que
es el único sujeto vivo.

\vskip1\baselineskip
\centerline{\fonttwo Fin}

\endinput
